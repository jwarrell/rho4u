\section{Type system for rholang}
\subsection{$\lambda$-calculus : ML :: rho-calculus : rholang }\label{sec:concurrent_process_calculi_and_spatial_logics_} % (fold) 
In the last thirty years the process calculi have matured into a
remarkably powerful analytic tool for reasoning about concurrent and
distributed systems. Process-calculus-based algebraical specification of
processes began with Milner's Calculus for Communicating Systems (CCS)
\cite{DBLP:books/sp/Milner80} and Hoare's Communicating Sequential Processes
(CSP) \cite{DBLP:books/ph/Hoare85}, and continue
through the development of the so-called mobile process calculi,
e.g. Milner, Parrow and Walker's $\pi$-calculus \cite{DBLP:journals/iandc/MilnerPW92a}, \cite{DBLP:journals/iandc/MilnerPW92b},
Cardelli and Caires's spatial logic \cite{DBLP:conf/fossacs/Caires04} \cite{DBLP:journals/iandc/CairesC03} \cite{DBLP:journals/tcs/CairesC04}, or Meredith and Radestock's reflective calculi
\cite{DBLP:journals/entcs/MeredithR05} \cite{DBLP:conf/tgc/MeredithR05}. The process-calculus-based
algebraical specification of processes has expanded its scope of
applicability to include the specification, analysis, simulation and
execution of processes in domains such as:

\begin{itemize}
\item telecommunications, networking, security and application level protocols
\cite{DBLP:conf/popl/AbadiB02} 
\cite{DBLP:journals/tcs/AbadiB03} 
\cite{DBLP:conf/epew/BrownLM05} 
\cite{DBLP:conf/fossacs/LaneveZ05}; 
\item programming language semantics and design
\cite{DBLP:conf/epew/BrownLM05}
\cite{djoin}
\cite{DBLP:conf/afp/FournetFMS02}
\cite{DBLP:journals/toplas/SewellWU10};
\item webservices
\cite{DBLP:conf/epew/BrownLM05}
\cite{DBLP:conf/fossacs/LaneveZ05}
\cite{DBLP:conf/wise/Meredith03};
\item{blockchain}
  \cite{meredith_2017}
\item and biological systems
\cite{DBLP:conf/cmsb/Cardelli04}
\cite{DBLP:conf/esop/DanosL03}
\cite{DBLP:conf/psb/RegevSS01}
\cite{DBLP:journals/ipl/PriamiRSS01}.
\end{itemize}

Among the many reasons for the continued success of this approach are
two central points. First, the process algebras provide a
compositional approach to the specification, analysis and execution of
concurrent and distributed systems. Owing to Milner's original
insights into computation as interaction \cite{DBLP:journals/cacm/Milner93}, the process
calculi are so organized that the behavior ---the semantics--- of a
system may be composed from the behavior of its components. This means that specifications can be constructed in
terms of components ---without a global view of the system--- and
assembled into increasingly complete descriptions.

The second central point is that process algebras have a potent proof
principle, yielding a wide range of effective and novel proof
techniques \cite{DBLP:conf/concur/SangiorgiM92} \cite{DBLP:conf/fmco/Sangiorgi05} \cite{DBLP:journals/toplas/Sangiorgi09}. In particular, \emph{bisimulation} encapsulates an effective
notion of process equivalence that has been used in applications as
far-ranging as algorithmic games semantics
\cite{DBLP:conf/sas/Abramsky05} and the construction of
model-checkers \cite{caires_2004}. The essential notion can be stated in
an intuitively recursive formulation: a \emph{bisimulation} between two
processes $P$ and $Q$ is an equivalence relation $E$ relating $P$
and $Q$ such that: whatever action of $P$ can be observed, taking it
to a new state $P'$, can be observed of $Q$, taking it to a new state
$Q'$, such that $P'$ is related to $Q'$ by $E$ and vice versa. $P$ and
$Q$ are \emph{bisimilar} if there is some bisimulation relating
them. Part of what makes this notion so robust and widely applicable
is that it is parameterized in the actions observable of processes
$P$ and $Q$, thus providing a framework for a broad range of
equivalences and up-to techniques \cite{DBLP:conf/concur/SangiorgiM92} all governed by the same core
principle \cite{DBLP:journals/toplas/Sangiorgi09}.

And yet, there is no mainstream language built from a calculus like
the rho-calculus in the same way that SML is built from the
$\lambda$-calculus. Until rholang.

\subsection{The syntax and semantics of the notation system}\label{sub:the_syntax_and_semantics_of_the_notation_system} % (fold)

We now summarize a technical presentation of the calculus that
embodies our theory of dynamics. The typical presentation of such a
calculus follows the style of giving generators and relations on
them. The grammar, below, describing term constructors, freely
generates the set of processes, $\Proc$. This set is then quotiented
by a relation known as structural congruence and it is over this set
that the notion of dynamics is expressed. This presentation is
essentially that of \cite{DBLP:journals/entcs/MeredithR05} with the addition of
polyadicity and summation. For readability we have relegated some of
the technical subtleties to an appendix.

\paragraph{Notational interlude} when it is clear that some expression $t$ is a sequence (such as a list or a vector), and $a$ is an object that might be meaningfully and safely prefixed to that sequence then we write $a:t$ for the sequence with $a$ prefixed (aka ``consed'') to $t$. We write $t(i)$ for the $i$th element of $t$.

\subsubsection{Process grammar}\label{subsub:process_grammar}

\begin{mathpar}
\inferrule* [lab=process] {} {P, Q \bc \pzero \;\bm\; \mathsf{for}(
  \arrvec{y} \leftarrow x )P \;\bm\; x\mathsf{!}(\arrvec{Q}) \;\bm\;
  \mathsf{*}x \;\bm\; P\mathsf{|}Q } \and \inferrule* [lab=name] {}
            {x, y \bc \mathsf{@}P }
\end{mathpar}

Note that $\arrvec{x}$ (resp. $\arrvec{P}$) denotes a vector of names
(resp. processes) of length $|\arrvec{x}|$ (resp. $|\arrvec{P}|$). We adopt
the following useful abbreviations.

\begin{mathpar}
  \Pi \arrvec{P} := \Pi_{i=1}^{|\arrvec{P}|}P_i := P_1 | \ldots | P_{|\arrvec{P}|}
\end{mathpar}

\begin{definition}
  \emph{Free and bound names} The calculation of the free names of a
  process, $P$, denoted $\freenames{P}$ is given recursively by
  
  \begin{mathpar}
    \freenames{\pzero} = \emptyset
    \and
    \freenames{\mathsf{for}(\arrvec{y} \leftarrow x)(P)} = \{ x \} \cup \freenames{P}\setminus\{\arrvec{y}\}
    \and
    \freenames{x!(\arrvec{P})} = \{ x \} \cup \freenames{\arrvec{P}}
    \and
    \freenames{P|Q} = \freenames{P} \cup \freenames{Q}
    \and
    \freenames{\mathsf{}{x}} = \{ x \} \\
  \end{mathpar}
  
  where $\{\arrvec{x}\} := \{x_1, \ldots, x_{|\arrvec{x}|}\}$ and $\freenames{\arrvec{P}} := \bigcup \freenames{P_i}$.
  
  An occurrence of $x$ in a process $P$ is \textit{bound} if it is not
  free. The set of names occurring in a process (bound or free) is
  denoted by $\names{P}$.
\end{definition}

\subsection{Substitution}

We use $\Proc$ for the set of processes, $\QProc$ for the set of
names, and $\id{\{}\arrvec{y} / \arrvec{x} \id{\}}$ to denote partial
maps, $s : \QProc \rightarrow \QProc$. A map, $s$ lifts, uniquely, to
a map on process terms, $\widehat{s} : \Proc \rightarrow
\Proc$. Historically, it is convention to use $\sigma$ to range over
lifted subsitutions, $\widehat{s}$, to write the application of a
substitution, $\sigma$ to a process, $P$, with the substitution on the
right, $P\sigma$, and the application of a substitution, $s$, to a
name, $x$, using standard function application notation, $s(x)$. In
this instance we choose not to swim against the tides of
history. Thus, 

\begin{definition}
  given $x = \quotep{P'}$, $u = \quotep{Q'}$, $s =
  \substn{u}{x}$ we define the lifting of $s$ to $\widehat{s}$ (written
  below as $\sigma$) recursively by the following equations.
  \begin{mathpar}
    0 \sigma := 0 \\
    (P \mathsf{|} Q) \sigma
    :=    
    P\sigma \mathsf{|} Q\sigma \\
    (\mathsf{for}(\arrvec{y} \leftarrow v)P) \sigma    
    :=
    \mathsf{for}(\arrvec{z} \leftarrow \sigma(v))((P \psubstn{\arrvec{z}}{\arrvec{y}}) \sigma) \\
    (\lift{x}{Q}) \sigma  
    :=
    \lift{\sigma(x)}{ Q \sigma } \\
    (\dropn{y})  \sigma       
    := 
    \left\{ 
      \begin{array}{ccc} 
        Q' & & y \;\nameeq\; x \\
        \dropn{y} & & otherwise \\
      \end{array}
      \right.
  \end{mathpar} 

  where

  \begin{eqnarray}
    \psubstp{Q}{P}(x) = \id{\{} \quotep{Q} / \quotep{P} \id{\}}(x) = 
    \left\{ 
      \begin{array}{ccc}
        \quotep{Q} & & x \;\nameeq\; \quotep{P} \\
        x & & otherwise \\
      \end{array}
      \right. \nonumber
  \end{eqnarray}
\end{definition}

and $z$ is chosen distinct from $\quotep{P}$, $\quotep{Q}$, the free
names in $Q$, and all the names in $R$. Our $\alpha$-equivalence will
be built in the standard way from this substitution.

\begin{definition}
Then two processes, $P,Q$, are alpha-equivalent if $P = Q\{\arrvec{y}/\arrvec{x}\}$ for
some $\arrvec{x} \in \boundnames{Q},\arrvec{y} \in \boundnames{P}$, where $Q\{\arrvec{y}/\arrvec{x}\}$
denotes the capture-avoiding substitution of $\arrvec{y}$ for $\arrvec{x}$ in $Q$.
\end{definition}

\begin{definition}
  The {\em structural congruence} $\equiv$
  between processes \cite{DBLP:books/daglib/0004377} is the least congruence containing
  alpha-equivalence and satisfying the commutative monoid laws
  (associativity, commutativity and $\pzero$ as identity) for parallel
  composition $|$.
\end{definition}

\begin{definition}
  The {\em name equivalence} $\nameeq$ is the least congruence
  satisfying these equations
  \begin{mathpar}
  \inferrule*[lab=Quote-drop] {}{ \quotep{\dropn{x}} \;\nameeq\; x }
  \and
  \inferrule*[lab=Struct-equiv] { P \;\scong\; Q } { \quotep{P} \;\nameeq\; \quotep{Q} }
  \end{mathpar}
\end{definition}

The astute reader will have noticed that the mutual recursion of names
and processes imposes a mutual recursion on alpha-equivalence and
structural equivalence via name-equivalence. Fortunately, all of this
works out pleasantly and we may calculate in the natural way, free of
concern. The reader interested in the details is referred to the
appendix \ref{appendix:rho_details}.

\begin{remark}\label{rem:no_self_referential_names}
  One particularly useful consequence of these definitions is that
  $\forall P. \quotep{P} \not\in \freenames{P}$. It gives us a
  succinct way to construct a name that is distinct from all the names
  in $P$ and hence fresh in the context of $P$. For those readers
  familiar with the work of Pitts and Gabbay, this consequence allows
  the system to completely obviate the need for a fresh operator, and
  likewise provides a canonical approach to the semantics of
  freshness.
\end{remark}

\subsection{Operational semantics}

Finally, we introduce the computational dynamics. What marks these
algebras as distinct from other more traditionally studied algebraic
structures, e.g. vector spaces or polynomial rings, is the manner in
which dynamics is captured. In traditional structures, dynamics is typically
expressed through morphisms between such structures, as in linear maps
between vector spaces or morphisms between rings. In algebras
associated with the semantics of computation, the dynamics is
expressed as part of the algebraic structure itself, through a
reduction reduction relation typically denoted by $\red$. Below, we
give a recursive presentation of this relation for the calculus used
in the encoding.

\begin{mathpar}
  \inferrule* [lab=COMM] {x_{t} \;\nameeq\; x_{s}, \;\;\; |\arrvec{y}| = |\arrvec{Q}|} {\mathsf{for}( \arrvec{y} \leftarrow x_{t} )P \;\mathsf{|}\; x_{s}!(\arrvec{Q})
  \red P\substn{\arrvec{\quotep{Q}}}{\arrvec{y}}}
  \and
  \inferrule* [lab=PAR]{P \red P'}{P\mathsf{|}Q \red P'\mathsf{|}Q}
  \and
  \inferrule* [lab=EQUIV]{{P \;\scong\; P'} \andalso {P' \red Q'} \andalso {Q' \;\scong\; Q}}{P \red Q}
\end{mathpar}

We write $P\red$ if $\exists Q $ such that $ P \red Q$ and $P\not\red$, otherwise.

%% \begin{definition}
%%   $\mathsf{COMM}$-events: when $P \red P'$ we can record the information
%%   justifying this reduction step with an expression of the form
%%   $\mathsf{COMM}_{(P,P')}(x_t,x_s,\sigma)$, leaving off the subscript
%%   $_{(P,P')}$ when the context makes the source ($P$) and target
%%   ($P'$) states clear. Likewise, if $c = \mathsf{COMM}_{(P,P')}(x_t,x_s,\sigma)$, we write $\mathsf{src}(c) = P$, $\mathsf{trgt}(c) = P'$, $\mathsf{src_{N}}(c) = x_{s}$, and $\mathsf{trgt_{N}}(c) = x_{t}$ and $P \stackrel{c}{\red} P'$.
%% \end{definition}

%% We use $\alpha, \beta, ...$ to range over reduction \emph{paths}. That is, if

%% \begin{mathpar}
%%   P_{1} \stackrel{\mathsf{COMM}(x_{t_{1}},x_{s_{1}},\sigma_{1})}{\red} P_{2} \stackrel{\mathsf{COMM}(x_{t_{2}},x_{s_{2}},\sigma_{2})}{\red} \cdots \stackrel{\mathsf{COMM}(x_{t_{n-1}},x_{s_{n-1}},\sigma_{n-1})}{\red} P_{n},
%% \end{mathpar}

%% then we may write

%% \begin{mathpar}
%%   \alpha = \mathsf{COMM}(x_{t_{1}},x_{s_{1}},\sigma_{1}):\mathsf{COMM}(x_{t_{2}},x_{s_{2}},\sigma_{2}):\ldots : \mathsf{COMM}(x_{t_{n-1}},x_{s_{n-1}},\sigma_{n-1}).
%% \end{mathpar}

%% \begin{definition}
%%   The set of paths between $P$ and $Q$, written $\mathsf{paths}(P,Q)$ is defined by $\mathsf{paths}(P,Q) := \{ c:\alpha \; : P \stackrel{c}{\red} P', \alpha \in \mathsf{paths}(P',Q) \}$ and $\epsilon$ denotes the empty path.
%% \end{definition}

\subsection{ Dynamic quote: an example }

Anticipating something of what's to come, let $z = \quotep{P}$, $u = \quotep{Q}$, and $x = \quotep{\lift{y}{\dropn{z}}}$. Now consider applying the substitution,
$\widehat{\id{\{}u / z \id{\}}}$, to the following pair of processes,
$\lift{w}{y!(\dropn{z})}$ and $\lift{w}{\dropn{x}} = \lift{w}{\dropn{\quotep{\lift{y}{\dropn{z}}}}}$.

\begin{eqnarray}
	\lift{w}{\lift{y}{\dropn{z}}}\widehat{\id{\{}u / z \id{\}}}
		& = &
		\lift{w}{\lift{y}{Q}} \nonumber\\
	\lift{w}{\dropn{x}} \widehat{ \id{\{}u / z \id{\}} }
		& = &
		\lift{w}{\dropn{x}} \nonumber
\end{eqnarray}

The body of the quoted process, $\quotep{\lift{y}{\dropn{z}}}$, is
impervious to substitution, thus we get radically different
answers. In fact, by examining the first process in an input context,
e.g. $\mathsf{for}(z \leftarrow x)\lift{w}{\lift{y}{\dropn{z}}}$, we see that the process
under the output operator may be shaped by prefixed inputs binding a
name inside it. In this sense, the combination of input prefix binding
and output operators will be seen as a way to dynamically construct
processes before reifying them as names.

\section{Replication}

As mentioned before, it is known that replication (and hence
recursion) can be implemented in a higher-order process algebra
\cite{DBLP:books/daglib/0004377}. As our first example of calculation with the
machinery thus far presented we give the construction explicitly in
the {\rhoc}.

\begin{eqnarray}
	D_{x} & := & \prefix{x}{y}{(\binpar{\outputp{x}{y}}{\dropn{y}})} \nonumber\\
	\bangp_{x}{P} & := & \binpar{\lift{x}{\binpar{D_{x}}{P}}}{D_{x}} \nonumber
\end{eqnarray}

\begin{eqnarray}
	\bangp_{x}{P} & & \nonumber\\
	=
	& \lift{x}{(\prefix{x}{y}{(\outputp{x}{y} | \dropn{y})) | P}} 
	      | \prefix{x}{y}{(\outputp{x}{y} | \dropn{y})} & \nonumber\\
	\red
	& (\outputp{x}{y} | \dropn{y})\substn{\quotep{(\prefix{x}{y}{(\dropn{y} | \outputp{x}{y})) | P}}}{y} & \nonumber\\
	=
	& \outputp{x}{\quotep{(\prefix{x}{y}{(\outputp{x}{y} | \dropn{y})) | P}}}
	  | {(\prefix{x}{y}{(\outputp{x}{y} | \dropn{y})) | P}} & \nonumber\\
	\red
	& \ldots & \nonumber\\
	\red^*
	& P | P | \ldots & \nonumber
\end{eqnarray}

Of course, this encoding, as an implementation, runs away, unfolding
$\bangp{P}$ eagerly. A lazier and more implementable replication
operator, restricted to input-guarded processes, may be obtained as follows.

\begin{eqnarray}
\bangp{\prefix{u}{v}{P}} 
	:= 
	\binpar{\lift{x}{\prefix{u}{v}{(\binpar{D(x)}{P})}}}{D(x)} \nonumber
\end{eqnarray}

\begin{remark}
  Note that the lazier definition still does not deal with summation
  or mixed summation (i.e. sums over input and output). The reader is
  invited to construct definitions of replication that deal with these
  features. 

  Further, the definitions are parameterized in a name, $x$. Can you,
  gentle reader, make a definition that eliminates this parameter and
  guarantees no accidental interaction between the replication
  machinery and the process being replicated -- i.e. no accidental
  sharing of names used by the process to get its work done and the
  name(s) used by the replication to effect copying. This latter
  revision of the definition of replication is crucial to obtaining
  the expected identity $!!P \sim !P$.
\end{remark}

\begin{remark}\label{rem:paradoxical_combinator}
  The reader familiar with the lambda calculus will have noticed the
  similarity between $D$ and the paradoxical combinator.

  [Ed. note: the existence of this seems to suggest we have to be more
  restrictive on the set of processes and names we admit if we are to
  support no-cloning.]
\end{remark}

\subsubsection{Bisimulation}

The computational dynamics gives rise to another kind of equivalence,
the equivalence of computational behavior. As previously mentioned
this is typically captured \emph{via} some form of bisimulation.

% The notion we use in this paper is weak barbed bisimulation
% \cite{milner91polyadicpi}.

The notion we use in this paper is derived from weak barbed
bisimulation \cite{milner91polyadicpi}. 

\begin{definition}
An \emph{observation relation}, $\downarrow_{\mathcal N}$, over a set
of names, $\mathcal N$, is the smallest relation satisfying the rules
below.

\infrule[Out-barb]{y \in {\mathcal N}, \; x \nameeq y}
		  {\outputp{x}{v} \downarrow_{\mathcal N} x}
\infrule[Par-barb]{\mbox{$P\downarrow_{\mathcal N} x$ or $Q\downarrow_{\mathcal N} x$}}
		  {\binpar{P}{Q} \downarrow_{\mathcal N} x}

We write $P \Downarrow_{\mathcal N} x$ if there is $Q$ such that 
$P \wred Q$ and $Q \downarrow_{\mathcal N} x$.
\end{definition}

\begin{definition}
%\label{def.bbisim}
An  ${\mathcal N}$-\emph{barbed bisimulation} over a set of names, ${\mathcal N}$, is a symmetric binary relation 
${\mathcal S}_{\mathcal N}$ between agents such that $P\rel{S}_{\mathcal N}Q$ implies:
\begin{enumerate}
\item If $P \red P'$ then $Q \wred Q'$ and $P'\rel{S}_{\mathcal N} Q'$.
\item If $P\downarrow_{\mathcal N} x$, then $Q\Downarrow_{\mathcal N} x$.
\end{enumerate}
$P$ is ${\mathcal N}$-barbed bisimilar to $Q$, written
$P \wbbisim_{\mathcal N} Q$, if $P \rel{S}_{\mathcal N} Q$ for some ${\mathcal N}$-barbed bisimulation ${\mathcal S}_{\mathcal N}$.
\end{definition}

\subsubsection{Contexts}

One of the principle advantages of computational calculi from the
$\lambda$-calculus to the $\pi$-calculus is a well-defined notion of context,
contextual-equivalence and a correlation between
contextual-equivalence and notions of bisimulation. The notion of
context allows the decomposition of a process into (sub-)process and
its syntactic environment, its context. Thus, a context may be
thought of as a process with a ``hole'' (written $\Box$) in it. The
application of a context $K$ to a process $P$, written $K[P]$, is
tantamount to filling the hole in $K$ with $P$. In this paper we do
not need the full weight of this theory, but do make use of the notion
of context in the proof the main theorem. 

\begin{mathpar}
\inferrule* [lab=context] {} {K \bc \Box \;\bm\; \mathsf{for}( \arrvec{y} \leftarrow x )K \;\bm\; x\mathsf{!}(\arrvec{P},K,\arrvec{Q}) \;\bm\; K\mathsf{|}P }
\end{mathpar}

\begin{definition}[contextual application] Given a context $K$, and
  process $P$, we define the \emph{contextual application}, $K[P] :=
  K\{P/\Box\}$. That is, the contextual application of K to P is the
  substitution of $P$ for $\Box$ in $K$.
\end{definition}

\begin{remark}
  Note that we can extend the definition of free and bound names to contexts.
\end{remark}

\subsection{Lifted types} 

In this section, for clarity, we illustrate the procedure for lifting
the rules of a specific rewrite system by way of
example. Specifically, we illustrate the procedure lifting the
rho-calculus theory to the $\mathsf{NT}(\mathsf{CCC})$ level.

\subsubsection{Lifted term constuctors}
Note that because of the lifted-singleton rule, it is not necessary to
give a typing for the $\pzero$ term.

\begin{mathpar}
  \inferrule* [lab=For-Comprehension] {}{y : \mathbf{\quotep{W}} : \mathbf{N}, \Gamma \vdash P : \mathbf{U} : \mathbf{P} \;\;\; \Delta \vdash x : \mathbf{V} : \mathbf{N} \Vdash \Gamma, \Delta \vdash \prefix{x}{y}{P} : \prefixt{V}{\quotep{W}}{U} : \mathbf{P}} \\
  \and
  \inferrule* [lab=Output] {}{\Gamma \vdash x : \mathbf{U} : \mathbf{N} \;\;\; \Delta \vdash Q : \mathbf{V} : \mathbf{P} \Vdash \Gamma, \Delta \vdash \outputp{x}{Q} : \outputt{U}{V} : \mathbf{P}} \\
  \and
  \inferrule* [lab=Parallel] {}{\Gamma \vdash P : \mathbf{U} : \mathbf{P} \;\;\; \Delta \vdash Q : \mathbf{V} : \mathbf{P} \Vdash \Gamma, \Delta \vdash {P}\mathsf{|}{Q} : \mathbf{U} \mathbf{|} \mathbf{V} : \mathbf{P}} \\
  \and
  \inferrule* [lab=Deref] {}{\Gamma \vdash x : \mathbf{U} : \mathbf{N} \Vdash \Gamma \vdash \dropn{x} : \dropt{U} : \mathbf{P}} \\
\inferrule* [lab=Quote] {}{\Gamma \vdash P : \mathbf{U} : \mathbf{P} \Vdash \Gamma \vdash \quotep{x} : \quotet{U} : \mathbf{P}} \\
\end{mathpar}

\subsubsection{Equations}
\begin{mathpar}
  \inferrule* [lab=ParMonoidId] {\Gamma \vdash P : \mathbf{U} : \mathbf{P}}{\Gamma \vdash \binpar{P}{0} = P : \mathbf{U} : \mathbf{P}} \\
  \and
  \inferrule* [lab=ParMonoidAssoc] {\Gamma \vdash \binpar{(\binpar{P}{Q})}{R} : \mathbf{U} : \mathbf{P}}{\Gamma \vdash \binpar{(\binpar{P}{Q})}{R} = \binpar{P}{(\binpar{Q}{R})} : \mathbf{U} : \mathbf{P}} \\
  \and
  \inferrule* [lab=ParMonoidComm] {\Gamma \vdash \binpar{P}{Q} : \mathbf{U} : \mathbf{P}}{\Gamma \vdash \binpar{P}{Q} = \binpar{Q}{P} : \mathbf{U} : \mathbf{P}}
\end{mathpar}

\subsubsection{Lifted redex constuctors}
\begin{mathpar}
  \inferrule* [lab=Comm] {\Gamma_{1} \vdash x : \mathbf{U} : \mathbf{N} \;\;\; y : \mathbf{\quotep{V}} : \mathbf{N}, \Gamma_{2} \vdash P : \mathbf{W} : \mathbf{P} \;\;\; \Gamma_{3} \vdash Q : \mathbf{V} : \mathbf{P}}{\Gamma_{1}, \Gamma_{2}, \Gamma_{3} \vdash \commr{x}{y}{P}{Q} : \commt{U}{V}{W}{V} : \mathbf{R}} \\
  \and  
  \inferrule* [lab=Eval] {}{\Gamma \vdash P : \mathbf{U} : \mathbf{P} \Vdash \Gamma \vdash \mathsf{eval}(P) : \mathbf{eval}(\mathbf{U}) : \mathbf{R}} \\
  \and
  \inferrule* [lab=ParL] {}{\Gamma \vdash R : \mathbf{E} : \mathbf{R} \;\;\; \Delta \vdash P : \mathbf{U} : \mathbf{P} \Vdash \Gamma, \Delta \vdash \mathsf{par}_{L}(R, P) : \binpartl{E}{U} : \mathbf{R}} \\
  \and
  \inferrule* [lab=ParR] {}{\Gamma \vdash R : \mathbf{E} : \mathbf{R} \;\;\; \Delta \vdash P : \mathbf{U} : \mathbf{P} \Vdash \Gamma, \Delta \vdash \mathsf{par}_{R}(P, R) : \binpartr{U}{E} : \mathbf{R}}
\end{mathpar}

\subsubsection{Reductions from the $\lambda$-theory}
\begin{mathpar}
  \inferrule* [lab=Comm-Src] {\Gamma \vdash \commr{x}{y}{P}{Q} : \commt{U}{V}{W}{V} : \mathbf{R}}{\Gamma \vdash \mathsf{src}(\commr{x}{y}{P}{Q}) = \prefix{x}{y}{P} \mathsf{|} \outputp{x}{Q} : \prefixt{U}{\quotep{V}}{W} \mathbf{|} \outputt{U}{V} : \mathbf{P}} \\
  \and
  \inferrule* [lab=Comm-Trgt] {\Gamma \vdash \commr{x}{y}{P}{Q} : \commt{U}{V}{W}{V} : \mathbf{R}}{\Gamma \vdash \mathsf{trgt}(\commr{x}{y}{P}{Q}) = P\substn{\quotep{Q}}{y} : \mathbf{W} : \mathbf{P}} \\
  \and
  \inferrule* [lab=Eval-Src] {}{\Gamma \vdash \mathsf{eval}(P) : \mathbf{eval}(\mathbf{U}) : \mathbf{R} \Vdash \Gamma \vdash \mathsf{src}(\mathsf{eval}(P)) = \dropn{\quotep{P}} : \dropt{\quotet{U}} : \mathbf{P}} \\
  \and
  \inferrule* [lab=Eval-Trgt] {}{\Gamma \vdash \mathsf{eval}(P) : \mathbf{eval}(\mathbf{U}) : \mathbf{R} \Vdash \Gamma \vdash \mathsf{trgt}(\mathsf{eval}(P)) = P : \mathbf{U} : \mathbf{P}} \\
  \and
  \inferrule* [lab=Par-Src] {\Gamma \vdash \binparx{R}{P} : \binpart{E}{U} : \mathbf{R}}{\Gamma \vdash \mathsf{src}(\binparx{R}{P}) = \binpar{\mathsf{src}(R)}{P}  : \binparpt{src(E)}{U} : \mathbf{P}} \\
  \and
  \inferrule* [lab=Par-Trgt] {\Gamma \vdash \binparx{R}{P} : \binpart{E}{U} : \mathbf{R}}{\Gamma \vdash \mathsf{trgt}(\binparx{R}{P}) = \binpar{\mathsf{trgt}(R)}{P} : \binparpt{trgt(E)}{U} : \mathbf{P}} \\
\end{mathpar}

\section
  The semantics implied by these rules in strictly interleaving. Effectively, it assumes there is only one thread executing in a non-deterministic fashion over the syntax. Below we present model which scales from single threaded to multi threaded using an attention resource.

% section rholang (end)
