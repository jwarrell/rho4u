\section{The algorithm}
\subsection{Notation}
$P, Q, R$ range over terms, while $p, q, r$ range over term variables,
$e$ ranging over terms and variables.  $\bold{U}, \bold{V}, \bold{W}$
range over filters $\bold{\{ Q \}}$ characteristic filter of process
$Q$.  $\bold{X}, \bold{Y}, \bold{Z}$ range over sorts.  $\bold{P}$ is
the sort of process terms.  $\bold{R}$ is the sort of reduction terms
(redexes).

Type assertions are of the form $e : \bold{U} : \bold{X}$ where
$\bold{X}$ may be considered a \emph{kind} and $\bold{U}$ is a type
interpreted as a \emph{filter} on the kind. Contexts are sequences of
type assertions. Judgments are written in more or less standard form
$\Gamma \vdash P : \bold{U} : \bold{X}$. Because there are rather a
bit more of them, inference rules are written in a somewhat
non-standard form. When space allows, we adopt a horizontal rather
than vertical syntax. Thus for an inference rule with judgments $J_{1}
\; \ldots \; J_{n}$ as the hypotheses, and $J$ as the consequent, we
write $J_{1} \; \ldots \; J_{n} \Vdash J$. When there are no
hypotheses, we write $\Vdash J$. When horizontal space is at a
premium, then we opt for the more traditional format

\begin{mathpar}
  \inferrule* [lab=Rule] {J_{1} \; \ldots \; J_{n}} {J}
\end{mathpar}

Note that the input is a $CCC$ presented as a $\lambda$-theory, and
hence also enjoys a judgment style presentation. Much of the machinery
below is simply lifting those judgments to the
$\mathsf{NT}(\mathsf{CCC})$ level. Likewise, it is possible to project
back down. We adopt the following notation to denote the projection:
$\Gamma_{\lambda} \vdash_{\lambda} e : \mathbf{X}$.

In general, we assume no ``heating'' rules in the rewrite
systems. That is, without loss of generality, every redex must be a
term built from a binary term constructor, with one of the terms
playing in the role of program and one playing in the role of
environment. For example, in $\lambda$-calculus application is the
binary term construction, the term in function position is the
program, and the term in argument position is environment. In the
rho-calculus parallel composition is the binary term constructor and
one of the subterms is program and the other is environment (it does
not matter which). Without loss of generality, we write $P \mathsf{|} Q$ for that binary term.

\subsection{Generic types}

\subsubsection{Boundaries}
\begin{mathpar}
  \inferrule* [lab=Axiom] {}{\Vdash p : \mathbf{U} : \mathbf {X} \vdash p : \mathbf{U} : \mathbf {X}} \\
  \and
  \inferrule* [lab=Character] {} {\Gamma \vdash Q : \mathbf{U} : \mathbf {X} \Vdash \Gamma \vdash Q : \mathbf{\{ Q \}} : \mathbf {X}}
  \and
  \inferrule* [lab=Top] {} {\Gamma \vdash P : \mathbf{U} : \mathbf {X} \Vdash \Gamma \vdash P : \mathbf{\top} : \mathbf {X}} \\
  \and
  \inferrule* [lab=Composition] {} {\Gamma \vdash P : \mathbf{U} : \mathbf {X} \;\;\; p : \mathbf{U} : \mathbf {X}, \Delta \vdash Q : \mathbf{V} : \mathbf {Y} \Vdash \Gamma, \Delta \vdash Q \substn{P}{p} : \mathbf{V} : \mathbf {Y}} \\
\end{mathpar}

\subsubsection{Disjunctions}
\begin{mathpar}
  \inferrule* [lab=Union] {}{\Gamma \vdash P : \mathbf{U} : \mathbf {X} \Vdash \Gamma \vdash P : \mathbf{U \vee V} : \mathbf {X}} \\
  \and
  \inferrule* [lab=InL,Right=(V:Y)] {} {\Gamma \vdash P : \mathbf{U} : \mathbf {X} \Vdash \Gamma \vdash \mathsf{in}_{L} P : \mathbf{U+V} : \mathbf {X + Y}}
  \and
  \inferrule* [lab=InR,Right=(U:X)] {} {\Gamma \vdash Q : \mathbf{V} : \mathbf {Y} \Vdash \Gamma \vdash \mathsf{in}_{R} Q : \mathbf{U+V} : \mathbf {X + Y}} \\
  \and
  \inferrule* [lab=Match-Case] {p : \mathbf{U} : \mathbf {X}, \Gamma \vdash P : \mathbf{W} : \mathbf {Z} \;\;\; q : \mathbf{V} : \mathbf {Y}, \Delta \vdash Q : \mathbf{W} : \mathbf {Z}} {r : \mathbf{U+V} : \mathbf{X+Y}, \Gamma, \Delta \vdash \mathsf{match} \; r \; \mathsf{case} \; \mathsf{in}_{L}\; p \Rightarrow P \mathsf{;} \; \mathsf{case} \; \mathsf{in}_{R} \; q \Rightarrow Q : \mathbf{W} : \mathbf {Z}}
\end{mathpar}

\subsubsection{Conjunctions}
\begin{mathpar}
  \inferrule* [lab=Intersection] {}{\Gamma \vdash P : \mathbf{U} : \mathbf {X} \;\;\; \Gamma \vdash P : \mathbf{V} : \mathbf {X} \Vdash \Gamma \vdash P : \mathbf{U \wedge V} : \mathbf {X}} \\
  \and
  \inferrule* [lab=Product] {}{\Gamma \vdash P : \mathbf{U} : \mathbf {X} \;\;\; \Delta \vdash Q : \mathbf{V} : \mathbf {Y} \Vdash \Gamma, \Delta \vdash P \times Q : \mathbf{U \times V} : \mathbf{X \times Y}} \\
  \and
  \inferrule* [lab=Proj1] {} {p : \mathbf{U} : \mathbf{X}, \Gamma \vdash P : \mathbf{W} : \mathbf {Z} \Vdash r : \mathbf{U} \times \mathbf{V} : \mathbf{X} \times \mathbf{Y}, \Gamma \vdash \mathsf{let} \; \langle p, \_ \rangle \; = \; r \; \mathsf{in}\; P : \mathbf{W} : \mathbf {Z}} \\
  \and
  \inferrule* [lab=Proj2] {} {q : \mathbf{V} : \mathbf{Y}, \Gamma \vdash Q : \mathbf{W} : \mathbf {Z} \Vdash r : \mathbf{U \times V} : \mathbf{X} \times \mathbf{Y}, \Gamma \vdash \mathsf{let} \; \langle \_, q \rangle \; = \; r \; \mathsf{in} \; Q : \mathbf{W} : \mathbf {Z}}
\end{mathpar}

\subsubsection{Implications}
\begin{mathpar}
  \inferrule* [lab=Implication1,Right=(U:X)] {}{\Gamma \vdash Q : \mathbf{V} : \mathbf {X} \Vdash \Gamma \vdash Q : \mathbf{U} \Rightarrow \mathbf{V} : \mathbf {X}}
  \and
  \inferrule* [lab=Implication2,Right=(U:X)] {}{\Gamma \vdash P : \mathbf{V} \Rightarrow \bot : \mathbf {X} \Vdash \Gamma \vdash P : \mathbf{U} \Rightarrow \mathbf{V} : \mathbf {X}} \\
  \and
  \inferrule* [lab=Separation,Right=(V:Y)] {}{\Gamma \vdash P : \mathbf{U} : \mathbf {X} \Vdash \Gamma \vdash P : \mathbf{V} \Rightarrow \bot : \mathbf {X+Y}} \\
  \and
  \inferrule* [lab=Abstraction] {}{p : \mathbf{U} : \mathbf {X}, \Gamma \vdash Q : \mathbf{V} : \mathbf {Y} \Vdash \Gamma \vdash \lambda p . Q : \mathbf{U}\to \mathbf{V} : \mathbf{X} \to \mathbf{Y}} \\
  \and
\inferrule* [lab=Application] {}{q : \mathbf{W} : \mathbf {Z}, \Gamma \vdash Q : \mathbf{V} : \mathbf {Y} \;\;\; \Delta \vdash P : \mathbf{U} : \mathbf{X} \Vdash \Gamma, r : \mathbf{U} \to \mathbf{W} : \mathbf{X} \to \mathbf{Z}, \Delta \vdash Q \substn{r(P)}{q} : \mathbf{V} : \mathbf{Y}}
\end{mathpar}

\subsubsection{Internalized operations}
\begin{mathpar}
  \inferrule* [lab=Pair] {}{\Gamma \vdash P : \mathbf{U} : \mathbf {P} \;\;\; \Delta \vdash Q : \mathbf{V} : \mathbf {P} \Vdash \Gamma,\Delta \vdash \langle P,Q \rangle : \langle \mathbf{U}, \mathbf{V} \rangle : \mathbf {P}} \\
  \and
  \inferrule* [lab=Fst] {}{\Vdash \; \vdash \pi_{1} : \{ \mathbf{\pi}_{1} \} : \mathbf {P}}
  \and
  \inferrule* [lab=Snd] {}{\Vdash \; \vdash \pi_{2} : \{ \mathbf{\pi}_{2} \} : \mathbf {P}} \\
  \and
  \inferrule* [lab=TagL,Right=(V:P)] {}{\Gamma \vdash P : \mathbf{U} : \mathbf {P} \Vdash \Gamma \vdash \mathsf{tag}_{L} \; P : \mathsf{U} \oplus \mathbf{V} : \mathbf {P}} \\
  \and
  \inferrule* [lab=TagR,Right=(V:P)] {}{\Gamma \vdash Q : \mathbf{V} : \mathbf {P} \Vdash \Gamma \vdash \mathsf{tag}_{R} \; Q : \mathsf{U} \oplus \mathbf{V} : \mathbf {P}} \\
  \inferrule* [lab=Sum] {}{\Gamma \vdash P : \mathbf{U} : \mathbf {P} \;\;\; \Delta \vdash Q : \mathbf{V} : \mathbf {P} \Vdash \Gamma,\Delta \vdash [ P,Q] : [ \mathbf{U}, \mathbf{V} ] : \mathbf {P}}
\end{mathpar}

\subsubsection{Internalized redex constuctors}
\begin{mathpar}
  \inferrule* [lab=MatchL-Redex] {\Gamma_{1} \vdash P_{1} : \mathbf{U}_{1} : \mathbf {P} \;\;\; \Gamma_{2} \vdash P_{2} : \mathbf{U}_{2} : \mathbf {P} \;\;\; \Gamma_{1} \vdash P_{3} : \mathbf{U}_{3} : \mathbf {P}}{\Gamma_{1} ,\Gamma_{2}, \Gamma_{3} \vdash \mathsf{match}_{L}(P_{1},P_{2},P_{3}) : \mathbf{match}_{L}(P_{1},P_{2},P_{3}) : \mathbf {R}} \\
  \and
  \inferrule* [lab=MatchR-Redex] {\Gamma_{1} \vdash P_{1} : \mathbf{U}_{1} : \mathbf {P} \;\;\; \Gamma_{2} \vdash P_{2} : \mathbf{U}_{2} : \mathbf {P} \;\;\; \Gamma_{1} \vdash P_{3} : \mathbf{U}_{3} : \mathbf {P}}{\Gamma_{1} ,\Gamma_{2}, \Gamma_{3} \vdash \mathsf{match}_{R}(P_{1},P_{2},P_{3}) : \mathbf{match}_{R}(P_{1},P_{2},P_{3}) : \mathbf {R}} \\
  \inferrule* [lab=Proj1-Redex] {}{\Gamma_{1} \vdash P_{1} : \mathbf{U}_{1} : \mathbf {P} \;\;\; \Gamma_{2} \vdash P_{2} : \mathbf{U}_{2} : \mathbf {P} \Vdash \Gamma_{1} ,\Gamma_{2} \vdash \mathsf{proj}_{1}(P_{1},P_{2}) : \mathbf{proj}_{1}(P_{1},P_{2}) : \mathbf {R}} \\
  \and
  \inferrule* [lab=Proj2-Redex] {}{\Gamma_{1} \vdash P_{1} : \mathbf{U}_{1} : \mathbf {P} \;\;\; \Gamma_{2} \vdash P_{2} : \mathbf{U}_{2} : \mathbf {P} \Vdash \Gamma_{1} ,\Gamma_{2} \vdash \mathsf{proj}_{2}(P_{1},P_{2}) : \mathbf{proj}_{2}(P_{1},P_{2}) : \mathbf {R}}
\end{mathpar}

\subsubsection{Equations from $\mathsf{NT(CCC)}$}
\begin{mathpar}
  \inferrule* [lab=MatchL-Eqn] {\Gamma \vdash R : \mathbf{U} : \mathbf {X} \;\;\; p : \mathbf{U} : \mathbf{X}, \Delta_{1} \vdash P : \mathbf{W} : \mathbf{Z} \;\;\; q : \mathbf{V} : \mathbf{Y}, \Delta_{2} \vdash Q : \mathbf{W} : \mathbf{Z}}{\Gamma,\Delta_{1}, \Delta_{2} \vdash (\mathsf{match} \; \mathsf{in}_{L}\; R \; \mathsf{case} \; \mathsf{in}_{L} \; p \Rightarrow P \mathsf{;} \; \mathsf{case} \; \mathsf{in}_{R} \; q \Rightarrow Q) = P\substn{R}{p} : \mathbf{W} : \mathbf {Z}} \\
  \and
  \inferrule* [lab=MatchR-Eqn] {\Gamma \vdash R : \mathbf{V} : \mathbf {Y} \;\;\; p : \mathbf{U} : \mathbf{X}, \Delta_{1} \vdash P : \mathbf{W} : \mathbf{Z} \;\;\; q : \mathbf{V} : \mathbf{Y}, \Delta_{2} \vdash Q : \mathbf{W} : \mathbf{Z}}{\Gamma,\Delta_{1}, \Delta_{2} \vdash (\mathsf{match} \; \mathsf{in}_{R}\; R \; \mathsf{case} \; \mathsf{in}_{L} \; p \Rightarrow P \mathsf{;} \; \mathsf{case} \; \mathsf{in}_{R} \; q \Rightarrow Q) = Q\substn{R}{q} : \mathbf{W} : \mathbf {Z}} \\
  \and
  \inferrule* [lab=Proj1-Eqn] {p : \mathbf{U} : \mathbf{X}, \Gamma \vdash P : \mathbf{W} : \mathbf {Z} \;\;\; \Delta \vdash Q_{1} \times Q_{2} : \mathbf{U} \times \mathbf{V} : \mathbf{X} \times \mathbf{Y}}{ \Gamma, \Delta \vdash (\mathsf{let} \; \langle p, \_ \rangle \; = \; Q_{1} \times Q_{2} \; \mathsf{in}\; P) = P\substn{Q_{1}}{p} : \mathbf{W} : \mathbf {Z}} \\
  \and
  \inferrule* [lab=Proj2-Eqn] {p : \mathbf{V} : \mathbf{Y}, \Gamma \vdash P : \mathbf{W} : \mathbf {Z} \;\;\; \Delta \vdash Q_{1} \times Q_{2} : \mathbf{U} \times \mathbf{V} : \mathbf{X} \times \mathbf{Y}}{ \Gamma, \Delta \vdash (\mathsf{let} \; \langle \_, p \rangle \; = \; Q_{1} \times Q_{2} \; \mathsf{in}\; P) = P\substn{Q_{2}}{p} : \mathbf{W} : \mathbf {Z}} \\
\end{mathpar}

\subsubsection{Liftings from the $\lambda$-theory}
\begin{mathpar}
  \inferrule* [lab=Lifted-Singleton] {}{\Gamma_{\lambda} \vdash_{\lambda} P : \mathbf{X} \Vdash \Gamma \vdash P : \mathbf{ \{ P \} } : \mathbf{X}} \\
  \and
  \inferrule* [lab=Swap] {}{\Gamma \vdash P : \mathbf{U} : \mathbf{X} \;\;\; \Gamma_{\lambda} \vdash_{\lambda} P = Q : \mathbf{X} \Vdash \Gamma \vdash Q : \mathbf{U} : \mathbf{X}} \\
  \and
  \inferrule* [lab=Lifted-Eqn] {}{\Gamma \vdash P : \mathbf{U} : \mathbf{X} \;\;\; \Gamma \vdash Q : \mathbf{U} : \mathbf{X} \;\;\; \Gamma_{\lambda} \vdash_{\lambda} P = Q : \mathbf{X} \Vdash \Gamma \vdash P = Q : \mathbf{U} : \mathbf{X}}
\end{mathpar}

\subsubsection{Reductions from $\mathsf{NT(CCC)}$}
\begin{mathpar}
  \inferrule* [lab=MatchL-Src] {\Gamma \vdash \mathsf{match}_{L}(P_{1},P_{2},P_{3}) : \mathbf{match}_{L}(\mathbf{U}_{1},\mathbf{U}_{2},\mathbf{U}_{3}) : \mathbf {R}}{\Gamma \vdash \mathsf{src}(\mathsf{match}_{L}(P_{1},P_{2},P_{3})) = \mathsf{tag}_{L}\; P_{1} \; \mathsf{|} \; [P_{2},P_{3}] : \mathbf{U}_{1} \oplus \mathbf{V} \; \mathsf{|} \; [\mathbf{U}_{2},\mathbf{U}_{3}] : \mathbf {P}} \\
  \and
  \inferrule* [lab=MatchL-Trgt] {\Gamma \vdash \mathsf{match}_{L}(P_{1},P_{2},P_{3}) : \mathbf{match}_{L}(\mathbf{U}_{1},\mathbf{U}_{2},\mathbf{U}_{3}) : \mathbf {R}}{\Gamma \vdash \mathsf{trgt}(\mathsf{match}_{L}(P_{1},P_{2},P_{3})) = P_{1} \; \mathsf{|} \; P_{2} : \mathbf{U}_{1} \; \mathsf{|} \; \mathbf{U}_{2} : \mathbf {P}} \\
  \and
  \inferrule* [lab=MatchR-Src] {\Gamma \vdash \mathsf{match}_{R}(P_{1},P_{2},P_{3}) : \mathbf{match}_{R}(\mathbf{U}_{1},\mathbf{U}_{2},\mathbf{U}_{3}) : \mathbf {R}}{\Gamma \vdash \mathsf{src}(\mathsf{match}_{R}(P_{1},P_{2},P_{3})) = \mathsf{tag}_{R}\; P_{1} \; \mathsf{|} \; [P_{2},P_{3}] : \mathbf{U}_{1} \oplus \mathbf{V} \; \mathsf{|} \; [\mathbf{U}_{2},\mathbf{U}_{3}] : \mathbf {P}} \\
  \and
  \inferrule* [lab=MatchR-Trgt] {\Gamma \vdash \mathsf{match}_{R}(P_{1},P_{2},P_{3}) : \mathbf{match}_{R}(\mathbf{U}_{1},\mathbf{U}_{2},\mathbf{U}_{3}) : \mathbf {R}}{\Gamma \vdash \mathsf{trgt}(\mathsf{match}_{R}(P_{1},P_{2},P_{3})) = P_{1} \; \mathsf{|} \; P_{2} : \mathbf{U}_{1} \; \mathsf{|} \; \mathbf{U}_{2} : \mathbf {P}}
\end{mathpar}

Put in proj reductions.

\subsubsection{Modal operators}
\begin{mathpar}
  \inferrule* [lab=Step-Future] {}{\Gamma \vdash R : \mathbf{E} : \mathbf {R} \Vdash \Gamma \vdash \mathsf{src}(R) : \lozenge \mathsf{trgt}(\mathbf{E}) : \mathbf {P}}
  \and
  \inferrule* [lab=Step-History] {}{\Gamma \vdash R : \mathbf{E} : \mathbf {R} \Vdash \Gamma \vdash \mathsf{trgt}(R) : \blacklozenge \mathsf{src}(\mathbf{E}) : \mathbf {P}}
\end{mathpar}
