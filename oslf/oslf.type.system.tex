\section{The algorithm}
\subsection{Notation}
$P, Q, R$ range over terms, while $p, q, r$ range over term variables,
$e$ ranging over terms and variables.  $\bold{U}, \bold{V}, \bold{W}$
range over filters $\bold{\{ Q \}}$ characteristic filter of process
$Q$.  $\bold{X}, \bold{Y}, \bold{Z}$ range over sorts.  $\bold{P}$ is
the sort of process terms.  $\bold{R}$ is the sort of reduction terms
(redexes).

Type assertions are of the form $e : \bold{U} : \bold{X}$ where
$\bold{X}$ may be considered a \emph{kind} and $\bold{U}$ is a type
interpreted as a \emph{filter} on the kind. Contexts are sequences of
type assertions. Judgments are written in more or less standard form
$\Gamma \vdash P : \bold{U} : \bold{X}$. 

Note that the input is a $CCC$ presented as a $\lambda$-theory, and
hence also enjoys a judgment style presentation. Much of the machinery
below is simply lifting those judgments to the
$\mathsf{NT}(\mathsf{CCC})$ level. Likewise, it is possible to project
back down. We adopt the following notation to denote the projection:
$\Gamma_{\lambda} \vdash_{\lambda} e : \mathbf{X}$.

In general, we assume no ``heating'' rules in the rewrite
systems. That is, without loss of generality, every redex must be a
term built from a binary term constructor, with one of the terms
playing in the role of program and one playing in the role of
environment. For example, in $\lambda$-calculus application is the
binary term construction, the term in function position is the
program, and the term in argument position is environment. In the
rho-calculus parallel composition is the binary term constructor and
one of the subterms is program and the other is environment (it does
not matter which). Without loss of generality, we write $P \odot Q$
for that binary term.

\subsection{Generic types}

We have a bit of a chicken and egg situation in that to understand the typing rules we need to have their intended denotations in mind.

\paragraph{Denotation}
In general, the denotation of inference rules will be a natural or dinatural transformation from functors whose source and target are products of copies of the Grothedieck category $\mathsf{NT}$. More specifically, since rules are of the form

\begin{mathpar}
  \inferrule* [lab=Rule]{\Gamma_{1} \vdash P_{1} : \mathbf{U}_{1} : \mathbf{X}_{1} \;\ldots\; \Gamma_{n} \vdash P_{n} : \mathbf{U}_{n} : \mathbf{X}_{n}}{K_{\Gamma}(\Gamma_{1},\ldots,\Gamma_{n}) \vdash K_{P}(P_{1},\ldots,P_{n}) : K_{\mathbf{U}}(\mathbf{U}_{1},\ldots,\mathbf{U}_{n}) : K_{\mathbf{X}}(\mathbf{X}_{1},\ldots,\mathbf{X}_{n})}
\end{mathpar}

then the denotation of such a rule is

$$\begin{array}{l}
  \meaningof{\mathsf{Rule}} : \meaningof{P_{1}} : \meaningof{\Gamma_{1}} \to \meaningof{\mathbf{U}_{1} : \mathbf{X}_{1}} \times \cdots \times \meaningof{P_{n}} : \meaningof{\Gamma_{n}} \to \meaningof{\mathbf{U}_{n} : \mathbf{X}_{n}} \\\Rightarrow\\ \meaningof{K_{P}(P_{1},\ldots,P_{n})} : \meaningof{K_{\Gamma}(\Gamma_{1},\ldots,\Gamma_{n})} \to \meaningof{K_{\mathbf{U}}(\mathbf{U}_{1},\ldots,\mathbf{U}_{n}) : K_{\mathbf{X}}(\mathbf{X}_{1},\ldots,\mathbf{X}_{n})}
\end{array}$$

Note that $P_{i}$ and likewise $\mathbf{U_{i}}$, $\mathbf{X_{i}}$, and
even $\Gamma_{i}$ are effectively meta-variables, rendering the
meaning of any judgment in the hypothesis of an inference rule as the
hom-functor of $\mathsf{NT}$. That is, $\meaningof{P_{i}} : \meaningof{\Gamma_{i}} \to \meaningof{\mathbf{U}_{i} : \mathbf{X}_{i}} = \mathsf{NT}(-_{1},-_{2})$ because barring any further information regarding the value of these meta-variables $\meaningof{P_{i}} : \meaningof{\Gamma_{i}} \to \meaningof{\mathbf{U}_{i} : \mathbf{X}_{i}}$ could be any morphism in $\mathsf{NT}(\meaningof{\Gamma_{i}},\meaningof{\mathbf{U}_{i} : \mathbf{X}_{i}})$; and since neither $\Gamma_{i}$ nor $\mathbf{U}_{i} : \mathbf{X}_{i}$ are specified it could be any morphism in $\mathsf{NT}(-_{1},-_{2})$. Thus, if we let $\mathsf{NT}(-_{1}, -_{2})^n = \Pi_{i}^{n}\mathsf{NT}(-_{1}, -_{2})$ the meaning of an inference rule is a (di-)natural transformation from $\mathsf{NT}(-_{1}, -_{2})^{n} \Rightarrow \mathsf{NT}(-_{1}, -_{2}) \circ k_{1} \times k_{2}$, constructing the consequent of the rule.

\subsubsection{Boundaries}
\begin{mathpar}
  \inferrule* [lab=Axiom] {}{\epsilon \Vdash p : \mathbf{U} : \mathbf {X} \vdash p : \mathbf{U} : \mathbf {X}} \\
  \and
  \inferrule* [lab=Character] {} {\Gamma \vdash Q : \mathbf{U} : \mathbf {X} \Vdash \Gamma \vdash \mathsf{Ch}(Q) : \mathbf{\{ Q \}} : \mathbf {X}}
  \and
  \inferrule* [lab=Top] {} {\Gamma \vdash P : \mathbf{U} : \mathbf {X} \Vdash \Gamma \vdash P : \mathbf{\top} : \mathbf {X}} \\
  \and
  \inferrule* [lab=Composition] {} {\Gamma \vdash P : \mathbf{U} : \mathbf {X} \;\;\; p : \mathbf{U} : \mathbf {X}, \Delta \vdash Q : \mathbf{V} : \mathbf {Y} \Vdash \Gamma, \Delta \vdash Q \substn{P}{p} : \mathbf{V} : \mathbf {Y}} \\
\end{mathpar}

%%

Thus, the meaning of the ground of the system will be

$$\begin{array}{rcl}
  \ldb \mathsf{Axiom}\; \epsilon & \Vdash & p : \mathbf{U} : \mathbf {X} \vdash p : \mathbf{U} : \mathbf {X} \rdb \\
  & = & \\
  \meaningof{\mathsf{Axiom}} : \meaningof{\epsilon} & \Rightarrow & \meaningof{p} : \meaningof{p : \mathbf{U : X}} \to \meaningof{\mathbf{U : X}} \\
 & = & \\
  \meaningof{\mathsf{Axiom}} : \mathsf{!} \circ \mathsf{1}_{\mathsf{NT^{op}}} \times \mathsf{1}_{\mathsf{NT}} & \Rightarrow & \mathsf{NT}^{1} \circ \mathsf{1}_{\mathsf{NT^{op}}} \times \mathsf{1}_{\mathsf{NT}} \\
	\bullet \in \, !(\meaningof{\mathbf{U} : \mathbf{X}}, \meaningof{\mathbf{U} : \mathbf{X}}) = \{\bullet \}& \mapsto & 1_{\meaningof{\mathbf{U} : \mathbf{X}}} \in \mathsf{NT}(\meaningof{\mathbf{U} : \mathbf{X}}, \meaningof{\mathbf{U} : \mathbf{X}})
\end{array}$$

Likewise, we calculate

$$\begin{array}{l}
  \meaningof{\Gamma \vdash Q : \mathbf{U} : \mathbf {X} \Vdash \Gamma \vdash \mathsf{Ch}(Q) : \mathbf{\{ Q \}} : \mathbf {X}} \\
  = \\
  \meaningof{\mathsf{Character}} : \meaningof{Q} : \meaningof{\Gamma} \to \meaningof{\mathbf{U} : \mathbf{X}} \Rightarrow \meaningof{\mathsf{Ch}(Q)} : \meaningof{\Gamma} \to \meaningof{\mathbf{\{ Q \}} : \mathbf {X}} \\
	\meaningof{Q}\in \mathsf{NT}(\meaningof{\Gamma}, \meaningof{\mathbf{U} : \mathbf{X}}) \mapsto \meaningof{\mathsf{Ch}(Q)} \in \mathsf{NT}(\meaningof{\Gamma}, {\rm Im}(\meaningof{Q}))
\end{array}$$
where ${\rm Im}(\meaningof{Q})$ denotes the subobject of the representable on $\mathbf{X}$ that assigns to each sort $S$ the set ${\rm Im}(\meaningof{Q}(S)).$  Because each set ${\rm Im}(\meaningof{Q}(S))$ is a subset of $\meaningof{U}(S)$, ${\rm Im}(\meaningof{Q})$ is a subobject of $\meaningof{\mathbf{U} : \mathbf{X}}$, which is in turn a subobject of the representable on $\mathbf{X}$.
%% left off here in discussion of is it the same Q.

$$\begin{array}{l}
  \meaningof{\Gamma \vdash P : \mathbf{U} : \mathbf {X} \;\;\; p : \mathbf{U} : \mathbf {X}, \Delta \vdash Q : \mathbf{V} : \mathbf {Y} \Vdash \Gamma, \Delta \vdash Q \substn{P}{p} : \mathbf{V} : \mathbf {Y}} \\
  = \\
  \meaningof{\mathsf{Composition}} : \langle \meaningof{P}, \meaningof{Q} \rangle \Rightarrow \meaningof{Q\substn{P}{p}} : \meaningof{\Gamma} \times \meaningof{\Delta} \to \meaningof{\mathbf{V} : \mathbf{Y}} \\
  = \\
  \quad \mathsf{NT}(\Gamma,\mathbf{U} : \mathbf {X}) \times \mathsf{NT}(\mathbf{U} : \mathbf {X} \times \Delta, \mathbf{V} : \mathbf {Y}) \Rightarrow \mathsf{NT}(\Gamma \times \Delta, \mathbf{V} : \mathbf {Y})
\end{array}$$

where $\meaningof{P} : \meaningof{\Gamma} \to \meaningof{\mathbf{U} : \mathbf{X}}$ and $ \meaningof{Q} : \meaningof{\mathbf{U} : \mathbf{X}} \times \meaningof{\Delta} \to \meaningof{\mathbf{V} : \mathbf{Y}}$; and the source and target of the denotation are $\mathsf{NT}^{\mathsf{op}} \times \mathsf{NT} \times \mathsf{NT}^{\mathsf{op}} \times \mathsf{NT}^{\mathsf{op}} \times \mathsf{NT} \to \mathsf{Set}$.



\subsubsection{Disjunctions}
\begin{mathpar}
  \inferrule* [lab=Union] {}{\Gamma \vdash P : \mathbf{U} : \mathbf {X} \Vdash \Gamma \vdash P : \mathbf{U \vee V} : \mathbf {X}} \\
  \and
  \inferrule* [lab=InL,Right=(V:Y)] {} {\Gamma \vdash P : \mathbf{U} : \mathbf {X} \Vdash \Gamma \vdash \mathsf{in}_{L} P : \mathbf{U+V} : \mathbf {X + Y}}
  \and
  \inferrule* [lab=InR,Right=(U:X)] {} {\Gamma \vdash Q : \mathbf{V} : \mathbf {Y} \Vdash \Gamma \vdash \mathsf{in}_{R} Q : \mathbf{U+V} : \mathbf {X + Y}} \\
  \and
  \inferrule* [lab=Match-Case] {p : \mathbf{U} : \mathbf {X}, \Gamma \vdash P : \mathbf{W} : \mathbf {Z} \;\;\; q : \mathbf{V} : \mathbf {Y}, \Delta \vdash Q : \mathbf{W} : \mathbf {Z}} {r : \mathbf{U+V} : \mathbf{X+Y}, \Gamma, \Delta \vdash \mathsf{match} \; r \; \mathsf{case} \; \mathsf{in}_{L}\; p \Rightarrow P \mathsf{;} \; \mathsf{case} \; \mathsf{in}_{R} \; q \Rightarrow Q : \mathbf{W} : \mathbf {Z}}
\end{mathpar}

\subsubsection{Conjunctions}
\begin{mathpar}
  \inferrule* [lab=Intersection] {}{\Gamma \vdash P : \mathbf{U} : \mathbf {X} \;\;\; \Gamma \vdash P : \mathbf{V} : \mathbf {X} \Vdash \Gamma \vdash P : \mathbf{U \wedge V} : \mathbf {X}} \\
  \and
  \inferrule* [lab=Product] {}{\Gamma \vdash P : \mathbf{U} : \mathbf {X} \;\;\; \Delta \vdash Q : \mathbf{V} : \mathbf {Y} \Vdash \Gamma, \Delta \vdash \langle P, Q \rangle : \mathbf{U \times V} : \mathbf{X \times Y}} \\
  \and
  \inferrule* [lab=Proj1] {} {p : \mathbf{U} : \mathbf{X}, \Gamma \vdash P : \mathbf{W} : \mathbf {Z} \Vdash r : \mathbf{U} \times \mathbf{V} : \mathbf{X} \times \mathbf{Y}, \Gamma \vdash \mathsf{let} \; \langle p, \_ \rangle \; = \; r \; \mathsf{in}\; P : \mathbf{W} : \mathbf {Z}} \\
  \and
  \inferrule* [lab=Proj2] {} {q : \mathbf{V} : \mathbf{Y}, \Gamma \vdash Q : \mathbf{W} : \mathbf {Z} \Vdash r : \mathbf{U \times V} : \mathbf{X} \times \mathbf{Y}, \Gamma \vdash \mathsf{let} \; \langle \_, q \rangle \; = \; r \; \mathsf{in} \; Q : \mathbf{W} : \mathbf {Z}}
\end{mathpar}

\subsubsection{Implications}
\begin{mathpar}
  \inferrule* [lab=Implication1,Right=(U:X)] {}{\Gamma \vdash Q : \mathbf{V} : \mathbf {X} \Vdash \Gamma \vdash Q : \mathbf{U} \Rightarrow \mathbf{V} : \mathbf {X}} \\
  \and
  \inferrule* [lab=Implication2,Right=(U:X)] {}{\Gamma \vdash P : \mathbf{V} \Rightarrow \bot : \mathbf {X} \Vdash \Gamma \vdash P : \mathbf{U} \Rightarrow \mathbf{V} : \mathbf {X}} \\
  \and
  \inferrule* [lab=Separation] {}{\Gamma \vdash P_{1} : \mathbf{U}_{1} : \mathbf {X} \;\;\; \Gamma \vdash P_{2} : \mathbf{U}_{2} : \mathbf {X} \Vdash \Gamma \vdash P_{2} : \mathbf{U}_{1} \Rightarrow \bot_{\mathbf{X}} : \mathbf {X}} \; \mbox{\small \((\mathbf{U}_{1} \wedge \mathbf{U}_{2} = \bot_{\mathbf{X}})\)}\\
  \and
  \inferrule* [lab=Abstraction] {}{p : \mathbf{U} : \mathbf {X}, \Gamma \vdash Q : \mathbf{V} : \mathbf {Y} \Vdash \Gamma \vdash \lambda p . Q : \mathbf{U}\to \mathbf{V} : \mathbf{X} \to \mathbf{Y}} \\
  \and
\inferrule* [lab=Application] {}{q : \mathbf{W} : \mathbf {Z}, \Gamma \vdash Q : \mathbf{V} : \mathbf {Y} \;\;\; \Delta \vdash P : \mathbf{U} : \mathbf{X} \Vdash \Gamma, r : \mathbf{U} \to \mathbf{W} : \mathbf{X} \to \mathbf{Z}, \Delta \vdash Q \substn{r(P)}{q} : \mathbf{V} : \mathbf{Y}}
\end{mathpar}

\subsubsection{Internalized operations}
\begin{mathpar}
  \inferrule* [lab=Pair] {}{\Gamma \vdash P : \mathbf{U} : \mathbf {P} \;\;\; \Delta \vdash Q : \mathbf{V} : \mathbf {P} \Vdash \Gamma,\Delta \vdash \mathsf{pair}( P,Q ) : \mathbf{U} \otimes \mathbf{V} : \mathbf {P}} \\
  \and
  \inferrule* [lab=Fst] {}{\Vdash \; \vdash \pi_{1} : \{ \mathbf{\pi}_{1} \} : \mathbf {P}}
  \and
  \inferrule* [lab=Snd] {}{\Vdash \; \vdash \pi_{2} : \{ \mathbf{\pi}_{2} \} : \mathbf {P}} \\
  \and
  \inferrule* [lab=TagL,Right=(V:P)] {}{\Gamma \vdash P : \mathbf{U} : \mathbf {P} \Vdash \Gamma \vdash \mathsf{tag}_{L} \; P : \mathbf{U} \oplus \mathbf{V} : \mathbf {P}} \\
  \and
  \inferrule* [lab=TagR,Right=(U:P)] {}{\Gamma \vdash Q : \mathbf{V} : \mathbf {P} \Vdash \Gamma \vdash \mathsf{tag}_{R} \; Q : \mathbf{U} \oplus \mathbf{V} : \mathbf {P}} \\
  \inferrule* [lab=Sum] {}{\Gamma \vdash P : \mathbf{U} : \mathbf {P} \;\;\; \Delta \vdash Q : \mathbf{V} : \mathbf {P} \Vdash \Gamma,\Delta \vdash [ P,Q] : [ \mathbf{U}, \mathbf{V} ] : \mathbf {P}}\\
  \inferrule* [lab=Abstraction] {}{p : \mathbf{U} : \mathbf {P}, \Gamma \vdash Q : \mathbf{V} : \mathbf {P} \Vdash \Gamma \vdash p \Rightarrow Q : \mathbf{U}\twoheadrightarrow \mathbf{V} : \mathbf{P}} \\
  \and
  \mathsf{int}^{+} : P + P \to P \and \mathsf{int}^{-} : ((P + P) \to P) \to P \\
  \and
  \mathsf{tag}_{i} P := \mathsf{int}^{+}(\mathsf{in}_{i} P) \\
  \and
  [P,Q] := \mathsf{int}^{-}(\lambda r.\mathsf{match}\; r \;\mathsf{case}\; \mathsf{in}_{L}\; p \Rightarrow P \odot p\mathsf{;}\; \mathsf{case}\; \mathsf{in}_{R}\; q \Rightarrow Q \odot q)
\end{mathpar}

$$\begin{array}{l}
  \meaningof{\Gamma \vdash P : \mathbf{U} : \mathbf {P} \;\;\; \Delta \vdash Q : \mathbf{V} : \mathbf {P} \Vdash \Gamma,\Delta \vdash \mathsf{pair}( P,Q ) : \mathbf{U} \otimes \mathbf{V} : \mathbf {P}} \\
  = \\
  \meaningof{\mathsf{Pair}} : \mathsf{NT}(\meaningof{\Gamma},\meaningof{\mathbf{U}:\mathbf{P}}) \times \mathsf{NT}(\meaningof{\Delta},\meaningof{\mathbf{V}:\mathbf{P}}) \Rightarrow \mathsf{NT}(\meaningof{\Gamma} \times \meaningof{\Delta},\mathsf{pair} \circ \meaningof{\mathbf{U}:\mathbf{P}} \times \meaningof{\mathbf{V}:\mathbf{P}}) \\
  = \\
  ??? \\
\end{array}$$

\subsubsection{Internalized redex constuctors}
\begin{mathpar}
  \inferrule* [lab=MatchL-Redex] {\Gamma_{1} \vdash P_{1} : \mathbf{U}_{1} : \mathbf {P} \;\;\; \Gamma_{2} \vdash P_{2} : \mathbf{U}_{2} : \mathbf {P} \;\;\; \Gamma_{1} \vdash P_{3} : \mathbf{U}_{3} : \mathbf {P}}{\Gamma_{1} ,\Gamma_{2}, \Gamma_{3} \vdash \mathsf{match}_{L}(P_{1},P_{2},P_{3}) : \mathbf{match}_{L}(P_{1},P_{2},P_{3}) : \mathbf {R}} \\
  \and
  \inferrule* [lab=MatchR-Redex] {\Gamma_{1} \vdash P_{1} : \mathbf{U}_{1} : \mathbf {P} \;\;\; \Gamma_{2} \vdash P_{2} : \mathbf{U}_{2} : \mathbf {P} \;\;\; \Gamma_{1} \vdash P_{3} : \mathbf{U}_{3} : \mathbf {P}}{\Gamma_{1} ,\Gamma_{2}, \Gamma_{3} \vdash \mathsf{match}_{R}(P_{1},P_{2},P_{3}) : \mathbf{match}_{R}(P_{1},P_{2},P_{3}) : \mathbf {R}} \\
  \inferrule* [lab=Proj1-Redex] {}{\Gamma_{1} \vdash P_{1} : \mathbf{U}_{1} : \mathbf {P} \;\;\; \Gamma_{2} \vdash P_{2} : \mathbf{U}_{2} : \mathbf {P} \Vdash \Gamma_{1} ,\Gamma_{2} \vdash \mathsf{proj}_{1}(P_{1},P_{2}) : \mathbf{proj}_{1}(P_{1},P_{2}) : \mathbf {R}} \\
  \and
  \inferrule* [lab=Proj2-Redex] {}{\Gamma_{1} \vdash P_{1} : \mathbf{U}_{1} : \mathbf {P} \;\;\; \Gamma_{2} \vdash P_{2} : \mathbf{U}_{2} : \mathbf {P} \Vdash \Gamma_{1} ,\Gamma_{2} \vdash \mathsf{proj}_{2}(P_{1},P_{2}) : \mathbf{proj}_{2}(P_{1},P_{2}) : \mathbf {R}} \\
  \inferrule* [lab=ApplyL-Redex] {}{\Gamma_{1} \vdash P_{1} : \mathbf{U} \twoheadrightarrow \mathbf{V} : \mathbf {P} \;\;\; \Gamma_{2} \vdash P_{2} : \mathbf{U} : \mathbf {P} \Vdash \Gamma_{1} ,\Gamma_{2} \vdash \mathsf{apply}_{l}(P_{1},P_{2}) : \mathbf{apply}_{l}(P_{1},P_{2}) : \mathbf {R}} \\
  \and
  \inferrule* [lab=ApplyR-Redex] {}{\Gamma_{1} \vdash P_{1} : \mathbf{U} : \mathbf {P} \;\;\; \Gamma_{2} \vdash P_{2} : \mathbf{U} \twoheadrightarrow \mathbf{V} : \mathbf {P} \Vdash \Gamma_{1} ,\Gamma_{2} \vdash \mathsf{apply}_{r}(P_{1},P_{2}) : \mathbf{apply}_{r}(P_{1},P_{2}) : \mathbf {R}}
\end{mathpar}

\subsubsection{Equations from $\mathsf{NT(CCC)}$}
\begin{mathpar}
  \inferrule* [lab=MatchL-Eqn] {\Gamma \vdash R : \mathbf{U} : \mathbf {X} \;\;\; p : \mathbf{U} : \mathbf{X}, \Delta_{1} \vdash P : \mathbf{W} : \mathbf{Z} \;\;\; q : \mathbf{V} : \mathbf{Y}, \Delta_{2} \vdash Q : \mathbf{W} : \mathbf{Z}}{\Gamma,\Delta_{1}, \Delta_{2} \vdash (\mathsf{match} \; \mathsf{in}_{L}\; R \; \mathsf{case} \; \mathsf{in}_{L} \; p \Rightarrow P \mathsf{;} \; \mathsf{case} \; \mathsf{in}_{R} \; q \Rightarrow Q) = P\substn{R}{p} : \mathbf{W} : \mathbf {Z}} \\
  \and
  \inferrule* [lab=MatchR-Eqn] {\Gamma \vdash R : \mathbf{V} : \mathbf {Y} \;\;\; p : \mathbf{U} : \mathbf{X}, \Delta_{1} \vdash P : \mathbf{W} : \mathbf{Z} \;\;\; q : \mathbf{V} : \mathbf{Y}, \Delta_{2} \vdash Q : \mathbf{W} : \mathbf{Z}}{\Gamma,\Delta_{1}, \Delta_{2} \vdash (\mathsf{match} \; \mathsf{in}_{R}\; R \; \mathsf{case} \; \mathsf{in}_{L} \; p \Rightarrow P \mathsf{;} \; \mathsf{case} \; \mathsf{in}_{R} \; q \Rightarrow Q) = Q\substn{R}{q} : \mathbf{W} : \mathbf {Z}} \\
  \and
  \inferrule* [lab=Proj1-Eqn] {p : \mathbf{U} : \mathbf{X}, \Gamma \vdash P : \mathbf{W} : \mathbf {Z} \;\;\; \Delta_{1} \vdash Q_{1} : \mathbf{U} : \mathbf{X} \;\;\; \Delta_{2} \vdash Q_{2} : \mathbf{V} : \mathbf{Y}}{ \Gamma, \Delta_{1},\Delta_{2} \vdash (\mathsf{let} \; \langle p, \_ \rangle \; = \; \langle Q_{1}, Q_{2} \rangle \; \mathsf{in}\; P) = P\substn{Q_{1}}{p} : \mathbf{W} : \mathbf {Z}} \\
  \and
  \inferrule* [lab=Proj2-Eqn] {p : \mathbf{V} : \mathbf{Y}, \Gamma \vdash P : \mathbf{W} : \mathbf {Z} \;\;\; \Delta_{1} \vdash Q_{1} : \mathbf{U} : \mathbf{X} \;\; \Delta_{2} \vdash Q_{2} : \mathbf{V} : \mathbf{Y}}{ \Gamma, \Delta_{1},\Delta_{2} \vdash (\mathsf{let} \; \langle \_, p \rangle \; = \; \langle Q_{1}, Q_{2} \rangle \; \mathsf{in}\; P) = P\substn{Q_{2}}{p} : \mathbf{W} : \mathbf {Z}} \\
\end{mathpar}

\subsubsection{Liftings from the $\lambda$-theory}
\begin{mathpar}
  \inferrule* [lab=Lifted-Singleton] {}{\Gamma_{\lambda} \vdash_{\lambda} P : \mathbf{X} \Vdash \Gamma \vdash P : \mathbf{ \{ P \} } : \mathbf{X}} \\
  \and
  \inferrule* [lab=Swap] {}{\Gamma \vdash P : \mathbf{U} : \mathbf{X} \;\;\; \Gamma_{\lambda} \vdash_{\lambda} P = Q : \mathbf{X} \Vdash \Gamma \vdash Q : \mathbf{U} : \mathbf{X}} \\
  \and
  \inferrule* [lab=Lifted-Eqn] {}{\Gamma \vdash P : \mathbf{U} : \mathbf{X} \;\;\; \Gamma \vdash Q : \mathbf{U} : \mathbf{X} \;\;\; \Gamma_{\lambda} \vdash_{\lambda} P = Q : \mathbf{X} \Vdash \Gamma \vdash P = Q : \mathbf{U} : \mathbf{X}}
\end{mathpar}

\subsubsection{Reductions from $\mathsf{NT(CCC)}$}
\begin{mathpar}
  \inferrule* [lab=MatchL-Src,Right=(V:P)] {\Gamma \vdash \mathsf{match}_{L}(P_{1},P_{2},P_{3}) : \mathbf{match}_{L}(\mathbf{U}_{1},\mathbf{U}_{2},\mathbf{U}_{3}) : \mathbf {R}}{\Gamma \vdash \mathsf{src}(\mathsf{match}_{L}(P_{1},P_{2},P_{3})) = [P_{2},P_{3}] \;  \; \odot \; \mathsf{tag}_{L} \; P_{1} : [\mathbf{U}_{2},\mathbf{U}_{3}] \; \odot \; \mathbf{U}_{1} \oplus \mathbf{V} : \mathbf {P}} \\
  \and
  \inferrule* [lab=MatchL-Trgt] {\Gamma \vdash \mathsf{match}_{L}(P_{1},P_{2},P_{3}) : \mathbf{match}_{L}(\mathbf{U}_{1},\mathbf{U}_{2},\mathbf{U}_{3}) : \mathbf {R}}{\Gamma \vdash \mathsf{trgt}(\mathsf{match}_{L}(P_{1},P_{2},P_{3})) = P_{2} \; \odot \; P_{1} : \mathbf{U}_{2} \; \odot \; \mathbf{U}_{1} : \mathbf {P}} \\
  \and
  \inferrule* [lab=MatchR-Src,Right=(V:P)] {\Gamma \vdash \mathsf{match}_{R}(P_{1},P_{2},P_{3}) : \mathbf{match}_{R}(\mathbf{U}_{1},\mathbf{U}_{2},\mathbf{U}_{3}) : \mathbf {R}}{\Gamma \vdash \mathsf{src}(\mathsf{match}_{R}(P_{1},P_{2},P_{3})) = [P_{2},P_{3}]  \; \odot \; \mathsf{tag}_{R}\; P_{1} : [\mathbf{U}_{2},\mathbf{U}_{3}] \; \odot \; \mathbf{U}_{1} \oplus \mathbf{V} : \mathbf {P}} \\
  \and
  \inferrule* [lab=MatchR-Trgt] {\Gamma \vdash \mathsf{match}_{R}(P_{1},P_{2},P_{3}) : \mathbf{match}_{R}(\mathbf{U}_{1},\mathbf{U}_{2},\mathbf{U}_{3}) : \mathbf {R}}{\Gamma \vdash \mathsf{trgt}(\mathsf{match}_{R}(P_{1},P_{2},P_{3})) = P_{3} \; \odot \; P_{1} : \mathbf{U}_{3} \; \odot \; \mathbf{U}_{1} : \mathbf {P}} \\
  \and
  \inferrule* [lab=Proj1-Src] {\Gamma \vdash \mathsf{Proj}_{1}(P_{1},P_{2}) : \mathbf{proj}_{1}(\mathbf{U}_{1},\mathbf{U}_{2}) : \mathbf {R}}{\Gamma \vdash \mathsf{src}(\mathsf{proj}_{1}(P_{1},P_{2})) = \pi_{1} \;  \; \odot \; \mathsf{pair}( P_{1}, P_{2} ) : \{ \pi_{1} \} \; \odot \; \mathbf{U}_{1} \otimes \mathbf{U}_{2} : \mathbf {P}} \\
  \and
  \inferrule* [lab=Proj1-Trgt] {\Gamma \vdash \mathsf{proj}_{1}(P_{1},P_{2}) : \mathbf{proj}_{1}(\mathbf{U}_{1},\mathbf{U}_{2}) : \mathbf {R}}{\Gamma \vdash \mathsf{trgt}(\mathsf{proj}_{1}(P_{1},P_{2})) = P_{1} : \mathbf{U}_{1} : \mathbf {P}} \\
  \and
  \inferrule* [lab=Proj2-Src] {\Gamma \vdash \mathsf{Proj}_{2}(P_{1},P_{2}) : \mathbf{proj}_{2}(\mathbf{U}_{1},\mathbf{U}_{2}) : \mathbf {R}}{\Gamma \vdash \mathsf{src}(\mathsf{proj}_{2}(P_{1},P_{2})) = \pi_{2} \;  \; \odot \; \mathsf{pair}( P_{1}, P_{2} ) : \{ \pi_{2} \} \; \odot \; \mathbf{U}_{1} \otimes \mathbf{U}_{2} : \mathbf {P}} \\
  \and
  \inferrule* [lab=Proj2-Trgt] {\Gamma \vdash \mathsf{proj}_{2}(P_{1},P_{2}) : \mathbf{proj}_{2}(\mathbf{U}_{1},\mathbf{U}_{2}) : \mathbf {R}}{\Gamma \vdash \mathsf{trgt}(\mathsf{proj}_{2}(P_{1},P_{2})) = P_{2} : \mathbf{U}_{2} : \mathbf {P}} \\
  \and
  \inferrule* [lab=ApplyL-Src] {\Gamma \vdash \mathsf{apply}_{L}(p \Rightarrow P_{1},P_{2}) : \mathbf{apply}_{L}(\mathbf{U} \twoheadrightarrow \mathbf{V},\mathbf{U}) : \mathbf {R}}{\Gamma \vdash \mathsf{src}(\mathsf{apply}_{L}(P_{1},P_{2})) = p \Rightarrow P_{1} \; \odot \; P_{2} : \mathbf{U} \twoheadrightarrow \mathbf{V} \; \odot \; \mathbf{U} : \mathbf {P}} \\
  \and
  \inferrule* [lab=ApplyL-Trgt] {\Gamma \vdash \mathsf{apply}_{L}(p \Rightarrow P_{1},P_{2}) : \mathbf{apply}_{L}(\mathbf{U} \twoheadrightarrow \mathbf{V},\mathbf{U}) : \mathbf {R}}{\Gamma \vdash \mathsf{trgt}(\mathsf{apply}_{L}(P_{1},P_{2})) = P_{1}\{P_{2} / p\} : \mathbf{V} : \mathbf {P}} \\
  \and
  \inferrule* [lab=ApplyR-Src] {\Gamma \vdash \mathsf{apply}_{R}(P_{1}, p \Rightarrow P_{2}) : \mathbf{match}_{R}(\mathbf{U},\mathbf{U} \twoheadrightarrow \mathbf{V}) : \mathbf {R}}{\Gamma \vdash \mathsf{src}(\mathsf{apply}_{R}(P_{1},P_{2})) = P_{1} \; \odot \; p \Rightarrow P_{2} : \mathbf{U} \; \odot \; \mathbf{U} \twoheadrightarrow \mathbf{V} : \mathbf {P}} \\
  \and
  \inferrule* [lab=ApplyR-Trgt] {\Gamma \vdash \mathsf{apply}_{R}(P_{1}, p \Rightarrow P_{2}) : \mathbf{apply}_{R}(\mathbf{U},\mathbf{U} \twoheadrightarrow \mathbf{V}) : \mathbf {R}}{\Gamma \vdash \mathsf{trgt}(\mathsf{apply}_{R}(P_{1},P_{2})) = P_{2}\{P_{1} / p\} : \mathbf{V} : \mathbf {P}} \\
\end{mathpar}

\subsubsection{Modal operators}
\begin{mathpar}
  \inferrule* [lab=Step-Future] {}{\Gamma \vdash R : \mathbf{E} : \mathbf {R} \Vdash \Gamma \vdash \mathsf{src}(R) : \lozenge \mathsf{trgt}(\mathbf{E}) : \mathbf {P}}
  \and
  \inferrule* [lab=Step-History] {}{\Gamma \vdash R : \mathbf{E} : \mathbf {R} \Vdash \Gamma \vdash \mathsf{trgt}(R) : \blacklozenge \mathsf{src}(\mathbf{E}) : \mathbf {P}}
\end{mathpar}
