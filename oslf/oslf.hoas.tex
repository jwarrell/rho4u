\section{Specifying a model of computation: lambda theories and higher order abstract syntax}

We follow Fiore and Hur \cite{DBLP:conf/csl/FioreH10} closely.

\begin{definition}
  A second order signature $\Sigma$ enjoys the following data $\Sigma = (T,O,|-|)$ with
  \begin{itemize}
    \item $T$ a set of types
    \item $O$ a set of operators
      \item $|-| : (T^{*} \times T^{*} \to T)$ and arity function.
  \end{itemize}  
\end{definition}

We write $o : (\arrvec{\sigma_{1}})\tau_{1} \times \ldots \times (\arrvec{\sigma_{n}})\tau_{n} \to \tau$ when $|o| = ((\arrvec{\sigma_{1}})\tau_{1}, \ldots, (\arrvec{\sigma_{n}})\tau_{n}, \tau)$

\subsection{Typing contexts}
We will consider terms in typing contexts. Typing contexts have two
zones, each respectively typing variables and metavariables. Variable
typings are types. Metavariable typings are parameterised types: a
metavariable of type $[\sigma_{1},\ldots,\sigma_{n}]\tau$, when
parameterised by terms of type $\sigma_{1},\ldots,\sigma_{n}$ yield a
term of type $\tau$. Thus, we use the following representation for
typing contexts $m_{1} : [\arrvec{\sigma_{1}}]\tau_{1}, \ldots, m_{j}: [\arrvec{\sigma_{j}}]\tau_{j} \rhd x_{1} : \sigma'_{1}, \ldots, x_{k}: \sigma'_{k}$.

\subsection{Terms}
 Signatures give rise to terms. These are built up by means of operators
from both variables and metavariables, and hence referred to as second-order.
Terms are considered up to $\alpha$-equivalence by stipulating that for every operator $o$ in the term $o(\ldots,(\arrvec{x}_{i})t_{i},\ldots)$ the $\arrvec{x}_{i}$ are bound in $t_{i}$.

\subsection{Judgments and inference rules}

\subsubsection{Judgments}
Judgments have a two part context, one for metavariables ($\Theta$),
and the other for variables ($\Gamma$), and a single type inference
for the consequence. Symbolically, we write $\Theta \rhd\Gamma \vdash t : \tau$, pronouncing it ``$\Theta$ and $\Gamma$ think that $t$ is of type $\tau$.''

\subsubsection{Inference rules}
Because we have inference rules both for lambda theories and also the
meta theory, there are rather a bit more of them, hence we allow a bit
of flexibility in the format. When space allows, we adopt a horizontal
rather than vertical syntax. Thus for an inference rule with judgments
$J_{1} \; \ldots \; J_{n}$ as the hypotheses, and $J$ as the
consequent, we write $J_{1} \; \ldots \; J_{n} \Vdash J$. When there
are no hypotheses, we write $\Vdash J$. When horizontal space is at a
premium, then we opt for the more traditional format

\begin{mathpar}
  \inferrule* [lab=Rule] {J_{1} \; \ldots \; J_{n}} {J}
\end{mathpar}

\subsubsection{Typed term formation}

The reader familiar with the ordinary simply typed $\lambda$-calculus
will see the family resemblance. The key difference is that we have
metavariables which enriches contexts, abstraction, and application.

\begin{mathpar}
  \inferrule* [lab=Variable]{ }{\Theta \rhd x : \tau,\Gamma \vdash x : \tau} \\
  \and
  \inferrule* [lab=Meta-variable]{m : [\tau_{1},\ldots,\tau_{n}]\tau, \Theta \rhd \Gamma \vdash t_{i} : \tau_{i}}{ m : [\tau_{1},\ldots,\tau_{n}]\tau, \Theta \rhd \Gamma \vdash m[t_{1},\ldots,t_{n}] : \tau} \\
  \and
  \inferrule* [lab=Abstraction]{\Theta \rhd \arrvec{x}_{1} : \arrvec{t}_{1},\ldots,\arrvec{x}_{n} : \arrvec{t}_{n}, \Gamma \vdash t_{i} : \tau_{i} \;\;\; o : (\arrvec{\sigma_{1}})\tau_{1} \times \ldots \times (\arrvec{\sigma_{n}})\tau_{n} \to \tau}{\Theta \rhd \Gamma \vdash o((\arrvec{x}_{1})t_{i},\ldots,(\arrvec{x}_{n})t_{n}) : \tau} \\
  \and
  \inferrule* [lab=Application]{\Theta \rhd \arrvec{x}_{1} : \arrvec{t}_{1},\ldots,\arrvec{x}_{n} : \arrvec{t}_{n}, \Gamma \vdash t : \tau \;\;\; \Theta \rhd \Gamma \vdash t_{i} : \tau_{i}}{\Theta \rhd \Gamma \vdash t\substn{t_{i}}{x_{i}} : \tau} \\
  \and
  \inferrule* [lab=Meta-application]{m_{1} : [\arrvec{\sigma}_{1}]\tau_{1},\ldots, m_{n} : [\arrvec{\sigma}_{n}]\tau_{n} \rhd \Gamma \vdash t : \tau \;\;\; \Theta \rhd \arrvec{x}_{1} : \arrvec{\sigma}_{1},\ldots,\arrvec{x}_{n} : \arrvec{\sigma}_{n}, \Gamma \vdash t_{i} : \tau_{i}}{\Theta \rhd \Gamma \vdash t\msubstn{(\arrvec{x}_{i})t_{i}}{m_{i}} : \tau}
\end{mathpar}

\subsubsection{Substitution}
\begin{mathpar}
  \inferrule* {}{x_{j}\substn{t_{i}}{x_{i}} = t_{j}} \\
  \and
  \inferrule* {}{m[\ldots,s,\ldots]\substn{t_{i}}{x_{i}} = m[\ldots,s\substn{t_{i}}{x_{i}},\ldots]} \\
  \and
  \inferrule* {}{(o(\ldots,(y_{1},\ldots,y_{k})s,\ldots))\substn{t_{i}}{x_{i}} = o(\ldots,(z_{1},\ldots,z_{k})((s\substn{z_{j}}{y_{j}})\substn{t_{i}}{x_{i}}),\ldots)}
\end{mathpar}

\subsubsection{Meta-substitution}
\begin{mathpar}
  \inferrule* {}{x\msubstn{(\arrvec{x}_{i})t_{i}}{m_{i}} = x} \\
  \and
  \inferrule* {}{m_{k}[s_{1},\ldots,s_{m}]\msubstn{(\arrvec{x}_{i})t_{i}}{m_{i}} = t_{k}[(s_{j}\msubstn{(\arrvec{x}_{i})t_{i}}{m_{i}})\substn{t_{i}}{x_{i,j}}]} \\
  \and
  \inferrule* {}{(o(\ldots,(\arrvec{x})s,\ldots))\msubstn{(\arrvec{x}_{i})t_{i}}{m_{i}} = o(\ldots,(\arrvec{x})s\msubstn{(\arrvec{x}_{i})t_{i}}{m_{i}},\ldots)}
\end{mathpar}
