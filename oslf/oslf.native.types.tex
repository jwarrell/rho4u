\section{A brief review of native types}
Consider the following recipe. Given a category, $\mathsf{Th}$ (we use this name to be suggestive that our category is a theory) we apply Yoneda
to it, resulting in a category of presheaves. To this we apply a
construction $\mathsf{sub}$ to the representables, converting them to
complete Heyting algebras. To this we apply the Grothendieck
construction. In symbols

\begin{eqnarray}
  \mathsf{NT} & := & \int \mathsf{sub} \; \mathsf{Y} \nonumber
\end{eqnarray}

This is the core construction of native types.

It treats models of computation as presented by some rewrite system
with binders, and thus begins with a meta-theory, $\mathsf{CCC}$, the
2-category of Cartesian closed categories. It treats $\mathsf{NT}$ as
an endofunctor on $\mathsf{CCC}$, deriving a new meta theory
$\mathsf{NT(C)}$ for each category, $\mathcal{C}$, in $\mathsf{|CCC|}$, the
objects of $\mathsf{CCC}$. The language presented below is a syntax
for the derived meta-theory.

\subsection{The topos}

First, let's review how subobject classifiers work in a topos. 

Given a category $\mathsf{Th} = 1 \xrightarrow{f} M$, where 1 is terminal, $\mathsf{Th}$ has five morphisms: $\{1, M, f, !, f!\}$, where $!\colon M \to 1$ and $f!\colon M \xrightarrow{!} 1 \xrightarrow{f} M.$

\subsubsection{The subobject classifier: $\Omega$.}

To understand the subobject classifier $\Omega$ in the topos $\hat{\mathsf{Th}} = {\rm Set}^{\mathsf{Th}^{\rm op}}$, let's consider the behavior on objects and morphisms.

\paragraph{Behavior on objects.}

$\Omega(1)$ is the set of sieves on $1$, the set of subsets of morphisms into $1$ that are closed under precomposition with morphisms in $\mathsf{Th}$.
$\{1, !\}= \top$
$\{\} = \bot$
We can't have $1$ without $! = 1!$.  We can't have $!$ without $1 = !f$.

$\Omega(M)$ is the set of sieves on $M$, the set of subsets of morphisms into $M$ that are closed under precomposition with morphisms in $\mathsf{Th}$.
$\{M, f, f!\} = \top$
$\{f, f!\}$
$\{\} = \bot$
We can't have $M$ without $f=Mf$ and $f!=Mf!$.  We can't have $f$ without $f!$.  We can't have $f!$ without $f = f!f$.

\paragraph{Behavior on morphisms.}

$\Omega(f)(S) =$ union of morphisms $g\in\{1, !\}$ such that $fg \in S$
$\begin{array}{rrcl}\Omega(f)\colon & \Omega(M) & \to & \Omega(1) \\ &\{M, f, f!\} & \mapsto & \{1, !\} \\ & \{f, f!\} & \mapsto & \{1, !\} \\ & \{\} & \mapsto & \{\}\end{array}$

$\Omega(!)(S) =$ union of morphisms $g\in\{M, f, f!\}$ such that $!g \in S$
$\begin{array}{rrcl}\Omega(!)\colon & \Omega(1) & \to & \Omega(M) \\ & \{1, !\} & \mapsto & \{M, f, f!\} \\ & \{\} & \mapsto & \{\}\end{array}$

$\Omega(f!)(S) =$ union of morphisms $g\in\{M, f, f!\}$ such that $f!g \in S$
$\begin{array}{rrcl}\Omega(M)\colon & \Omega(M) & \to & \Omega(M) \\ & \{M, f, f!\} & \mapsto & \{M, f, f!\} \\ & \{f, f!\} & \mapsto & \{M, f, f!\} \\ & \{\} & \mapsto & \{\}\end{array}$

Also, obviously, $\Omega$ of an identity morphism is the identity function on the set $S$.

\subsubsection{Logical operations}

\paragraph{Negation.}
In this section, $!$ is redefined temporarily: $\neg\colon\Omega \to \Omega$ is the character of $\bot\colon 1\to \Omega$, where $\bot$ is the character of $!\colon 0 \to 1$.

\begin{remark} The object 0 in the topos
  $\begin{array}{rcl}0:\mathsf{Th}^{\mathsf{op}} &\to& {\rm Set}\\c & \mapsto & \emptyset\end{array}$
\end{remark}

\begin{remark}
The object 1 in the topos
  $\begin{array}{rcl}1:\mathsf{Th}^{\mathsf{op}} &\to& {\rm Set}\\c & \mapsto & 1\end{array}$
\end{remark}

\begin{remark}
The natural transformation $!\colon 0\to 1$
  The natural transformation $!\colon 0\to 1$ assigns to each object the trivial function $!\colon \emptyset \to 1$.
\end{remark}  

\begin{remark}
The character $\bot\colon 1 \to \Omega$ of $!$
  The character $\bot\colon 1 \to \Omega$ of $!$ is a natural transformation that assigns to every object $X$ the function that maps $\bullet \in 1$ to $\bot \in \Omega(X)$.  It has to send $1$ to $\bot$ because if it didn't, the pullback would be $1$, not $0$ as given.
\end{remark}

\begin{remark}
The character $\neg\colon \Omega \to \Omega$ of $\bot$
The character $\neg\colon \Omega \to \Omega$ of $\bot$ is a natural transformation assigning to each object the function that takes $\bot$ to $\top$ and everything else to $\bot$. If anything other than $\bot$ mapped to $\top$, the pullback would be bigger than $1$.
$\begin{array}{cc}1&\begin{array}{rcl}\neg_1\colon\Omega(1)&\to&\Omega(1)\\ \top &\mapsto &\bot\\ \bot &\mapsto & \top\end{array}\\\\ M & \begin{array}{rcl}\neg_M\colon\Omega(M)&\to&\Omega(M)\\\top & \mapsto & \bot \\ \{f,f!\} & \mapsto & \bot \\\bot & \mapsto & \top\end{array}\end{array}$
\end{remark}

\paragraph{Conjunction.}
TBD

\paragraph{Disjunction.}
TBD

\subsection{Once again, but with monoids}

Here we take $\mathsf{Th}$ to be the finite product category generated by
\begin{itemize}
  \item one generating sort $M$
  \item $n+2$ term constructors:
    \begin{itemize}
      \item $n$ generators $g_{i}: 1 \to M$
      \item identity $e: 1 \to M$
      \item multiplication $*: M \times M \to M$
    \end{itemize}
  \item equations for assoc, unit laws
\end{itemize}

%% Ω: Th^op -> Set
%%       c |-> sieves(c)
%% f:c<-c' |-> Ω(f): Ω(c')
%%     -> Ω(c)
%%                   subset of morphisms into c' closed under
%% precomposition |-> union_{g:d->c' in Ω(c')} preimage of g under C(c',
%% f)
%%   with anything in Th

%% Ω(M) = sieves on M
%% = set of subsets of morphisms in Th into M that are closed under precomposition
%% e.g.
%% all morphisms that factor into a head and a tail that is a particular
%% morphism, like a generator or the identity or multiplication or a term
%% context  ... -> 1 -g_i-> M
%% all morphisms that factor into a head and a tail that is in a set of
%% morphisms, like a subset of the generators or a set of term contexts

%% (can't) Ω(M)^k ⊆ Ω(M^k) (can have entangled tuples) (I think)

\subsubsection{Interpretation of the logic of monoids}

In section \ref{oslf:monoidal-logic} we introduced a logic of monoids
naively generated from sets of monoid expressions. Here we show how to
interpret the formulae of that logic via the native type theory
construction, in terms of a recursive specification of the function $\meaningof{\cdot} : \mathcal{L}(M[G]) \to \int \mathsf{sub} \; \mathsf{Y}$, which is factored into $\meaningof{\cdot}_{2} : \mathcal{L}(M[G]) \to (\mathsf{sub} \; \mathsf{Y})(M)$ the result of which is then paired with $M$ to yield an object in $\int \mathsf{sub} \; \mathsf{Y}$. In general the shape of the translation looks like: $\meaningof{\mathsf{\phi}}_2  =  \lambda X.\bigcup_{z \in \meaningof{\mathsf{\phi}}_{n}} X \xrightarrow{!} 1 \xrightarrow{z} M$.

\begin{eqnarray*}
  & collection \; operations & \\
  \meaningof{\mathsf{true}}_{2} & = & \lambda X.\bigcup_{z \in \mathsf{Th}(1,M)} X \xrightarrow{!} 1 \xrightarrow{z} M \\
  \meaningof{\neg \phi}_{2} & = & \meaningof{\mathsf{true}}_{2} \backslash \meaningof{\phi}_{2} \\
  \meaningof{\phi \& \psi}_{2} & = & \meaningof{\phi}_{2} \cap \meaningof{\psi}_{2} \\
  & spatial \; operations & \\
  \meaningof{\mathbf{e}}_{2} & = & \lambda X.\bigcup_{z \in \{\mathsf{e}\}} \{ X \xrightarrow{!} 1 \xrightarrow{e} M \} \\
  \meaningof{\mathbf{g}}_{2} & = & \lambda X.\bigcup_{z \in \{\mathsf{g}\}} \{ X \xrightarrow{!} 1 \xrightarrow{g} M \} \\
  \meaningof{\phi \mathbf{*} \psi}_{2} & = & \lambda X.\bigcup_{ z \in \{ m \in \mathsf{Th}(1,M) : \exists m_{1}, m_{2}.m = m_{1} \mathsf{*} m_{2}, m_{1} \in \meaningof{\phi}_{2}(1), m_{2} \in \meaningof{\psi}_{2}(1) \}} X \xrightarrow{!} 1 \xrightarrow{z} M
\end{eqnarray*}


\subsubsection{Comparison to topos theoretic constructions}
Here are two presheaves, one for $\mathsf{*}$, the other for $\cap$.  Both are
subobjects of the representable on $M$.  To get natural transformations
from $\mathsf{Th}(-, M)$ into $\Omega(-)$, we take the character of the presheaves.

\begin{mathpar}
  X \in \mathsf{Th}^{op} \\
  \mapsto (X, X) \in (\mathsf{Th}^{op})^{2} \;\text{via}\; \Delta:\mathsf{Th}^{op} \to (\mathsf{Th}^{op})^{2} \;\text{in}\; \mathsf{Cat} \\
  \mapsto (\phi'(X), \psi'(X)) \in P(\mathsf{Th}(X,M))^{2} \subseteq \mathsf{Set}^{2} \;\text{via}\; \phi' \;\text{and}\; \psi' \;\text{and the product in}\; \mathsf{Cat} \\
  \mapsto \phi'(X) \times \psi'(X) \subseteq \mathsf{Th}(X,M)^{2} \in \mathsf{Set} \;\text{via the product in}\; \mathsf{Set} \\
  \mapsto \phi'(X) \mathsf{*} \psi'(X) \subseteq \mathsf{Th}(X,M) \in \mathsf{Set} \;\text{via mapping}\; \mathsf{*} \circ - \circ \Delta \;\text{over the set of pairs}\; \\

  \phi'(X) \times \psi'(X) \subseteq \mathsf{Th}(X,M)^{2} \to \phi'(X) \mathsf{*} \psi'(X) \subseteq \mathsf{Th}(X,M) \\
  (f, g) \mapsto \lambda x.f(x) \mathsf{*} g(x)  =  \mathsf{*} \circ - \circ \Delta
\end{mathpar}

\begin{mathpar}
  X \in \mathsf{Th}^{op}
  \mapsto (X, X) \in (\mathsf{Th}^{op})^{2} \;\text{via}\; \Delta:\mathsf{Th}^{op} \to (\mathsf{Th}^{op})^{2} \;\text{in}\; \mathsf{Cat}
\end{mathpar}
Here are two presheaves, one for $\mathsf{*}$, the other for $\cap$.  Both are
subobjects of the representable on M.  To get natural transformations
from $\mathsf{Th}( -, M )$ into $\Omega(-)$, we take the character of the presheaves.

\begin{mathpar}
  X \in \mathsf{Th}^{op} \\
  \mapsto (X, X) \in (\mathsf{Th}^{op})^{2} \;\text{via}\; \Delta:\mathsf{Th}^{op} \to (\mathsf{Th}^{op})^{2} \;\text{in}\; \mathsf{Cat} \\
  \mapsto (\phi'(X), \psi'(X)) \in \mathcal{P}(\mathsf{Th}(X,M))^{2} \subseteq \mathsf{Set}^{2} \;\text{via}\; \phi' \;\text{and}\; \psi' \;\text{and the product in}\; \mathsf{Cat} \\
  \mapsto \phi'(X) \times \psi'(X) \subseteq \mathsf{Th}(X,M)^{2} \in \mathsf{Set} \;\text{via the product in}\; \mathsf{Set} \\
  \mapsto \phi'(X) \mathsf{*} \psi'(X) \subseteq \mathsf{Th}(X,M) \in \mathsf{Set} \;\text{via mapping}\; \mathsf{*} \circ - \circ \Delta \;\text{over the set of pairs}\; \\

  \phi'(X) \times \psi'(X) \subseteq \mathsf{Th}(X,M)^{2} \to \phi'(X) \mathsf{*} \psi'(X) \subseteq \mathsf{Th}(X,M) \\
  (f, g) \mapsto \lambda x.f(x) \mathsf{*} g(x)  =  \mathsf{*} \circ - \circ \Delta \\

  X \in \mathsf{Th}^{op} \\
  \mapsto (X, X) \in (\mathsf{Th}^{op})^{2} \;\text{via}\; \Delta:\mathsf{Th}^{op} \to (\mathsf{Th}^{op})^{2} \;\text{in}\; \mathsf{Cat}
\end{mathpar}
