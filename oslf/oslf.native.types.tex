\section{A brief review of native types}
Consider the following recipe. Given a category, $C$, we apply Yoneda
to it, resulting in a category of presheaves. To this we apply a
construction $\mathsf{sub}$ to the representables, converting them to
complete Heyting algebras. To this we apply the Grothendieck
construction. In symbols

\begin{eqnarray}
  \mathsf{NT} & := & \int \mathsf{sub} \; \mathsf{Y} \nonumber
\end{eqnarray}

This is the core construction of native types.

It treats models of computation as presented by some rewrite system
with binders, and thus begins with a meta-theory, $\mathsf{CCC}$, the
2-category of Cartesian closed categories. It treats $\mathsf{NT}$ as
an endofunctor on $\mathsf{CCC}$, deriving a new meta theory
$\mathsf{NT(C)}$ for each category, $C$, in $\mathsf{|CCC|}$, the
objects of $\mathsf{CCC}$. The language presented below is a syntax
for the derived meta-theory.
