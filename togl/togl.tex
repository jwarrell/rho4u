\documentclass[12pt]{llncs}

\usepackage[pdftex]{hyperref}                   
\usepackage {mathpartir}
\usepackage{bigpage}
\usepackage{bcprules}
\usepackage {listings}
\usepackage{amsmath}
\usepackage{amssymb}
\usepackage{amsfonts}
\usepackage{amstext}
\usepackage{latexsym}
\usepackage{longtable}
\usepackage{stmaryrd}
%\usepackage{fdsymbol}
\usepackage{graphicx} 
%\usepackage[margins=2.5cm,nohead,nofoot]{geometry}
%\usepackage{geometry}
\usepackage{color}


%\include{myPreamble}

% Double brackets
\newcommand{\ldb}{[\![}
\newcommand{\rdb}{]\!]}
\newcommand{\ldrb}{(\!(}
\newcommand{\rdrb}{)\!)}
\newcommand{\lrbb}{(\!|}
\newcommand{\rrbb}{|\!)}
\newcommand{\lliftb}{\langle\!|}
\newcommand{\rliftb}{|\!\rangle}
\newcommand{\lrcb}{\{\!\!\{}
\newcommand{\rrcb}{\}\!\!\}}
%\newcommand{\plogp}{:\!-}
\newcommand{\plogp}{\leftarrow}
%\newcommand{\plogp}{\coloneq}
% \newcommand{\lpquote}{\langle}
% \newcommand{\rpquote}{\rangle}
% \newcommand{\lpquote}{\lceil}
% \newcommand{\rpquote}{\rceil}
\newcommand{\lpquote}{\ulcorner}
\newcommand{\rpquote}{\urcorner}
\newcommand{\newkw}{\nu}

% SYNTAX
\newcommand{\id}[1]{\texttt{#1}}
\newcommand{\none}{\emptyset}
\newcommand{\eps}{\epsilon}
\newcommand{\set}[1]{\{#1\}}
\newcommand{\rep}[2]{\id{\{$#1$,$#2$\}}}
\newcommand{\elt}[2]{\id{$#1$[$#2$]}}
\newcommand{\infinity}{$\infty$}

\newcommand{\pzero}{\mathsf{0}}
\newcommand{\seq}{\mathsf{\id{,}}}
\newcommand{\all}{\mathsf{\id{\&}}}
\newcommand{\choice}{\mathsf{\id{|}}}
\newcommand{\altern}{\mathsf{\id{+}}}
\newcommand{\juxtap}{\mathsf{\id{|}}}
\newcommand{\concat}{\mathsf{.}}
\newcommand{\punify}{\mathsf{\id{:=:}}}
\newcommand{\fuse}{\mathsf{\id{=}}}
\newcommand{\scong}{\mathsf{\equiv}}
\newcommand{\nameeq}{\mathsf{\equiv_N}}
\newcommand{\alphaeq}{\mathsf{\equiv_{\alpha}}}
\newcommand{\arrvec}[1]{\overrightarrow{#1}}
\newcommand{\names}[1]{\mathsf{N}(#1)}
\newcommand{\freenames}[1]{\mathsf{FN}(#1)}
\newcommand{\boundnames}[1]{\mathsf{BN}(#1)}
%\newcommand{\lift}[2]{\texttt{lift} \; #1 \concat #2}
\newcommand{\binpar}[2]{#1 \mathsf{|} #2}
\newcommand{\binparx}[2]{\mathsf{par}(#1, #2)}
\newcommand{\binpart}[2]{\mathbf{par}(\mathbf{#1}, \mathbf{#2})}
\newcommand{\binpartl}[2]{\mathbf{par}_{L}(\mathbf{#1}, \mathbf{#2})}
\newcommand{\binpartr}[2]{\mathbf{par}_{R}(\mathbf{#1}, \mathbf{#2})}
\newcommand{\outputp}[2]{#1\mathsf{!}(#2)}
\newcommand{\outputt}[2]{\mathbf{#1}\mathsf{!}(\mathbf{#2})}
\newcommand{\prefix}[3]{\mathsf{for(}#2 \leftarrow #1 \mathsf{)} #3}
\newcommand{\prefixt}[3]{\mathbf{for}\mathbf{(}\mathbf{#2} \leftarrow \mathbf{#1} \mathbf{)} \mathbf{#3}}
%%\newcommand{\lift}[2]{#1 \lliftb #2 \rliftb}
\newcommand{\lift}[2]{#1 \mathsf{!}(#2)}
\newcommand{\clift}[1]{\lliftb #1 \rliftb}
\newcommand{\quotep}[1]{\mathsf{@} #1}
\newcommand{\quotet}[1]{\mathbf{@} \mathbf{#1}}
\newcommand{\dropn}[1]{\mathsf{*} #1}
\newcommand{\dropt}[1]{\mathbf{*} \mathbf{#1}}
\newcommand{\procn}[1]{\stackrel{\cdot}{#1}}
\newcommand{\commr}[4]{\mathsf{comm(}#1, #2, #3, #4  \mathsf{)}}
\newcommand{\commt}[3]{\mathbf{comm(}\mathbf{#1}, \mathbf{#2}, \mathbf{#3} \mathbf{)}}

\newcommand{\newp}[2]{(\newkw \; #1 ) #2}
\newcommand{\bangp}[1]{! #1}

\newcommand{\substp}[2]{\{ \quotep{#1} / \quotep{#2} \}}
\newcommand{\substn}[2]{\{ #1 / #2 \}}
\newcommand{\msubstn}[2]{\lrcb #1 / #2 \rrcb}

\newcommand{\psubstp}[2]{\widehat{\substp{#1}{#2}}}
\newcommand{\psubstn}[2]{\widehat{\substn{#1}{#2}}}

\newcommand{\applyp}[2]{#1 \langle #2 \rangle}
\newcommand{\absp}[2]{( #1 ) #2}
\newcommand{\annihilate}[1]{#1^{\times}}
\newcommand{\dualize}[1]{#1^{\bullet}}
\newcommand{\ketp}[1]{\mathsf{|}#1 \rangle}
\newcommand{\brap}[1]{\langle #1 \mathsf{|}}
\newcommand{\testp}[2]{\langle #1 \mathsf{|} #2 \rangle}
\newcommand{\testnp}[3]{\langle #1 \mathsf{|} #2 \mathsf{|} #3 \rangle}

\newcommand{\transitions}[3]{\mathbin{#1 \stackrel{#2}{\longrightarrow} #3}}
\newcommand{\meaningof}[1]{\ldb #1 \rdb}
\newcommand{\pmeaningof}[1]{\ldb #1 \rdb}
\newcommand{\nmeaningof}[1]{\lrbb #1 \rrbb}

\newcommand{\Proc}{\mathsf{Proc}}
\newcommand{\QProc}{\quotep{\mathsf{Proc}}}
\newcommand{\MixSumProc}{\mathsf{MixSumProc}}

\newcommand{\entailm}{\mathbin{\vdash_{\mathfrak m}}} %matching
\newcommand{\entailp}{\mathbin{\vdash_{\mathfrak p}}} %behavioral
\newcommand{\entailv}{\mathbin{\vdash_{\mathfrak v}}} %validation
\newcommand{\congd}{\mathbin{\equiv_{\mathfrak d}}}
\newcommand{\congs}{\mathbin{\equiv_{\mathfrak s}}}
\newcommand{\congp}{\mathbin{\equiv_{\mathfrak p}}}
%\newcommand{\logequiv}{\mathbin{\leftrightarrow}}

\newcommand{\barb}[2]{\mathbin{#1 \downarrow_{#2}}}
\newcommand{\dbarb}[2]{\mathbin{#1 \Downarrow_{#2}}}

% From pi-duce paper
\newcommand{\red}{\rightarrow}
\newcommand{\wred}{\Rightarrow}
\newcommand{\redhat}{\hat{\longrightarrow}}
\newcommand{\lred}[1]{\stackrel{#1}{\longrightarrow}} %transitions
\newcommand{\wlred}[1]{\stackrel{#1}{\Longrightarrow}}

\newcommand{\opm}[2]{\overline{#1} [ #2 ]} % monadic
\newcommand{\ipm}[2]{{#1} ( #2 )} 
\newcommand{\ipmv}[2]{{#1} ( #2 )} % monadic
\newcommand{\parop}{\;|\;}		% parallel operator
\newcommand{\patmatch}[3]{#2 \in #3 \Rightarrow #1}
\newcommand{\sdot}{\, . \,}		% Space around '.'
\newcommand{\bang}{!\,}
%\newcommand{\fuse}[1]{\langle #1 \rangle}		
\newcommand{\fusion}[2]{#1 = #2} % fusion prefix/action
\newcommand{\rec}[2]{\mbox{\textsf{rec}} \, #1. \, #2}
\newcommand{\match}[2]{\mbox{\textsf{match}} \; #1 \; \mbox{\textsf{with}} \; #2}
\newcommand{\sep}{:}
\newcommand{\val}[2]{\mbox{\textsf{val}} \; #1 \; \mbox{\textsf{as}} \; #2}

\newcommand{\rel}[1]{\;{\mathcal #1}\;} %relation
\newcommand{\bisim}{\stackrel{.}{\sim}_b} %bisimilar
\newcommand{\wb}{\approx_b} %weak bisimilar
\newcommand{\bbisim}{\stackrel{\centerdot}{\sim}} %barbed bisimilar
\newcommand{\wbbisim}{\stackrel{\centerdot}{\approx}} %weak barbed bisimilar
\newcommand{\bxless}{\lesssim}	%expansion less (amssymb required)
\newcommand{\bxgtr}{\gtrsim}	%expansion greater (amssymb required)
\newcommand{\beq}{\sim}		%barbed congruent
\newcommand{\fwbeq}{\stackrel{\circ}{\approx}}	%weak barbed congruent
\newcommand{\wbeq}{\approx}	%weak barbed congruent
\newcommand{\sheq}{\simeq}	%symbolic hypereq
\newcommand{\wbc}{\approx_{cb}}

% rho logic

\newcommand{\ptrue}{\mathbin{true}}
\newcommand{\psatisfies}[2]{#1 \models #2}
\newcommand{\pdropf}[1]{\rpquote #1 \lpquote}
\newcommand{\plift}[2]{#1 \lliftb #2 \rliftb}
\newcommand{\pprefix}[3]{\langle #1 ? #2 \rangle #3}
\newcommand{\pgfp}[2]{\textsf{rec} \; #1 \mathbin{.} #2}
\newcommand{\pquant}[3]{\forall #1 \mathbin{:} #2 \mathbin{.} #3}
\newcommand{\pquantuntyped}[2]{\forall #1 \mathbin{.} #2}
\newcommand{\riff}{\Leftrightarrow}

\newcommand{\PFormula}{\mathbin{PForm}}
\newcommand{\QFormula}{\mathbin{QForm}}
\newcommand{\PropVar}{\mathbin{\mathcal{V}}}

% qm notation
\newcommand{\state}[1]{\juxtap #1 \rangle}
\newcommand{\event}[1]{\langle #1 \juxtap}
\newcommand{\innerprod}[2]{\langle #1 \juxtap #2 \rangle}
\newcommand{\prmatrix}[2]{\juxtap #1 \rangle \langle #2 \juxtap}
\newcommand{\fprmatrix}[3]{\juxtap #1 \rangle #2 \langle #3 \juxtap}

% End piduce contribution

\newcommand{\typedby}{\mathbin{\:\colon}}
\newcommand{\mixedgroup}[1]{\id{mixed($#1$)}}
\newcommand{\cast}[2]{\id{CAST AS} \; #1 \; (#2)}
\newcommand{\bslsh}{\mathbin{\id{\\}}}
\newcommand{\bslshslsh}{\mathbin{\id{\\\\}}}
\newcommand{\fslsh}{\mathbin{\id{/}}}
\newcommand{\fslshslsh}{\mathbin{\id{//}}}
\newcommand{\bb}[1]{\mbox{#1}}
\newcommand{\bc}{\mathbin{\mathbf{::=}}}
\newcommand{\bm}{\mathbin{\mathbf\mid}}
%\newcommand{\bm}{\mathbin{\mathbf\mid}}
\newcommand{\be}{\mathbin{=}}
\newcommand{\bd}{\mathbin{\buildrel {\rm \scriptscriptstyle def} \over \be}}
%\newcommand{\category}[1]{\mbox{\bf #1}}

%GRAMMAR
\newlength{\ltext}
\newlength{\lmath}
\newlength{\cmath}
\newlength{\rmath}
\newlength{\rtext}

\settowidth{\ltext}{complex type name}
\settowidth{\lmath}{$xxx$}
\settowidth{\cmath}{$::=$}
\settowidth{\rmath}{\id{attributeGroup}}
\settowidth{\rtext}{repetition of $g$ between $m$ and $n$ times}

\newenvironment{grammar}{
  \[
  \begin{array}{l@{\quad}rcl@{\quad}l}
  \hspace{\ltext} & \hspace{\lmath} & \hspace{\cmath} & \hspace{\rmath} & \hspace{\rtext} \\
}{
  \end{array}\]
}

% Over-full v-boxes on even pages are due to the \v{c} in author's name
%\vfuzz2pt % Don't report over-full v-boxes if over-edge is small

% THEOREM Environments ---------------------------------------------------
% MATH -------------------------------------------------------------------
 \newcommand{\veps}{\varepsilon}
 \newcommand{\To}{\longrightarrow}
 \newcommand{\h}{\mathcal{H}}
 \newcommand{\s}{\mathcal{S}}
 \newcommand{\A}{\mathcal{A}}
 \newcommand{\J}{\mathcal{J}}
 \newcommand{\M}{\mathcal{M}}
 \newcommand{\W}{\mathcal{W}}
 \newcommand{\X}{\mathcal{X}}
 \newcommand{\BOP}{\mathbf{B}}
 \newcommand{\BH}{\mathbf{B}(\mathcal{H})}
 \newcommand{\KH}{\mathcal{K}(\mathcal{H})}
 \newcommand{\Real}{\mathbb{R}}
 \newcommand{\Complex}{\mathbb{C}}
 \newcommand{\Field}{\mathbb{F}}
 \newcommand{\RPlus}{\Real^{+}}
 \newcommand{\Polar}{\mathcal{P}_{\s}}
 \newcommand{\Poly}{\mathcal{P}(E)}
 \newcommand{\EssD}{\mathcal{D}}
 \newcommand{\Lom}{\mathcal{L}}
 \newcommand{\States}{\mathcal{T}}
 \newcommand{\abs}[1]{\left\vert#1\right\vert}
% \newcommand{\set}[1]{\left\{#1\right\}}
%\newcommand{\seq}[1]{\left<#1\right>}
 \newcommand{\norm}[1]{\left\Vert#1\right\Vert}
 \newcommand{\essnorm}[1]{\norm{#1}_{\ess}}

%%% NAMES
\newcommand{\Names}{{\mathcal N}}
\newcommand{\Channels}{{\sf X}}
\newcommand{\Variables}{{\mathcal V}}
\newcommand{\Enames}{{\mathcal E}}
\newcommand{\Nonterminals}{{\mathcal S}}
\newcommand{\Pnames}{{\mathcal P}}
\newcommand{\Dnames}{{\mathcal D}}
\newcommand{\Types}{{\mathcal T}}

\newcommand{\fcalc}{fusion calculus}
\newcommand{\xcalc}{${\mathfrak x}$-calculus}
\newcommand{\lcalc}{$\lambda$-calculus}
\newcommand{\pic}{$\pi$-calculus}
\newcommand{\spic}{spi-calculus}
\newcommand{\rhoc}{$\rho$-calculus}
\newcommand{\rhol}{$\rho$-logic}
\newcommand{\hcalc}{highwire calculus}
\newcommand{\dcalc}{data calculus}
%XML should be all caps, not small caps. --cb
%\newcommand{\xml}{\textsc{xml}}
\newcommand{\xml}{XML} 
 

%\ifpdf
%\usepackage[pdftex]{graphicx}
%\else
%\usepackage{graphicx}
%\fi

 % \ifpdf
%  \usepackage{pdfsync}
%  \if


%\title{Brief Article}
%\author{David F. Snyder}
%\author{L.G. Meredith}

%\address{Dept. of Math., Texas State University--San Marcos, San Marcos, TX 78666}
       
\pagestyle{empty}


\begin{document}

\lstset{language=[Objective]Caml,frame=shadowbox}


\def\lastname{Meredith}

\title{Notes on a formal theory of graphs}
\titlerunning{oslf}

\author{ Lucius Gregory Meredith\inst{1}
}
\institute{
        {Managing Partner, RTech Unchained 9336 California Ave SW, Seattle, WA 98103, USA} \\
  \email{ lgreg.meredith@gmail.com } 
}
 

\maketitle              % typeset the title of the contribution

%%% ----------------------------------------------------------------------

\begin{abstract}

  We describe a formal theory of graphs with good complexity properties stemming from its compositional structure.

\end{abstract}

%\keywords{Process calculi, knots, invariants}

% \begin{keyword}
% concurrency, message-passing, process calculus, reflection, program logic
% \end{keyword}

%\end{frontmatter}


We describe an algebraic theory of graphs, \(\mathsf{G}[X,V]\). The
theory is dependent on a theory of variables, \(X\), and a theory of
vertices, \(V\). Each theory is required to provide an effective
procedure for deciding membership. More specifically, given any \(u\) we
can decide \(u \in X\) (and \(u \in V\), respectively). Likewise, both
dependencies must come with an effective notion of equality. That is,
for variables \(x_1\) and \(x_2\) there is an effective procedure for
deciding whether \(x_1 = x_2\), and likewise for vertices. We refer to
the graphs captured by the theory as the domain of graphs of
\(\mathsf{G}[X,V]\). The theory is unique in that it explicitly includes
a notion of a reference to a vertex and only admits edges between
references \cite{DBLP:journals/corr/abs-0809-3023} \cite{DBLP:journals/iandc/DawarGG07} \cite{DBLP:conf/icalp/CardelliGG02} \cite{godsil2013algebraic}. Indeed, variables supply the references. We write

\[\mathsf{G}[X,V];\Gamma \vdash g\]

to mean that -- given the references \(\Gamma\) -- \(g\) is well formed
in \(\mathsf{G}[X,V]\) and we use type inference rules of the form

\[\frac{ \mathsf{G}[X,V]; \Gamma_1 \vdash g_1 \; \ldots \; \mathsf{G}[X,V]; \Gamma_n \vdash g_n}{ \mathsf{G}[X,V]; K_{\Gamma}(\Gamma_1,\ldots,\Gamma_n) \vdash K_g(g_1,\ldots, g_n)}K_{ctor}\]

to indicate that given dependencies
\(K_{\Gamma}(\Gamma_1,\ldots,\Gamma_n)\) the graph,
\(K_g(g_1,\ldots,g_n)\), constructed from well-formed graphs
\(g_1,\ldots,g_n\), using the constructor \(K_g\) is well-formed. The
effective procedure (aka algorithm) for determining when a graph is well
formed is given by a collection of such rules.

There are two principal reasons for the development of this theory. One
is complexity management. Whether they are the objects of investigations
themselves, as in graph theory, or are being used as tools to
support some other investigation such as in biology or physics or
cryptocurrency, graphs are used primarily in \emph{computations}. As
such they should be \emph{algorithmically} specified. The standard
presentation of graphs as collections of vertices and edges does not
treat them as algorithmically specified. To understand this lacunae
consider the standard presentation of a linked list.

\[\mathsf{List}[A] = 1 + A \times \mathsf{List}[A]\]

This recursive specification matches the standard interpretation of
lists containing elements of type \(A\) as either empty (the \(1\) in
the domain equation) or formed by consing an element of type \(A\) onto
a list containing elements of type \(A\) (the product
\(A \times \mathsf{List}[A]\)). The recursive specification recognizes
that lists are recursively built up from \emph{lists} and a small set of
operations.

Such an algebraic characterization of linked lists is typically used in
algorithmic specifications of calculations involving lists. By contrast
graphs are typically presented as a pair of collections of vertices and
edges, rather than as a recursive data type involving operations that
build graphs from graphs. This conception permeates many tools and
software libraries that provide computational support for calculating
with graphs and result in nearly overwhelming complexity.

An excellent example is the popular tool GraphViz and its dotty
language. Using dotty to specify the complete graph of 100 nodes is an
enormous task. The recursive data type resulting from the theory of
graphs presented here results in a single line specification of the
complete graph. More generally, in the examples we present we see an
exponential jump in complexity in the types of graphs being
characterized and yet the algorithmic characterizations remain typical
1-line recursive specifications.

The other motivation is foundational We seek a minimal characterization
of subtypes of of the type of graphs.

\hypertarget{well-formedness}{%
\subsection{Well-formedness}\label{well-formedness}}

This is an algebraic theory, and thus it is syntactic in nature. It
provides a language of graphs that are built out of logical sentences.
In more detail, the theory admits three kinds of sentences:

\begin{itemize}
\item
  recognizing an admissible vertex:
  \(\mathsf{G}[X,V]; \Gamma \vdash v\);
\item
  recognizing an admissible variable:
  \(\mathsf{G}[X,V]; \Gamma \vdash x\);
\item
  and, recognizing a well formed graph:
  \(\mathsf{G}[X,V]; \Gamma \vdash g\)
\end{itemize}

Variables are used to capture references to vertices which are in turn
used to form edges between vertices. As such, judging the wellformedness
of a graph depends on the use of references. Hence, a judgement makes
use of a dependency list, \(\Gamma\) which is just a sequence of
variables. That is,

\[\Gamma ::= () \;|\; x, \Gamma\]

Notationally, we overload the comma to indicate concatenation of
sequences. Thus, given \(\Gamma_1 = x_{11},\ldots,x_{1m}\) and
\(\Gamma_2 = x_{21},\ldots,x_{2n}\), then
\(\Gamma_1,\Gamma_2 = x_{11},\ldots,x_{1m},x_{21},\ldots,x_{2n}\)

A graph expression is given by the grammar

\[g,h ::= 0 \;|\; v|g \;|\; g \otimes h \; |\;\mathsf{let}\; x = v \; \mathsf{in}\; g \;|\; \langle \mathsf{let}\; x_1 = v_1 \; \mathsf{in}\; g, \mathsf{let}\; x_2 = v_2 \; \mathsf{in}\; h\rangle\]

\hypertarget{pronunciation}{%
\subsubsection{Pronunciation}\label{pronunciation}}

The graph constructors are pronounced as follows.

\begin{itemize}
\item
  \(0\) -- ``the empty graph'', or just ``empty'';
\item
  \(v | g\) -- ``\(g\) with the vertex \(v\) adjoined'', or ``adjoin
  \(v\) to \(g\)'';
\item
  \(g_1 \otimes g_2\) -- ``the graph formed by juxtaposing \(g_1\) and
  \(g_2\)'', or ``juxtapose \(g_1\) and \(g_2\)'', or just ``\(g_1\) and
  \(g_2\)'';
\item
  \(\mathsf{let}\; x = v \; \mathsf{in}\; g\) -- ``let \(x\) stand for
  \(v\) in \(g\)''; or ``let \(x\) be \(v\) in \(g\)'';
\item
  \(\langle \mathsf{let}\; x_1 = v_1 \; \mathsf{in}\; g_1,\mathsf{let}\; x_2 = v_2 \; \mathsf{in}\; g_2\rangle\)
  -- ``the graph formed by connecting \(g_1\) to \(g_2\) with and edge
  from \(x_1\) to \(x_2\)''; or, ``connect \(g_1\) to \(g_2\) with and
  edge from \(x_1\) to \(x_2\)''.
\end{itemize}

The judgment \(\mathsf{G}[X,V]; \Gamma \vdash g\) is pronounced
``\(\mathsf{G}[X,V]\) thinks that \(g\) is well-formed, given
dependencies, \(\Gamma\) .'' Similarly,
\(\mathsf{G}[X,V]; \Gamma \vdash v\) is pronounced ``\(\mathsf{G}[X,V]\)
thinks that \(v\) is an admissible vertex, given dependencies,
\(\Gamma\);'' and \(\mathsf{G}[X,V]; \Gamma \vdash x\) is pronounced
``\(\mathsf{G}[X,V]\) thinks that \(x\) is an admissible variable, given
dependencies, \(\Gamma\).''

A rule of the form

\[\frac{ H_1, \ldots , H_n }{ C }R\]

is pronounced ``\(R\) concludes that \(C\) given \(H_1\), \(\ldots\),
\(H_n\)''.

The rules for judging when a vertex or a variable are admissible or
graph is well formed are as follows

\[\frac{ }{ \mathsf{G}[X,V]; () \vdash 0}Foundation\]

\[\frac{ v \in V }{ \mathsf{G}[X,V]; () \vdash v}Verticity \;\;\frac{ x \in X }{ \mathsf{G}[X,V]; \emptyset \vdash x}Variation\]

\[\frac{ \mathsf{G}[X,V]; \Gamma \vdash g \;\; \mathsf{G}[X,V]; () \vdash v }{ \mathsf{G}[X,V]; \Gamma \vdash v | g}Participation \; \; \frac{ \mathsf{G}[X,V]; \Gamma \vdash g \;\; \mathsf{G}[X,V]; () \vdash x }{ \mathsf{G}[X,V]; \Gamma, x \vdash x | g}Dependence\]

\[\frac{ \mathsf{G}[X,V]; \Gamma_1 \vdash g_1 \; \mathsf{G}[X,V]; \Gamma_2 \vdash g_2}{ \mathsf{G}[X,V]; \Gamma_1, \Gamma_2 \vdash g_1 \otimes g_2}Juxtaposition[\Gamma_1 \cap \Gamma_2 = \emptyset]\]

\[\frac{ \mathsf{G}[X,V]; \Gamma,x \vdash g \; \;\mathsf{G}[X,V]; () \vdash v}{ \mathsf{G}[X,V]; \Gamma \vdash \mathsf{let}\; x = v \; \mathsf{in}\; g}Nomination\]

\[\frac{ \mathsf{G}[X,V]; \Gamma_1 \vdash \mathsf{let}\; x_1 = v_1 \; \mathsf{in}\; g_1 \; \;\mathsf{G}[X,V]; \Gamma_2 \vdash \mathsf{let}\; x_2 = v_2 \; \mathsf{in}\; g_2}{ \mathsf{G}[X,V]; \Gamma_1,\Gamma_2 \vdash \langle \mathsf{let}\; x_1 = v_1 \; \mathsf{in}\; g_1, \mathsf{let}\; x_2 = v_2 \; \mathsf{in}\; g_2 \rangle}Connection[\Gamma_1 \cap \Gamma_2 = \emptyset]\]

\hypertarget{interpretation}{%
\subsubsection{Interpretation}\label{interpretation}}

\begin{itemize}
\item
  The \(Foundation\) rule says that regardless of the theory of vertices
  or variables, the empty graph is always well formed.
\item
  The rules \(Verticity\) and \(Variation\) say that admissibility
  derives from the effective notion of membership required of the theory
  of vertices and the theory variables.
\item
  The rule for \(Participation\) says that if \(\mathsf{G}[X,V]\) thinks
  that \(g\) is well formed, given \(\Gamma\), and \(\mathsf{G}[X,V]\)
  thinks that \(v\) is admissible as a vertex, then \(\mathsf{G}[X,V]\)
  thinks that adjoining \(v\) to \(g\), i.e.~\(v|g\), is well formed.
\item
  The rule for \(Dependence\) is similar to \(Participation\). It says
  that \(\mathsf{G}[X,V]\) thinks that it can adjoin \(x\) to \(g\),
  i.e.~\(x|g\) to get a well formed graph given \(\Gamma\) if
  \(\mathsf{G}[X,V]\) thinks that \(g\) is well formed, given
  \(\Gamma,x\).
\item
  The rule for \(Juxtaposition\) says that simply juxtaposing two well
  formed graphs results in a well formed graph. A careful accounting
  will observe that the dependencies of the two graphs must be disjoint.
\item
  Logicians and computer scientists will recognize the rule for
  \(Nomination\) as a kind of cut rule. It's like \(\beta\)-reduction
  with an explicit substitution. Specifically, supposing that
  \(\mathsf{G}[X,V]\) thinks that \(g\) is well formed, given
  \(\Gamma,x\) and that \(\mathsf{G}[X,V]\) thinks that \(v\) is an
  admissible vertex then \(\mathsf{G}[X,V]\) thinks that letting \(x\)
  stand for \(v\) in \(g\) is well formed, given \(\Gamma\). A graph
  given by an expression of this form is said to be in nominated form.
\item
  Finally, the rule for \(Connection\) says that an edge is only
  admissible between two well formed graphs in nominated form and then
  only when the dependencies are disjoint.
\end{itemize}

Notice that the intuitive interpretation for \(Nomination\) matches the
pronunciation.

\[\frac{ \mathsf{G}[X,V];\Gamma,x \vdash g \; \;\mathsf{G}[X,V]; () \vdash v}{ \mathsf{G}[X,V]; \Gamma \vdash \mathsf{let}\; x = v \; \mathsf{in}\; g}Nomination\]

``Nomination concludes that \(\mathsf{G}[X,V]\) thinks that the graph
that lets \(x\) stand for \(v\) in \(g\) is well formed given \(\Gamma\)
if \(\mathsf{G}[X,V]\) thinks that \(g\) is well formed given
\(\Gamma,x\), and \(\mathsf{G}[X,V]\) thinks that \(v\) is an admissible
vertex.''

For clarity we will often write \(v|0\) as \([v]\).

\hypertarget{membership}{%
\subsection{Membership}\label{membership}}

We can extend the effective membership relation provided by \(V\) to one
on \(\mathsf{G}[X,V]\). Written \(\mathsf{G}[X,V] \vdash v \in g\), the
algorithm is given by the following set of rules

\[\frac{ }{ \mathsf{G}[X,V] \vdash v \in v | g}Ground\]

\[\frac{ \mathsf{G}[X,V] \vdash v \in g }{ \mathsf{G}[X,V] \vdash v \in g \otimes g'}Union\]

\[\frac{ \mathsf{G}[X,V] \vdash v \in g }{ \mathsf{G}[X,V] \vdash v \in \mathsf{let}\; x = v \; \mathsf{in}\; g}Transparency\]

\[\frac{ \mathsf{G}[X,V] \vdash v \in g_1 }{ \mathsf{G}[X,V] \vdash v \in \langle\mathsf{let}\; x_1 = v_1 \; \mathsf{in}\; g_1, \mathsf{let}\; x_2 = v_2 \; \mathsf{in}\; g_2\rangle}Link_L\]

\[\frac{ \mathsf{G}[X,V] \vdash v \in g_2 }{ \mathsf{G}[X,V] \vdash v \in \langle\mathsf{let}\; x_1 = v_1 \; \mathsf{in}\; g_1, \mathsf{let}\; x_2 = v_2 \; \mathsf{in}\; g_2\rangle}Link_R\]

As a matter of convenience we write
\(\mathsf{G}[X,V] \vdash v \not\in g\) to mean that it is not the case
that \(\mathsf{G}[X,V] \vdash v \in g\). The inquisitive reader is
invited to write a similar set of rules for edge membership. Of course,
given a graph, \(g\), it is very useful to refer to its collection of
vertices and its collection of edges, which we denote \(\mathsf{v}(g)\)
and \(\mathsf{e}(g)\) respectively.

\hypertarget{equations}{%
\subsection{Equations}\label{equations}}

The syntactic theory is too fine grained. It makes syntactic
distinctions that do not correspond to distinct graphs. Juxtaposition is
a great example. It corresponds to the disjoint sum of two graphs and as
such the order of juxtaposition should not matter. We erase these
syntactic distinctions with a set of equations on the graph expressions.

\[\frac{ }{ \mathsf{G}[X,V] \vdash 0 \otimes g = g}Identity\]

\[\frac{ }{ \mathsf{G}[X,V] \vdash g_1 \otimes g_2 = g_2 \otimes g_1}Symmetry\]

\[\frac{ }{ \mathsf{G}[X,V] \vdash g_1 \otimes ( g_2 \otimes g_3 ) = ( g_1 \otimes g_2) \otimes g_3}Associativity\]

\[\frac{ }{ \mathsf{G}[X,V] \vdash v_1|\ldots|v_i|v_j|g = v_1|\ldots|v_j|v_i|g}Permutation\]

\[\frac{ }{ \mathsf{G}[X,V] \vdash \mathsf{let}\; x = v \; \mathsf{in}\; \mathsf{let}\; x' = v \; \mathsf{in}\; g = \mathsf{let}\; x = v \; \mathsf{in}\; g}Conservation\]

\[\frac{ \mathsf{G}[X,V] \vdash x \notin g_2}{ \mathsf{G}[X,V] \vdash (\mathsf{let}\; x = v \; \mathsf{in}\; g_1) \otimes g_2 = \mathsf{let}\; x = v \; \mathsf{in}\; ( g_1 \otimes g_2)}Extrusion\]

Note the mutual dependency amongst \(Extrusion\) and \(Permutation\) and
the extended \(\in\) relation.

\hypertarget{examples}{%
\section{Examples}\label{examples}}

\hypertarget{graphs-as-morphism}{%
\subsection{Graphs as morphism}\label{graphs-as-morphism}}

Let \(X\) be some set of variables, \(D\) be some domain, such as
\(Nat\), the natural numbers, or \(Bool\), the booleans. We can describe
many graphs as morphisms, \(g(X,D): D \to \mathsf{G}[X,D]\).

\hypertarget{the-discrete-graph}{%
\subsubsection{The discrete graph}\label{the-discrete-graph}}

The discrete graph of \(n\) elements, written \(\mathsf{discrete}(n)\),
is simply \(n\) unconnected vertices. Noting that the graph consisting
of a single vertex, \(n\), has the form \(n \to 0\), which adjoins \(n\)
to the empty graph, \(0\), we can define the domain of discrete graphs
recursively.

\(\mathsf{discrete}(0) = 0\)

\(\mathsf{discrete}(n) = n | 0 \otimes \mathsf{discrete}(n-1)\)

\(= [n]\otimes \mathsf{discrete}(n-1)\)

Thus, \(\mathsf{discrete}(n) = [n]\otimes[n-1]\otimes\cdots\otimes0\)

\hypertarget{domain-elements-as-variables}{%
\subsubsection{Domain elements as
variables}\label{domain-elements-as-variables}}

Sometimes it is convenient to use domain elements as variables as well
as vertices; that is, it is often useful to form graphs from
\(\mathsf{G}[D,D]\). In this case, we distinguish the mention of a
domain element, say \(d\), in the role of a variable from the mention in
the role of a vertex by quotation marks. Specifically, we write
\(\ulcorner d\urcorner\) when using \(d\) as a variable versus merely
\(d\) when using \(d\) as a vertex. An example in context this might
look like
\(\mathsf{let}\; \ulcorner d\urcorner = d \; \mathsf{in}\; g\).

\hypertarget{the-chain}{%
\subsubsection{The chain}\label{the-chain}}

Using these two ideas we can give a recursive definition of the domain
of chains. As with the discrete graph, the chain of \(0\) elements is
the empty graph. The chain of a single element is not simply adjoining
that element as a vertex to the empty graph. It also selects that
element. Then the recursive specification creates an edge between the
graph containing a single element, \(n \to 0\), and the chain of \(n-1\)
elements by selecting the element, \(n\) and calling it
\(\ulcorner 2n-1\urcorner\), and then linking the two resulting graphs.
To ensure that the result is linkable it selects the element again, this
time calling it \(\ulcorner 2n \urcorner\).

\(\mathsf{chain}(0) = 0\)

\(\mathsf{chain}(1) = \mathsf{let}\; \ulcorner 2\urcorner = 1 \; \mathsf{in}\; [1]\)

\(\mathsf{chain}(n) = \mathsf{let}\; \ulcorner2n\urcorner = n \; \mathsf{in}\;(\mathsf{let}\; \ulcorner 2n-1\urcorner = n \; \mathsf{in}\; [n], \mathsf{chain}(n-1))\)

The reader can easily verify that for each \(n \geq 1\)
\(\mathsf{chain}(n)\) is in nominated form.

Here is the chain from \(2\) to \(1\).

\[\mathsf{chain}(2) = \mathsf{let}\; \ulcorner 4 \urcorner = 2 \; \mathsf{in}\; (\mathsf{let}\; \ulcorner 3 \urcorner = 2 \; \mathsf{in}\; [2], \mathsf{let}\; \ulcorner 2\urcorner = 1 \; \mathsf{in}\; [1])\]

\hypertarget{the-cycle}{%
\subsubsection{The cycle}\label{the-cycle}}

The cyclic graph of \(n\) vertices and \(n\) edges simply creates an
additional edge from the end of the \(\mathsf{chain}(n)\) back to the
beginning.

\(\mathsf{cycle}(0) = 0\)

\(\mathsf{cycle}(n) = \langle \mathsf{chain}(n), \mathsf{chain}(1) \rangle\)

\hypertarget{the-complete-graph}{%
\subsubsection{The complete graph}\label{the-complete-graph}}

This next example is made even more compact if we admit some syntactic
sugar. First, let us stipulate that \(D\) supports a means of checking
elements, such as \(d\), for properties, say \(p\), and write \(p(d)\)
for the predicate that determines when \(d\) inhabits the property
\(p\). We can lift the edge expression to an expression between
collections of elements from \(D\). Specifically, we can form

\[\langle \mathsf{for} ( x_1 \leftarrow \mathsf{v}(g_1)\; \mathsf{if}\; p_1(x_1) )\{ g_1 \}, \mathsf{for} ( y \leftarrow \mathsf{v}(g_2)\; \mathsf{if}\; p_2(x_2) )\{ g_2 \} \rangle\]

which adds to the graph \(g_1 \otimes g_2\) an edge from each vertex in
\(\{ v \in \mathsf{v}(g_1) | p_1(v) \}\) to each vertex in
\(\{ v \in \mathsf{v}(g_2) | p_2(v) \}\)

When \(|\{ v \in \mathsf{v}(g_i) | p_i(v) \}| = 1\) then this form is
the same as

\[\langle \mathsf{let}\; x_1 = c_1[0] \; \mathsf{in}\; g_1 , \mathsf{let}\; x_2 = c_2[0] \; \mathsf{in}\; g_2 \rangle\]

where \(c_i[0]\) denotes the singleton element inhabiting
\(\{ v \in \mathsf{v}(g_i) | p_i(v) \}|\).

With this syntactic sugar we have a 1-line specification for the
complete graph.

\(\mathsf{complete}(0) = 0\)

\(\mathsf{complete}(n) = \langle \mathsf{for} ( x \leftarrow \mathsf{v}(\mathsf{discrete(1)}))\{ \mathsf{discrete(1)} \}, \mathsf{for} ( y \leftarrow \mathsf{v}(\mathsf{complete}(n-1)))\{ \mathsf{complete}(n-1) \} \rangle\)

These examples illustrate our main motivation for introducing the theory
of graphs. As we move from discrete to chain to cycle to complete graphs
the complexity of the graph structure grows dramatically; and yet the
complexity of the recursive specifications of the graphs remains almost
constant.

It is also important to recognize that each of these examples are also
subdomains of their respective graph domains. For example,

\[\{ \mathsf{chain(i)} | i \in \mathsf{Nat}\} \subset \mathsf{G}[\mathsf{Nat},\mathsf{Nat}]\]

In this sense we can think of \(\mathsf{G}[\mathsf{Nat},\mathsf{Nat}]\)
as a container domain for
\(\{ \mathsf{chain(i)} | i \in \mathsf{Nat}\}\). More standard language
would dub the latter a subtype of the former. In each of these examples
we arrive at the subdomain or subtype of the container domain by
imposing additional constraints on edge formation. In fact, each of
these examples and infinitely many more can be computed either directly
as we have done or as filters on their container domains, or super
types, that selects graphs that adhere to the edge criteria or they can
be computed by adjoining to the theory the additional constraints on
edges. We will formalize this idea in a subsequent section.

The cycle and complete graph examples raise an important point about
graph equality and references to whole graphs. In the section below we
develop a notion of graph references to allow for more sophisticated
compositional and recursive specification of graphs.

\hypertarget{graph-references}{%
\subsubsection{Graph references}\label{graph-references}}

In many instances it it useful to refer not only to vertices but whole
graphs. For this we assume that the collection of variables is actually
divided into two distinct sub-collections; that is, we require
\(X = X_v + X_g\). Notationally, we will use lowercase letters such as
\(x\), \(y\), \(z\) to range over \(X_v\) and uppercase letters such as
\(A\), \(B\), \(C\), etc to range over \(X_g\). When we are building
graphs using references to graphs we need to expand the theory to
include more common cyclic definitions as is typically found in
functional languages involving \(\mathsf{letrec}\).

Just as we separate the collection of variables into variables for
vertices and variables for graphs, we separate the functional
dependencies. This requires expanding the basic judgment to include
graph variable dependencies on the left hand side of the turnstile:
\(\mathsf{G}[X,V]; \Gamma; \mathcal{G} \vdash g\). This expanded form
allows mentions of \(A\), \(B\), etc to occur in \(g\), which can be
discharged later and gives rise to our first new rule. Just as with
\(\Gamma\), we use \(\mathcal{G}\) to range over comma separated
sequences of graph variables. This allows us to form wires.

\[\frac{ }{ \mathsf{G}[X,V]; \Gamma; A \vdash A}Wire\]

Equipped with this rule we can form graphs that have ``holes'' in them.
Note that we could have achieved the same effect by calculating a graph
context type from our graph type. However, with this next rule we allow
bundling of wires and establish that our contexts have multiple holes.

\[\frac{ \mathsf{G}[X,V]; \Gamma_1; \mathcal{G}_1 \vdash g_1\; \mathsf{G}[X,V]; \Gamma_2; \mathcal{G}_2 \vdash g_2 }{ \mathsf{G}[X,V]; \mathcal{G}_1, \mathcal{G}_2 \vdash g_1 \otimes g_2 }Bundle [\Gamma_1 \cap \Gamma_2 = \emptyset = \mathcal{G}_1\cap\mathcal{G}_2 ]\]

We can define a cut rule for graph variables just as we defined a cut
rule for vertex variables.

\[\frac{ \mathsf{G}[X,V]; \Gamma_1; \mathcal{G}_1 \vdash g_1 \; \;\mathsf{G}[X,V]; \Gamma_2; \mathcal{G}_2, B \vdash g_2}{ \mathsf{G}[X,V]; \vec{A}, \vec{C} \vdash \mathsf{let}\; B = g_1 \; \mathsf{in}\; g_2}Cut\]

This in turn makes it possible to describe a wide range of graphs with
recursive structure. In particular, prior to the addition of graph
references the theory describes only \emph{finite} graphs built from
\(X\) and \(V\). With the addition of graph references the graphs can be
\emph{infinitary}. In the sequel we will refer to graphs that make no
mention of graph variables as \emph{ground}. Likewise, if graph \(g\)
enjoys wellformedness without any dependencies,
i.e.~\(\mathsf{G}[X,V]; (); () \vdash g\), we say \(g\) is
\emph{closed}.

\hypertarget{rewrite-systems}{%
\subsubsection{Rewrite systems}\label{rewrite-systems}}

\hypertarget{the-lambda-calculus}{%
\paragraph{\texorpdfstring{The
\(\lambda\)-calculus}{The \textbackslash lambda-calculus}}\label{the-lambda-calculus}}

Let \(M\) denote the set of terms in the \(\lambda\)-calculus, and
\(t\), \(u\), \(\ldots\) range over terms in \(M\). We form graphs of
the reductions of terms, \(\mathsf{G}[M,M]\) by filtering edges for
\(\beta\)-reduction. To avoid confusion in the sequel we will write
\(\ulcorner t\urcorner\) when using \(t\) as a variable, and unadorned
when considering it as a term. More concretely, given

\[\uparrow t = \mathsf{let}\; \ulcorner t \urcorner = t \; \mathsf{in}\; [\ulcorner t \urcorner] \]

we lift terms to graphs using

\[\frac{ }{ \mathsf{G}[M,M]; (); () \vdash \uparrow t }Embedding\]

and then filter edges by the condition

\[\frac{ \mathsf{G}[M,M]; (); () \vdash \uparrow t \; \;\mathsf{G}[M,M]; (); () \vdash \uparrow u \;\; t \to u}{ \mathsf{G}[X,V]; (); () \vdash \langle \uparrow t,\uparrow u \rangle}Reduction\]

This gives rise to the graph of reductions of a term,
\(\mathsf{red}(t)\) which is defined by

\(\mathsf{red}(x) = \uparrow x\)

\(\mathsf{red}(\lambda x.t) = \uparrow (\lambda x.t)\)

\(\mathsf{red}((\lambda x.t)u) = \langle \uparrow ((\lambda x.t)u),\mathsf{red}(t[u/x])\rangle\)

\(\mathsf{red}(tu) = \langle \uparrow (tu),\mathsf{for}( t' \leftarrow \mathsf{v}( \mathsf{red}(t) )\{\mathsf{red}(t'u)\}\rangle\)

\hypertarget{categories}{%
\subsubsection{Categories}\label{categories}}

We can provide a theory of categories, \(\mathcal{C}[X,V]\), that are
subdomains of \(\mathsf{G}[X,V]\).

\hypertarget{reflection-and-the-syntax-of-a-graph}{%
\subsubsection{Reflection and the syntax of a
graph}\label{reflection-and-the-syntax-of-a-graph}}

In some sense the entire point of the previous sections is to provide
proof that the domain \(\mathsf{G}[X,V]\) serves equally well as theory
of variables or as a theory of vertices, provided that \(X\) and \(V\)
serve those roles respectively. That is, \(\mathsf{G}[X,V]\) comes with
an effective procedure for deciding when a graph, say \(g\) is in the
domain, \(\mathsf{G}[X,V]\). Specifically, we define
\(g \in \mathsf{G}[X,V]\) to be the determination that
\(\mathsf{G}[X,V];();() \vdash g\), i.e.~that \(g\) is closed in
\(\mathsf{G}[X,V]\). Likewise, \(\mathsf{G}[X,V]\) comes with an
effective procedure for determining if two graphs are equal. This
observation yields additional expressive power. We can consider graph
domains that are recursively defined.

\[\mathcal{D}[X,V] = \mathsf{G}[X + \mathcal{D}[X,V],V + \mathcal{D}[X,V]]\]

Domains like \(\mathcal{D}[X,V]\) allow us to create graphs in which we
freely mix vertices from \(V\) with vertices made out of entire graphs.
As an example, with the recursive domain we given a graph
\(g \in \mathsf{G}[X,V]\) we can compute the graph of its syntax,
\(\mathsf{s}(g) \in \mathcal{D}[X,V]\).

\(\mathsf{s}( 0 ) = \ulcorner 0 \urcorner | 0\)

\(\mathsf{s}( v | g ) = (\mathsf{let}\; \ulcorner g \urcorner = g \; \mathsf{in}\; \ulcorner g \urcorner | \mathsf{s}(g),\mathsf{let}\; \ulcorner v \urcorner = v \; \mathsf{in}\; \ulcorner v \urcorner | 0)\)

\(\mathsf{s}( g_1 \otimes g_2 )\)

\(=\)

\((\mathsf{let}\; \ulcorner g_1 \urcorner = g_1 \; \mathsf{in}\; \ulcorner g_1 \urcorner | \mathsf{s}(g_1),\mathsf{let}\; \ulcorner g_2 \urcorner = g_2 \; \mathsf{in}\; \ulcorner g_2 \urcorner | \mathsf{s}(g_2))\)

\(\mathsf{s}( \mathsf{let}\; x = v \;\mathsf{in}\; g )\)

\(=\)

\((\mathsf{let}\; \ulcorner g \urcorner = g \; \mathsf{in}\; \ulcorner g \urcorner | \mathsf{s}(g),\mathsf{let}\; \ulcorner x \urcorner = x \; \mathsf{in}\; \mathsf{let} \; \ulcorner v \urcorner = v \; \mathsf{in}\; \ulcorner x \urcorner | \ulcorner v \urcorner | 0)\)

\(\mathsf{s}((\mathsf{let}\; x_1 = v_1 \;\mathsf{in}\; g_1, \mathsf{let}\; x_2 = v_2 \;\mathsf{in}\; g_2))\)

\(=\)

\((\mathsf{let}\; \ulcorner \mathsf{let}\; x_1 = v_1 \;\mathsf{in}\; g_1 \urcorner = \mathsf{let}\; x_1 = v_1 \;\mathsf{in}\; g_1 \; \mathsf{in}\; \ulcorner \mathsf{let}\; x_1 = v_1 \;\mathsf{in}\; g_1 \urcorner | \mathsf{s}(g_1),\)

\(\;\mathsf{let}\; \ulcorner \mathsf{let}\; x_2 = v_2 \;\mathsf{in}\; g_2 \urcorner = \mathsf{let}\; x_2 = v_2 \;\mathsf{in}\; g_2 \; \mathsf{in}\; \ulcorner \mathsf{let}\; x_2 = v_2 \;\mathsf{in}\; g_2 \urcorner | \mathsf{s}(g_2))\)

It is easy to see that
\(\mathcal{S[X,V] = \{ \mathsf{s}(g) | g \in \mathcal{D}[X,V] \}}\) is a
subtype of \(\mathcal{D}[X,V]\).

And given the syntax of a graph we can always recover the original
graph.

\hypertarget{oslf-and-graphs}{%
\subsection{OSLF and Graphs}\label{oslf-and-graphs}}

We can apply the \(\mathsf{OSLF}\) procedure to the theory of graphs.
When the collection is a set then the types are given as

\[\phi, \psi ::= \mathsf{true}\;|\; \phi \;\mathsf{and}\; \psi \; |\; 0 \;|\; v|\phi \;|\; \mathsf{let}\; x = v \; \mathsf{in}\; \phi \;|\; \langle \mathsf{let}\; x_1 = v_1 \; \mathsf{in}\; \phi, \mathsf{let}\; x_2 = v_2 \; \mathsf{in}\; \psi\rangle\]

This sets up our foundational investigation. Recall that our examples
\(\mathsf{discrete}\), \(\mathsf{chain}\), \(\mathsf{cycle}\), and
\(\mathsf{complete}\) gave rise to subtypes of
\(\mathsf{G}[\mathsf{Nat},\mathsf{Nat}]\) by exploiting structure of
\(\mathsf{Nat}\). Similarly, the \(\mathsf{OSLF}\) types give rise to
subtypes of \(\mathsf{G}[X,V]\). Our focus of investigation is the
question of the conditions under which the structure of \(X\) and \(V\)
supply subtypes of \(\mathsf{G}[X,V]\) that coincide with the
\(\mathsf{OSLF}\) types?

Not surprisingly, we find that when we can faithfully map \(X\) (resp.,
\(V\)) into \(\mathsf{G}[X',V']\) for some \(X'\) and \(V'\), then we
can establish an isomorphism between the subtypes available from the
structure of \(X\) and \(V\) and the \(\mathsf{OSLF}\) types. This
observation sets up a process by which domains of \(X\) and \(V\) are
replaced by graph-based representations of them. The limit of this
process, the smallest structure for which these coincide, is given by

\[\mathcal{D} = \mathsf{G}[1 + \mathcal{D},1 + \mathcal{D}]\]

This domain is the least witness of the microcosm principle applied to
the domain of graphs.

\hypertarget{miscellaneous-operations}{%
\subsection{Miscellaneous operations}\label{miscellaneous-operations}}

\[\mathsf{G}^{\circ}[X,V] = \mathsf{G}[V,X]\]

\[\mathsf{G}^{\mathsf{V}}[X,V] = \mathsf{G}[V,V]\]

\[\mathsf{G}^{\mathsf{X}}[X,V] = \mathsf{G}[X,X]\]

\[\mathsf{G}[X_1,V_1] + \mathsf{G}[X_2,V_2] = \mathsf{G}[X_1+X_2,V_1 + V_2]\]

\[\mathsf{G}[X_1,V_1] \times \mathsf{G}[X_2,V_2] = \mathsf{G}[X_1 \times X_2,V_1 \times V_2]\]

\[\mathsf{G}[X_1,V_1] \cdot \mathsf{G}[X_2,V_2] = \mathsf{G}[X_1 + V_1,X_2 + V_2]\]

Note that the crossover style operations like ``\(\cdot\)'' together
with reflection allow us to define \emph{evolutionary} processes (in the
sense of Holland \cite{Holland1975}) for graphs and graph domains. The principal missing
ingredient is a fitness function for evaluating graphs and graph domains
for selection and ``reproduction.''

\section{Conclusions and future work}

We have presented a formal theory of graphs that enjoys good
complexity properties by comparison to other formal models.

% subsection other_calculi_other_bisimulations_and_geometry_as_behavior (end)




\paragraph{Acknowledgments.}
The author wishes to thank Phillipa Gardner, Giorgio Ghelli, and Luca
Cardelli for their inspiring work which lead to these insights.


\bibliographystyle{plain}   
\bibliography{togl.bib}

\end{document}
