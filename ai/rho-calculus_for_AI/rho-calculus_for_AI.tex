% +------------------------------------+
% |   Generated by www.docx2latex.com  |
% |   Version: 2.0.0                   |
% +------------------------------------+

%\documentclass[10pt]{report}
\documentclass[runningheads]{llncs}

\usepackage {mathpartir}
\usepackage{bigpage}
\usepackage{bcprules}
\usepackage{amsmath}
\usepackage{amssymb}
\usepackage{amsfonts}
\usepackage{amstext}
\usepackage{latexsym}
\usepackage{longtable}
\usepackage{stmaryrd}
\usepackage[T1]{fontenc}
\usepackage[utf8]{inputenc}
\usepackage[normalem]{ulem}
\usepackage[svgnames]{xcolor}
\usepackage[paperheight=27.94cm,paperwidth=21.59cm,left=2.54cm,right=2.54cm,top=2.54cm,bottom=2.54cm]{geometry}
\usepackage[hidelinks]{hyperref}


\setlength\parindent{0pt}
\renewcommand{\arraystretch}{1.3}

% Double brackets
\newcommand{\ldb}{[\![}
\newcommand{\rdb}{]\!]}
\newcommand{\ldrb}{(\!(}
\newcommand{\rdrb}{)\!)}
\newcommand{\lrbb}{(\!|}
\newcommand{\rrbb}{|\!)}
\newcommand{\lliftb}{\langle\!|}
\newcommand{\rliftb}{|\!\rangle}
\newcommand{\lrcb}{\{\!\!\{}
\newcommand{\rrcb}{\}\!\!\}}
%\newcommand{\plogp}{:\!-}
\newcommand{\plogp}{\leftarrow}
%\newcommand{\plogp}{\coloneq}
% \newcommand{\lpquote}{\langle}
% \newcommand{\rpquote}{\rangle}
% \newcommand{\lpquote}{\lceil}
% \newcommand{\rpquote}{\rceil}
\newcommand{\lpquote}{\ulcorner}
\newcommand{\rpquote}{\urcorner}
\newcommand{\newkw}{\nu}

% SYNTAX
\newcommand{\id}[1]{\texttt{#1}}
\newcommand{\none}{\emptyset}
\newcommand{\eps}{\epsilon}
\newcommand{\set}[1]{\{#1\}}
\newcommand{\rep}[2]{\id{\{$#1$,$#2$\}}}
\newcommand{\elt}[2]{\id{$#1$[$#2$]}}
\newcommand{\infinity}{$\infty$}

\newcommand{\pzero}{\mathsf{0}}
\newcommand{\seq}{\mathsf{\id{,}}}
\newcommand{\all}{\mathsf{\id{\&}}}
\newcommand{\choice}{\mathsf{\id{|}}}
\newcommand{\altern}{\mathsf{\id{+}}}
\newcommand{\juxtap}{\mathsf{\id{|}}}
\newcommand{\concat}{\mathsf{.}}
\newcommand{\punify}{\mathsf{\id{:=:}}}
\newcommand{\fuse}{\mathsf{\id{=}}}
\newcommand{\scong}{\mathsf{\equiv}}
\newcommand{\nameeq}{\mathsf{\equiv_N}}
\newcommand{\alphaeq}{\mathsf{\equiv_{\alpha}}}
\newcommand{\arrvec}[1]{\overrightarrow{#1}}
\newcommand{\names}[1]{\mathsf{N}(#1)}
\newcommand{\freenames}[1]{\mathsf{FN}(#1)}
\newcommand{\boundnames}[1]{\mathsf{BN}(#1)}
%\newcommand{\lift}[2]{\texttt{lift} \; #1 \concat #2}
\newcommand{\binpar}[2]{#1 \mathsf{|} #2}
\newcommand{\binparx}[2]{\mathsf{par}(#1, #2)}
\newcommand{\binpart}[2]{\mathbf{par}(\mathbf{#1}, \mathbf{#2})}
\newcommand{\binpartl}[2]{\mathbf{par}_{L}(\mathbf{#1}, \mathbf{#2})}
\newcommand{\binpartr}[2]{\mathbf{par}_{R}(\mathbf{#1}, \mathbf{#2})}
\newcommand{\outputp}[2]{#1\mathsf{!}(#2)}
\newcommand{\outputt}[2]{\mathbf{#1}\mathsf{!}(\mathbf{#2})}
\newcommand{\prefix}[3]{\mathsf{for(}#2 \leftarrow #1 \mathsf{)} #3}
\newcommand{\prefixt}[3]{\mathbf{for}\mathbf{(}\mathbf{#2} \leftarrow \mathbf{#1} \mathbf{)} \mathbf{#3}}
%%\newcommand{\lift}[2]{#1 \lliftb #2 \rliftb}
\newcommand{\lift}[2]{#1 \mathsf{!}(#2)}
\newcommand{\clift}[1]{\lliftb #1 \rliftb}
\newcommand{\quotep}[1]{\mathsf{@} #1}
\newcommand{\quotet}[1]{\mathbf{@} \mathbf{#1}}
\newcommand{\dropn}[1]{\mathsf{*} #1}
\newcommand{\dropt}[1]{\mathbf{*} \mathbf{#1}}
\newcommand{\procn}[1]{\stackrel{\cdot}{#1}}
\newcommand{\commr}[4]{\mathsf{comm(}#1, #2, #3, #4  \mathsf{)}}
\newcommand{\commt}[3]{\mathbf{comm(}\mathbf{#1}, \mathbf{#2}, \mathbf{#3} \mathbf{)}}

\newcommand{\newp}[2]{(\newkw \; #1 ) #2}
\newcommand{\bangp}[1]{! #1}

\newcommand{\substp}[2]{\{ \quotep{#1} / \quotep{#2} \}}
\newcommand{\substn}[2]{\{ #1 / #2 \}}
\newcommand{\msubstn}[2]{\lrcb #1 / #2 \rrcb}

\newcommand{\psubstp}[2]{\widehat{\substp{#1}{#2}}}
\newcommand{\psubstn}[2]{\widehat{\substn{#1}{#2}}}

\newcommand{\applyp}[2]{#1 \langle #2 \rangle}
\newcommand{\absp}[2]{( #1 ) #2}
\newcommand{\annihilate}[1]{#1^{\times}}
\newcommand{\dualize}[1]{#1^{\bullet}}
\newcommand{\ketp}[1]{\mathsf{|}#1 \rangle}
\newcommand{\brap}[1]{\langle #1 \mathsf{|}}
\newcommand{\testp}[2]{\langle #1 \mathsf{|} #2 \rangle}
\newcommand{\testnp}[3]{\langle #1 \mathsf{|} #2 \mathsf{|} #3 \rangle}

\newcommand{\transitions}[3]{\mathbin{#1 \stackrel{#2}{\longrightarrow} #3}}
\newcommand{\meaningof}[1]{\ldb #1 \rdb}
\newcommand{\pmeaningof}[1]{\ldb #1 \rdb}
\newcommand{\nmeaningof}[1]{\lrbb #1 \rrbb}

\newcommand{\Proc}{\mathsf{Proc}}
\newcommand{\QProc}{\quotep{\mathsf{Proc}}}
\newcommand{\MixSumProc}{\mathsf{MixSumProc}}

\newcommand{\entailm}{\mathbin{\vdash_{\mathfrak m}}} %matching
\newcommand{\entailp}{\mathbin{\vdash_{\mathfrak p}}} %behavioral
\newcommand{\entailv}{\mathbin{\vdash_{\mathfrak v}}} %validation
\newcommand{\congd}{\mathbin{\equiv_{\mathfrak d}}}
\newcommand{\congs}{\mathbin{\equiv_{\mathfrak s}}}
\newcommand{\congp}{\mathbin{\equiv_{\mathfrak p}}}
%\newcommand{\logequiv}{\mathbin{\leftrightarrow}}

\newcommand{\barb}[2]{\mathbin{#1 \downarrow_{#2}}}
\newcommand{\dbarb}[2]{\mathbin{#1 \Downarrow_{#2}}}

% From pi-duce paper
\newcommand{\red}{\rightarrow}
\newcommand{\wred}{\Rightarrow}
\newcommand{\redhat}{\hat{\longrightarrow}}
\newcommand{\lred}[1]{\stackrel{#1}{\longrightarrow}} %transitions
\newcommand{\wlred}[1]{\stackrel{#1}{\Longrightarrow}}

\newcommand{\opm}[2]{\overline{#1} [ #2 ]} % monadic
\newcommand{\ipm}[2]{{#1} ( #2 )} 
\newcommand{\ipmv}[2]{{#1} ( #2 )} % monadic
\newcommand{\parop}{\;|\;}		% parallel operator
\newcommand{\patmatch}[3]{#2 \in #3 \Rightarrow #1}
\newcommand{\sdot}{\, . \,}		% Space around '.'
\newcommand{\bang}{!\,}
%\newcommand{\fuse}[1]{\langle #1 \rangle}		
\newcommand{\fusion}[2]{#1 = #2} % fusion prefix/action
\newcommand{\rec}[2]{\mbox{\textsf{rec}} \, #1. \, #2}
\newcommand{\match}[2]{\mbox{\textsf{match}} \; #1 \; \mbox{\textsf{with}} \; #2}
\newcommand{\sep}{:}
\newcommand{\val}[2]{\mbox{\textsf{val}} \; #1 \; \mbox{\textsf{as}} \; #2}

\newcommand{\rel}[1]{\;{\mathcal #1}\;} %relation
\newcommand{\bisim}{\stackrel{.}{\sim}_b} %bisimilar
\newcommand{\wb}{\approx_b} %weak bisimilar
\newcommand{\bbisim}{\stackrel{\centerdot}{\sim}} %barbed bisimilar
\newcommand{\wbbisim}{\stackrel{\centerdot}{\approx}} %weak barbed bisimilar
\newcommand{\bxless}{\lesssim}	%expansion less (amssymb required)
\newcommand{\bxgtr}{\gtrsim}	%expansion greater (amssymb required)
\newcommand{\beq}{\sim}		%barbed congruent
\newcommand{\fwbeq}{\stackrel{\circ}{\approx}}	%weak barbed congruent
\newcommand{\wbeq}{\approx}	%weak barbed congruent
\newcommand{\sheq}{\simeq}	%symbolic hypereq
\newcommand{\wbc}{\approx_{cb}}

% rho logic

\newcommand{\ptrue}{\mathbin{true}}
\newcommand{\psatisfies}[2]{#1 \models #2}
\newcommand{\pdropf}[1]{\rpquote #1 \lpquote}
\newcommand{\plift}[2]{#1 \lliftb #2 \rliftb}
\newcommand{\pprefix}[3]{\langle #1 ? #2 \rangle #3}
\newcommand{\pgfp}[2]{\textsf{rec} \; #1 \mathbin{.} #2}
\newcommand{\pquant}[3]{\forall #1 \mathbin{:} #2 \mathbin{.} #3}
\newcommand{\pquantuntyped}[2]{\forall #1 \mathbin{.} #2}
\newcommand{\riff}{\Leftrightarrow}

\newcommand{\PFormula}{\mathbin{PForm}}
\newcommand{\QFormula}{\mathbin{QForm}}
\newcommand{\PropVar}{\mathbin{\mathcal{V}}}

% qm notation
\newcommand{\state}[1]{\juxtap #1 \rangle}
\newcommand{\event}[1]{\langle #1 \juxtap}
\newcommand{\innerprod}[2]{\langle #1 \juxtap #2 \rangle}
\newcommand{\prmatrix}[2]{\juxtap #1 \rangle \langle #2 \juxtap}
\newcommand{\fprmatrix}[3]{\juxtap #1 \rangle #2 \langle #3 \juxtap}

% End piduce contribution

\newcommand{\typedby}{\mathbin{\:\colon}}
\newcommand{\mixedgroup}[1]{\id{mixed($#1$)}}
\newcommand{\cast}[2]{\id{CAST AS} \; #1 \; (#2)}
\newcommand{\bslsh}{\mathbin{\id{\\}}}
\newcommand{\bslshslsh}{\mathbin{\id{\\\\}}}
\newcommand{\fslsh}{\mathbin{\id{/}}}
\newcommand{\fslshslsh}{\mathbin{\id{//}}}
\newcommand{\bb}[1]{\mbox{#1}}
\newcommand{\bc}{\mathbin{\mathbf{::=}}}
\newcommand{\bm}{\mathbin{\mathbf\mid}}
%\newcommand{\bm}{\mathbin{\mathbf\mid}}
\newcommand{\be}{\mathbin{=}}
\newcommand{\bd}{\mathbin{\buildrel {\rm \scriptscriptstyle def} \over \be}}
%\newcommand{\category}[1]{\mbox{\bf #1}}

%GRAMMAR
\newlength{\ltext}
\newlength{\lmath}
\newlength{\cmath}
\newlength{\rmath}
\newlength{\rtext}

\settowidth{\ltext}{complex type name}
\settowidth{\lmath}{$xxx$}
\settowidth{\cmath}{$::=$}
\settowidth{\rmath}{\id{attributeGroup}}
\settowidth{\rtext}{repetition of $g$ between $m$ and $n$ times}

\newenvironment{grammar}{
  \[
  \begin{array}{l@{\quad}rcl@{\quad}l}
  \hspace{\ltext} & \hspace{\lmath} & \hspace{\cmath} & \hspace{\rmath} & \hspace{\rtext} \\
}{
  \end{array}\]
}

% Over-full v-boxes on even pages are due to the \v{c} in author's name
%\vfuzz2pt % Don't report over-full v-boxes if over-edge is small

% THEOREM Environments ---------------------------------------------------
% MATH -------------------------------------------------------------------
 \newcommand{\veps}{\varepsilon}
 \newcommand{\To}{\longrightarrow}
 \newcommand{\h}{\mathcal{H}}
 \newcommand{\s}{\mathcal{S}}
 \newcommand{\A}{\mathcal{A}}
 \newcommand{\J}{\mathcal{J}}
 \newcommand{\M}{\mathcal{M}}
 \newcommand{\W}{\mathcal{W}}
 \newcommand{\X}{\mathcal{X}}
 \newcommand{\BOP}{\mathbf{B}}
 \newcommand{\BH}{\mathbf{B}(\mathcal{H})}
 \newcommand{\KH}{\mathcal{K}(\mathcal{H})}
 \newcommand{\Real}{\mathbb{R}}
 \newcommand{\Complex}{\mathbb{C}}
 \newcommand{\Field}{\mathbb{F}}
 \newcommand{\RPlus}{\Real^{+}}
 \newcommand{\Polar}{\mathcal{P}_{\s}}
 \newcommand{\Poly}{\mathcal{P}(E)}
 \newcommand{\EssD}{\mathcal{D}}
 \newcommand{\Lom}{\mathcal{L}}
 \newcommand{\States}{\mathcal{T}}
 \newcommand{\abs}[1]{\left\vert#1\right\vert}
% \newcommand{\set}[1]{\left\{#1\right\}}
%\newcommand{\seq}[1]{\left<#1\right>}
 \newcommand{\norm}[1]{\left\Vert#1\right\Vert}
 \newcommand{\essnorm}[1]{\norm{#1}_{\ess}}

%%% NAMES
\newcommand{\Names}{{\mathcal N}}
\newcommand{\Channels}{{\sf X}}
\newcommand{\Variables}{{\mathcal V}}
\newcommand{\Enames}{{\mathcal E}}
\newcommand{\Nonterminals}{{\mathcal S}}
\newcommand{\Pnames}{{\mathcal P}}
\newcommand{\Dnames}{{\mathcal D}}
\newcommand{\Types}{{\mathcal T}}

\newcommand{\fcalc}{fusion calculus}
\newcommand{\xcalc}{${\mathfrak x}$-calculus}
\newcommand{\lcalc}{$\lambda$-calculus}
\newcommand{\pic}{$\pi$-calculus}
\newcommand{\spic}{spi-calculus}
\newcommand{\rhoc}{$\rho$-calculus}
\newcommand{\rhol}{$\rho$-logic}
\newcommand{\hcalc}{highwire calculus}
\newcommand{\dcalc}{data calculus}
%XML should be all caps, not small caps. --cb
%\newcommand{\xml}{\textsc{xml}}
\newcommand{\xml}{XML} 
\newcommand{\Opp}[1]{\mathsf{#1}^{\mathsf{op}}}
 

\begin{document}
\def\lastname{Meredith}

\title{Multiagent Systems and Reflection}
\titlerunning{rho4ai}

\date{2 March 2023}

\author{ Lucius Gregory Meredith\inst{1} \\
         \and 
         Ben Goertzel\inst{2}
}
\institute{
        {CEO, F1R3FLY.io 9336 California Ave SW, Seattle, WA 98103, USA} \\
  \email{ f1r3fly.ceo.com } \\
  \and
  {CEO, SingularityNet} \\
  \email{ ben@singularitynet.io } 
}
 

\maketitle              % typeset the title of the contribution

%%% ----------------------------------------------------------------------

\begin{abstract}

  We make the case for a reflective multi-agent formalism.

\end{abstract}

%\keywords{Process calculi, knots, invariants}

% \begin{keyword}
% concurrency, message-passing, process calculus, reflection, program logic
% \end{keyword}

%\end{frontmatter}


\section{Multi-agent System and AI}
Multi-agent systems are at least as old as AI, and have enjoyed active research in a number of fields. Among the various research programmes, concurrency theory has produced one of the most advanced versions of mult-agent systems. More specifically, the mobile process calculi formalize a notion of multi-agent systems that is simultaneously a foundation for computing, in general, and has been used in a wide range of successful applications. Surprisingly, the AI community has not adopted these formalisms, despite the fact that they enjoy features for reasoning about and executing multi-agent specifications not found in any other framework.

Notable among these features are
\begin{itemize}
  \item an effective notion of equality of ensembles of agents (bisimulation) that comes with powerful proof techniques;
  \item a notion of spatial and behavioral types allowing for exceptionally powerful and fine-grained classification of the structure and behavior of ensembles of agents;
  \item a theorem relating the notion of equality to classification, affording a Liskov-style substitution principle saying when an agent or ensemble of agents may be substituted in for another such without change to the behavior of the overall ensemble.
\end{itemize}

In this paper we will argue that these features, together with computational reflection, constitute the smallest formalism necessary to develop a plausible theory of mind.

\section{Concurrent process calculi and spatial logics }\label{sec:concurrent_process_calculi_and_spatial_logics_} % (fold) 
In the last thirty years the process calculi have matured into a
remarkably powerful analytic tool for reasoning about concurrent and
distributed systems. Process-calculus-based algebraical specification of
processes began with Milner's Calculus for Communicating Systems (CCS)
\cite{DBLP:books/sp/Milner80} and Hoare's Communicating Sequential Processes
(CSP) \cite{DBLP:books/ph/Hoare85}, and continue
through the development of the so-called mobile process calculi,
e.g. Milner, Parrow and Walker's $\pi$-calculus \cite{DBLP:journals/iandc/MilnerPW92a}, \cite{DBLP:journals/iandc/MilnerPW92b},
Cardelli and Caires's spatial logic \cite{DBLP:conf/fossacs/Caires04} \cite{DBLP:journals/iandc/CairesC03} \cite{DBLP:journals/tcs/CairesC04}, or Meredith and Radestock's reflective calculi
\cite{DBLP:journals/entcs/MeredithR05} \cite{DBLP:conf/tgc/MeredithR05}. The process-calculus-based
algebraical specification of processes has expanded its scope of
applicability to include the specification, analysis, simulation and
execution of processes in domains such as:

\begin{itemize}
\item telecommunications, networking, security and application level protocols
\cite{DBLP:conf/popl/AbadiB02} 
\cite{DBLP:journals/tcs/AbadiB03} 
\cite{DBLP:conf/epew/BrownLM05} 
\cite{DBLP:conf/fossacs/LaneveZ05}; 
\item programming language semantics and design
\cite{DBLP:conf/epew/BrownLM05}
\cite{djoin}
\cite{DBLP:conf/afp/FournetFMS02}
\cite{DBLP:journals/toplas/SewellWU10};
\item webservices
\cite{DBLP:conf/epew/BrownLM05}
\cite{DBLP:conf/fossacs/LaneveZ05}
\cite{DBLP:conf/wise/Meredith03};
\item{blockchain}
  \cite{meredith_2017}
\item and biological systems
\cite{DBLP:conf/cmsb/Cardelli04}
\cite{DBLP:conf/esop/DanosL03}
\cite{DBLP:conf/psb/RegevSS01}
\cite{DBLP:journals/ipl/PriamiRSS01}.
\end{itemize}

Among the many reasons for the continued success of this approach are
two central points. First, the process algebras provide a
compositional approach to the specification, analysis and execution of
concurrent and distributed systems. Owing to Milner's original
insights into computation as interaction \cite{DBLP:journals/cacm/Milner93}, the process
calculi are so organized that the behavior ---the semantics--- of a
system may be composed from the behavior of its components. This means that specifications can be constructed in
terms of components ---without a global view of the system--- and
assembled into increasingly complete descriptions.

The second central point is that process algebras have a potent proof
principle, yielding a wide range of effective and novel proof
techniques \cite{DBLP:conf/concur/SangiorgiM92} \cite{DBLP:conf/fmco/Sangiorgi05} \cite{DBLP:journals/toplas/Sangiorgi09}. In particular, \emph{bisimulation} encapsulates an effective
notion of process equivalence that has been used in applications as
far-ranging as algorithmic games semantics
\cite{DBLP:conf/sas/Abramsky05} and the construction of
model-checkers \cite{caires_2004}. The essential notion can be stated in
an intuitively recursive formulation: a \emph{bisimulation} between two
processes $P$ and $Q$ is an equivalence relation $E$ relating $P$
and $Q$ such that: whatever action of $P$ can be observed, taking it
to a new state $P'$, can be observed of $Q$, taking it to a new state
$Q'$, such that $P'$ is related to $Q'$ by $E$ and vice versa. $P$ and
$Q$ are \emph{bisimilar} if there is some bisimulation relating
them. Part of what makes this notion so robust and widely applicable
is that it is parameterized in the actions observable of processes
$P$ and $Q$, thus providing a framework for a broad range of
equivalences and up-to techniques \cite{DBLP:conf/concur/SangiorgiM92} all governed by the same core
principle \cite{DBLP:journals/toplas/Sangiorgi09}.
% section concurrent_process_calculi_and_spatial_logics_ (end)


\vspace{1\baselineskip}
\section{Consciousness as the colonization of the brain}

In a recent conversation with \href{https://en.wikipedia.org/wiki/Joscha_Bach}{\uline{\textcolor[HTML]{1155CC}{Joscha Bach}}}, one his generation’s most original thinkers, he made the startling assertion that mobile concurrency is at odds with \href{https://en.wikipedia.org/wiki/Artificial_general_intelligence}{\uline{\textcolor[HTML]{1155CC}{artificial general intelligence}}}. In this context mobile concurrency means the sort of concurrency one finds when agents (aka computational processes) can discover each other, that is the communication topology (who knows whom and who is talking to whom) is evolving. This model is very different from a model where computational elements are soldered together like components on a motherboard. Mobile concurrency is more like the Internet or the telephony networks where people who just met for the first time learn each other’s websites, email addresses, and phone numbers. Joscha’s argument is that brains are only plastic, i.e. the connections between neurons are only changing during learning, but not during general computation.

\vspace{1\baselineskip}
My response to this proposition is that it assumes that the mind is not hosted in a logical model of computation that runs on the brain’s hardware. After all, the \href{https://en.wikipedia.org/wiki/Java_virtual_machine}{\uline{\textcolor[HTML]{1155CC}{Java virtual machine}}} (JVM) is a very different model of computation than the hardware it runs on. \href{https://en.wikipedia.org/wiki/Haskell}{\uline{\textcolor[HTML]{1155CC}{Haskell}}}’s model of computation is an even more dramatic variation on the idea of computation than the one embodied in the hardware the glorious Haskell compiler (\href{https://en.wikipedia.org/wiki/GHC}{\uline{\textcolor[HTML]{1155CC}{GHC}}}) is typically hosted on. Why wouldn’t the mind be organized like this? To use Joscha’s graphic metaphor, why wouldn’t the mind arise as a colonizing computational model that hosts itself on the brain’s hardware. If it is, well, rholang, an implementation of the rho-calculus, is hosted on Intel and AMD chips, for a start, and their computational models are very different from the one captured by the rho-calculus. 

\vspace{1\baselineskip}
In particular, the \href{https://www.sciencedirect.com/science/article/pii/S1571066105051893}{\uline{\textcolor[HTML]{1155CC}{rho-calculus}}} does provide direct support for mobile concurrent computation. The communication topology amongst a society of processes executing in the rho-calculus is dynamic. Who knows whom and can talk to whom in this society changes over the course of the computation. This way of thinking about the rho-calculus foreshadows my argument that there are very good reasons to suppose that a model like the rho-calculus may have hosted itself on and colonized the hardware of the human brain, in fact any brain that supports a theory of mind. 

\vspace{1\baselineskip}
\section{Code, data, and computation}

To make this argument i want to make some distinctions that not every computer scientist, let alone every developer makes. i make a distinction between code, data, and computation. Arbitrary code can be treated as some data which is an instance of some type of data structure in which the model of computation is expressed, or hosted. For example, you can host a \href{https://en.wikipedia.org/wiki/Turing_completeness}{\uline{\textcolor[HTML]{1155CC}{Turing-complete computational model}}}, like Haskell, in a term language expressed via a \href{https://en.wikipedia.org/wiki/Context-free_grammar}{\uline{\textcolor[HTML]{1155CC}{context-free grammar}}}, e.g. the grammar for well formed Haskell programs. Yet, we know that context-free grammars are not Turing-complete. How can this be? How can something provably less expressive than Turing-complete models represent Turing-complete computation?

\vspace{1\baselineskip}
It cuts to the heart of the distinction between syntax and semantics. The grammar of the term language expresses the \textit{syntax} of programs, not the \textit{dynamics of computation}, i.e, the semantics of code. Instead, the dynamics of computation arise by the interaction of rules (that operate on the syntax) with a particular piece of syntax, i.e. some code, representing the computation one wants to effect. In the lambda calculus (the model of computation on which Haskell is based) the workhorse of computation is a rule called beta reduction. This rule represents the operation of a function on data through the act of substituting the data for variables occurring in code. The data it operates on is a syntactic representation of the application of a function to data, but it is not the computation that corresponds to \textit{applying the function to the data}. That computation happens when beta reduction operates on the syntax, transforming it to a new piece of syntax. This distinction is how models that are less expressive than Turing-complete (e.g. context-free grammars) can host Turing-complete computation. 

\vspace{1\baselineskip}
Not to belabor the point, but the same distinction happens in Java and JVM. The dynamics of computation in the JVM happen through rules that operate on a combination of registers in the virtual machine together with a representation of the code. A Java programmer staring at a piece of Java code is not looking at the computation. Far from it. The syntax of a Java program is a window into a whole range of possibly different computations that come about depending on the state of the registers of the JVM at the time the code is run. The difference between these two forms of evaluation, beta reduction in the lambda calculus versus the transitions of the JVM is very important and we will return to it.

\vspace{1\baselineskip}
For now, though, one way to think about this distinction between code and computation is through an analogy with physics. Traditionally, the laws of physics are expressed through three things: a representation of physical states (think of this as the syntax of programs); laws of motion that say how states change over time (think of this as the rules that operate on syntax); and initial conditions (think of this as a particular piece of code you want to run). In this light, physics is seen as a special purpose programming language whose execution corresponds in a particular fashion to the way the physical world evolves, based on our observations of it. Physics is \textit{testable} because it lets us run a program and see if the evolution of some initial state to a state it reaches via the laws of motion matches our observations. In particular, when we see the physical world in a configuration that matches our initial state, does it go through a process of evolution that matches what our laws of motion say it should and does it land in a state that our laws of motion say it should. The fact that physics has this shape is why we can represent it effectively in code.

\vspace{1\baselineskip}
Once we see the distinction between code and computation, then the distinction between code and data is relatively intuitive, though somewhat subtle. Data in a computer program is also just syntax. In this sense it is no different than code, which is also just syntax. Every Lisp programmer understands this idea that somehow code is data and data is code. Even Java supports a kind of metaprogramming in which Java code can be manipulated as Java objects. The question is, is there any real dividing line between code and data? 

\vspace{1\baselineskip}
The answer is a definitive yes. Data is code that has very specific properties, for example the code always provably runs to termination. Not all code does this. In fact, Turing’s famous resolution of the \href{https://en.wikipedia.org/wiki/Entscheidungsproblem}{\uline{\textcolor[HTML]{1155CC}{Entscheidungsproblem}}} shows us that we cannot, in general, know when a program will halt for a language enjoying a certain quality of expressiveness, i.e Turing-completeness. But, there are less expressive languages, and Turing-complete languages enjoy suitable sublanguages or fragments that are less expressive than the whole language. Data resides in syntax that allows proving that the computation associated with a piece of syntax will halt. Likewise, data resides in syntax that allows proving that the computation will only enjoy finite branching. 

\vspace{1\baselineskip}
Programmers don’t think about data like this, they just know data when they see it. But in models of computation like the lambda calculus that doesn’t come equipped with built in data types, everything, even things like the counting numbers or the Boolean values true and false are represented as code. Picking out which code constitutes data and which constitutes general purpose programs has to do with being able to detect when code has the sorts of properties we discussed above. In general, there are type systems that can detect properties like this. Again, it’s a subtle issue, but fortunately we don’t need to understand all the subtleties, nor do we need to understand exactly where the dividing line between data and code is, just that there is one. 

\vspace{1\baselineskip}
In summary code and data are both just syntax that represents a state upon which a rule, or many rules, will operate. Data is expressed in a less expressive fragment of the syntax than code, giving it a definite or finitary character that code doesn’t always enjoy. Computation is the process of evolution arising when some rules interact with a representation of a state. Now, what does all this have to do with AI or the mind or even the rho-calculus?

\vspace{1\baselineskip}
\section{Reflection as a defining characteristic of intelligence}

The rho-calculus has a syntactic representation of the distinction between computation and code. It has an operation that expresses packaging up a computation as a piece of code so that it can be operated on, transforming it into new code. It also has an operation for turning a piece of code back into a computation. Whoah, you might say, that is some next level sh!t. But, as we mentioned Lisp programmers, and Java programmers have been doing this sort of metaprogramming for a long time. They have to. The reason has to do with scale. It is impossible for human teams to manage codebases involving millions and millions of lines of code without automated support. They use computer programs to write computer programs. They use computer programs to build computer program deployments. Metaprogramming is a necessity in today’s world.

\vspace{1\baselineskip}
But way back in the ‘80’s, still the early days of AI, a researcher named \href{https://en.wikipedia.org/wiki/Brian_Cantwell_Smith#:~:text=Brian\%20Cantwell\%20Smith\%20is\%20a,\%2C\%20and\%20philosophy\%2C\%20especially\%20ontology.}{\uline{\textcolor[HTML]{1155CC}{Brian Cantwell Smith}}} made an observation that resonated with me and many other people in AI and AI-adjacent fields. Smith’s argument is that introspection, the ability for the mind to look at the mind’s own process, is a key feature of intelligence. For some this is even the defining feature of intelligence. To make this idea of introspection, which he called computational reflection, concrete, Smith designed \href{https://www.ics.uci.edu/~jajones/INF102-S18/readings/17_Smith84.pdf}{\uline{\textcolor[HTML]{1155CC}{a language called 3-Lisp}}} that has the same operators that the rho-calculus has. Specifically, 3-Lisp has syntax to represent reifying a computation into code, and syntax for reflecting code back into running computation.

\vspace{1\baselineskip}
Now, there is a good reason to suspect that there is a connection between the problem of scale that today’s developers face and the problem of modeling our reflective, introspective capacity as reasoning beings. In particular, managing the complexity of representing our own reasoning becomes tractable in the presence of computational reflection. We can apply all of our algorithmic tricks to representations of our own reasoning to obtain better reasoning. This observation is amplified in the context of what evolutionary biologists and psychologists call \href{https://en.wikipedia.org/wiki/Theory_of_mind}{\uline{\textcolor[HTML]{1155CC}{theory of mind}}}. 

\vspace{1\baselineskip}
Specifically, introspection arises from the evolutionary advantage gained by being able to computationally model the behaviors of others, in particular members of your own species. If Alice develops the ability to model Barbara’s behavior, and Barbara is remarkably similar to Alice (as in same species, same tribe, even same extended family structure), then Alice is very close to being able to model Alice’s behavior. And when Alice needs to model Barbara’s behavior when Barbara is interacting with Alice, then Alice is directly involved in modeling Alice’s behavior. Taking this up to a scale where Alice can model her family unit, or the behavior of her tribe is where things get really interesting. More on that shortly, but for now we can see that something about computational reflection has to do with improving reasoning at scale in two senses of that word: (the complexity scale) improving reasoning by applying reasoning to itself; and (the social scale) improving reasoning about large numbers of reasoning agents.

\vspace{1\baselineskip}
In fact, Smith’s ideas about computational reflection and its role in intelligence and the design of programming languages were an inspiration for the design of the rho-calculus, which takes reification and reflection as primitive computational operators. However, where 3-Lisp and the rho-calculus part company is that 3-Lisp is decidedly sequential. It has no way to reasonably represent a society of autonomous computational processes running independently while interacting and coordinating. But in the context of a theory of mind this is just what a reasoner needs to do. They need an explicit model of their social context, which is made up of autonomous agents acting independently while also communicating and coordinating.

\vspace{1\baselineskip}
\section{The rho-calculus: from 3-Lisp to Society of Mind}

\vspace{1\baselineskip}
Around the same time Smith was developing his ideas of computational reflection \href{https://en.wikipedia.org/wiki/Marvin_Minsky}{\uline{\textcolor[HTML]{1155CC}{Marvin Minsky}}} was developing his famous \href{https://en.wikipedia.org/wiki/Society_of_Mind}{\uline{\textcolor[HTML]{1155CC}{Society of Mind}}} thesis. My take on Minsky’s proposal is that the mind is something like the US Congress, or any other deliberative body. It consists of a bunch of independent agents who are all vying for different resources (such as funding from the tax base). What we think of as a conscious decision is more like the result of a long deliberative process amongst a gaggle of independent, autonomous agents that often goes on well below conscious experience. But, the deliberative process results in a binding vote, and that binding vote is what is experienced as a conscious decision.

\subsection{The syntax and semantics of the notation system}\label{sub:the_syntax_and_semantics_of_the_notation_system} % (fold)

We now summarize a technical presentation of the calculus that
embodies the core of rholang. The typical presentation of such a
calculus follows the style of giving generators and relations on
them. The grammar, below, describing term constructors, freely
generates the set of processes, $\Proc$. This set is then quotiented
by a relation known as structural congruence and it is over this set
that the notion of dynamics is expressed. This presentation is
essentially that of \cite{DBLP:journals/entcs/MeredithR05} with the
addition of polyadicity and summation. For readability we have
relegated some of the technical subtleties to an appendix.

\paragraph{Notational interlude} when it is clear that some expression $t$ is a sequence (such as a list or a vector), and $a$ is an object that might be meaningfully and safely prefixed to that sequence then we write $a:t$ for the sequence with $a$ prefixed (aka ``consed'') to $t$. We write $t(i)$ for the $i$th element of $t$.

\subsubsection{Process grammar}\label{subsub:process_grammar}

\begin{mathpar}
\inferrule* [lab=process] {} {P, Q \bc \pzero \;\bm\; \mathsf{for}(
  \arrvec{y} \leftarrow x )P \;\bm\; x\mathsf{!}(\arrvec{Q}) \;\bm\;
  \mathsf{*}x \;\bm\; P\mathsf{|}Q } \and \inferrule* [lab=name] {}
            {x, y \bc \mathsf{@}P }
\end{mathpar}

Note that $\arrvec{x}$ (resp. $\arrvec{P}$) denotes a vector of names
(resp. processes) of length $|\arrvec{x}|$ (resp. $|\arrvec{P}|$). We adopt
the following useful abbreviations.

\begin{mathpar}
  \Pi \arrvec{P} := \Pi_{i=1}^{|\arrvec{P}|}P_i := P_1 | \ldots | P_{|\arrvec{P}|}
\end{mathpar}

\begin{definition}
  \emph{Free and bound names} The calculation of the free names of a
  process, $P$, denoted $\freenames{P}$ is given recursively by
  
  \begin{mathpar}
    \freenames{\pzero} = \emptyset
    \and
    \freenames{\mathsf{for}(\arrvec{y} \leftarrow x)(P)} = \{ x \} \cup \freenames{P}\setminus\{\arrvec{y}\}
    \and
    \freenames{x!(\arrvec{P})} = \{ x \} \cup \freenames{\arrvec{P}}
    \and
    \freenames{P|Q} = \freenames{P} \cup \freenames{Q}
    \and
    \freenames{\mathsf{}{x}} = \{ x \} \\
  \end{mathpar}
  
  where $\{\arrvec{x}\} := \{x_1, \ldots, x_{|\arrvec{x}|}\}$ and $\freenames{\arrvec{P}} := \bigcup \freenames{P_i}$.
  
  An occurrence of $x$ in a process $P$ is \textit{bound} if it is not
  free. The set of names occurring in a process (bound or free) is
  denoted by $\names{P}$.
\end{definition}

\subsection{Substitution}

We use $\Proc$ for the set of processes, $\QProc$ for the set of
names, and $\id{\{}\arrvec{y} / \arrvec{x} \id{\}}$ to denote partial
maps, $s : \QProc \rightarrow \QProc$. A map, $s$ lifts, uniquely, to
a map on process terms, $\widehat{s} : \Proc \rightarrow
\Proc$. Historically, it is convention to use $\sigma$ to range over
lifted subsitutions, $\widehat{s}$, to write the application of a
substitution, $\sigma$ to a process, $P$, with the substitution on the
right, $P\sigma$, and the application of a substitution, $s$, to a
name, $x$, using standard function application notation, $s(x)$. In
this instance we choose not to swim against the tides of
history. Thus, 

\begin{definition}
  given $x = \quotep{P'}$, $u = \quotep{Q'}$, $s =
  \substn{u}{x}$ we define the lifting of $s$ to $\widehat{s}$ (written
  below as $\sigma$) recursively by the following equations.
  \begin{mathpar}
    0 \sigma := 0 \\
    (P \mathsf{|} Q) \sigma
    :=    
    P\sigma \mathsf{|} Q\sigma \\
    (\mathsf{for}(\arrvec{y} \leftarrow v)P) \sigma    
    :=
    \mathsf{for}(\arrvec{z} \leftarrow \sigma(v))((P \psubstn{\arrvec{z}}{\arrvec{y}}) \sigma) \\
    (\lift{x}{Q}) \sigma  
    :=
    \lift{\sigma(x)}{ Q \sigma } \\
    (\dropn{y})  \sigma       
    := 
    \left\{ 
      \begin{array}{ccc} 
        Q' & & y \;\nameeq\; x \\
        \dropn{y} & & otherwise \\
      \end{array}
      \right.
  \end{mathpar} 

  where

  \begin{eqnarray}
    \psubstp{Q}{P}(x) = \id{\{} \quotep{Q} / \quotep{P} \id{\}}(x) = 
    \left\{ 
      \begin{array}{ccc}
        \quotep{Q} & & x \;\nameeq\; \quotep{P} \\
        x & & otherwise \\
      \end{array}
      \right. \nonumber
  \end{eqnarray}
\end{definition}

and $z$ is chosen distinct from $\quotep{P}$, $\quotep{Q}$, the free
names in $Q$, and all the names in $R$. Our $\alpha$-equivalence will
be built in the standard way from this substitution.

\begin{definition}
Then two processes, $P,Q$, are alpha-equivalent if $P = Q\{\arrvec{y}/\arrvec{x}\}$ for
some $\arrvec{x} \in \boundnames{Q},\arrvec{y} \in \boundnames{P}$, where $Q\{\arrvec{y}/\arrvec{x}\}$
denotes the capture-avoiding substitution of $\arrvec{y}$ for $\arrvec{x}$ in $Q$.
\end{definition}

\begin{definition}
  The {\em structural congruence} $\equiv$
  between processes \cite{DBLP:books/daglib/0004377} is the least congruence containing
  alpha-equivalence and satisfying the commutative monoid laws
  (associativity, commutativity and $\pzero$ as identity) for parallel
  composition $|$.
\end{definition}

\begin{definition}
  The {\em name equivalence} $\nameeq$ is the least congruence
  satisfying these equations
  \begin{mathpar}
  \inferrule*[lab=Quote-drop] {}{ \quotep{\dropn{x}} \;\nameeq\; x }
  \and
  \inferrule*[lab=Struct-equiv] { P \;\scong\; Q } { \quotep{P} \;\nameeq\; \quotep{Q} }
  \end{mathpar}
\end{definition}

The astute reader will have noticed that the mutual recursion of names
and processes imposes a mutual recursion on alpha-equivalence and
structural equivalence via name-equivalence. Fortunately, all of this
works out pleasantly and we may calculate in the natural way, free of
concern. The reader interested in the details is referred to the
appendix \ref{appendix:rho_details}.

\begin{remark}\label{rem:no_self_referential_names}
  One particularly useful consequence of these definitions is that
  $\forall P. \quotep{P} \not\in \freenames{P}$. It gives us a
  succinct way to construct a name that is distinct from all the names
  in $P$ and hence fresh in the context of $P$. For those readers
  familiar with the work of Pitts and Gabbay, this consequence allows
  the system to completely obviate the need for a fresh operator, and
  likewise provides a canonical approach to the semantics of
  freshness.
\end{remark}

Equipped with the structural features of the term language we can
present the dynamics of the calculus.

\subsection{Operational semantics}

Finally, we introduce the computational dynamics. What marks these
algebras as distinct from other more traditionally studied algebraic
structures, e.g. vector spaces or polynomial rings, is the manner in
which dynamics is captured. In traditional structures, dynamics is typically
expressed through morphisms between such structures, as in linear maps
between vector spaces or morphisms between rings. In algebras
associated with the semantics of computation, the dynamics is
expressed as part of the algebraic structure itself, through a
reduction reduction relation typically denoted by $\red$. Below, we
give a recursive presentation of this relation for the calculus used
in the encoding.

\begin{mathpar}
  \inferrule* [lab=COMM] {x_{t} \;\nameeq\; x_{s}, \;\;\; |\arrvec{y}| = |\arrvec{Q}|} {\mathsf{for}( \arrvec{y} \leftarrow x_{t} )P \;\mathsf{|}\; x_{s}!(\arrvec{Q})
  \red P\substn{\arrvec{\quotep{Q}}}{\arrvec{y}}}
  \and
  \inferrule* [lab=PAR]{P \red P'}{P\mathsf{|}Q \red P'\mathsf{|}Q}
  \and
  \inferrule* [lab=EQUIV]{{P \;\scong\; P'} \andalso {P' \red Q'} \andalso {Q' \;\scong\; Q}}{P \red Q}
\end{mathpar}

We write $P\red$ if $\exists Q $ such that $ P \red Q$ and $P\not\red$, otherwise.

%% \begin{definition}
%%   $\mathsf{COMM}$-events: when $P \red P'$ we can record the information
%%   justifying this reduction step with an expression of the form
%%   $\mathsf{COMM}_{(P,P')}(x_t,x_s,\sigma)$, leaving off the subscript
%%   $_{(P,P')}$ when the context makes the source ($P$) and target
%%   ($P'$) states clear. Likewise, if $c = \mathsf{COMM}_{(P,P')}(x_t,x_s,\sigma)$, we write $\mathsf{src}(c) = P$, $\mathsf{trgt}(c) = P'$, $\mathsf{src_{N}}(c) = x_{s}$, and $\mathsf{trgt_{N}}(c) = x_{t}$ and $P \stackrel{c}{\red} P'$.
%% \end{definition}

%% We use $\alpha, \beta, ...$ to range over reduction \emph{paths}. That is, if

%% \begin{mathpar}
%%   P_{1} \stackrel{\mathsf{COMM}(x_{t_{1}},x_{s_{1}},\sigma_{1})}{\red} P_{2} \stackrel{\mathsf{COMM}(x_{t_{2}},x_{s_{2}},\sigma_{2})}{\red} \cdots \stackrel{\mathsf{COMM}(x_{t_{n-1}},x_{s_{n-1}},\sigma_{n-1})}{\red} P_{n},
%% \end{mathpar}

%% then we may write

%% \begin{mathpar}
%%   \alpha = \mathsf{COMM}(x_{t_{1}},x_{s_{1}},\sigma_{1}):\mathsf{COMM}(x_{t_{2}},x_{s_{2}},\sigma_{2}):\ldots : \mathsf{COMM}(x_{t_{n-1}},x_{s_{n-1}},\sigma_{n-1}).
%% \end{mathpar}

%% \begin{definition}
%%   The set of paths between $P$ and $Q$, written $\mathsf{paths}(P,Q)$ is defined by $\mathsf{paths}(P,Q) := \{ c:\alpha \; : P \stackrel{c}{\red} P', \alpha \in \mathsf{paths}(P',Q) \}$ and $\epsilon$ denotes the empty path.
%% \end{definition}

\subsection{ Dynamic quote: an example }

Anticipating something of what's to come, let $z = \quotep{P}$, $u = \quotep{Q}$, and $x = \quotep{\lift{y}{\dropn{z}}}$. Now consider applying the substitution,
$\widehat{\id{\{}u / z \id{\}}}$, to the following pair of processes,
$\lift{w}{y!(\dropn{z})}$ and $\lift{w}{\dropn{x}} = \lift{w}{\dropn{\quotep{\lift{y}{\dropn{z}}}}}$.

\begin{eqnarray}
	\lift{w}{\lift{y}{\dropn{z}}}\widehat{\id{\{}u / z \id{\}}}
		& = &
		\lift{w}{\lift{y}{Q}} \nonumber\\
	\lift{w}{\dropn{x}} \widehat{ \id{\{}u / z \id{\}} }
		& = &
		\lift{w}{\dropn{x}} \nonumber
\end{eqnarray}

The body of the quoted process, $\quotep{\lift{y}{\dropn{z}}}$, is
impervious to substitution, thus we get radically different
answers. In fact, by examining the first process in an input context,
e.g. $\mathsf{for}(z \leftarrow x)\lift{w}{\lift{y}{\dropn{z}}}$, we see that the process
under the output operator may be shaped by prefixed inputs binding a
name inside it. In this sense, the combination of input prefix binding
and output operators will be seen as a way to dynamically construct
processes before reifying them as names.

\section{Replication}

As mentioned before, it is known that replication (and hence
recursion) can be implemented in a higher-order process algebra
\cite{DBLP:books/daglib/0004377}. As our first example of calculation with the
machinery thus far presented we give the construction explicitly in
the {\rhoc}.

\begin{eqnarray}
	D_{x} & := & \prefix{x}{y}{(\binpar{\outputp{x}{y}}{\dropn{y}})} \nonumber\\
	\bangp_{x}{P} & := & \binpar{\lift{x}{\binpar{D_{x}}{P}}}{D_{x}} \nonumber
\end{eqnarray}

\begin{eqnarray}
	\bangp_{x}{P} & & \nonumber\\
	=
	& \lift{x}{(\prefix{x}{y}{(\outputp{x}{y} | \dropn{y})) | P}} 
	      | \prefix{x}{y}{(\outputp{x}{y} | \dropn{y})} & \nonumber\\
	\red
	& (\outputp{x}{y} | \dropn{y})\substn{\quotep{(\prefix{x}{y}{(\dropn{y} | \outputp{x}{y})) | P}}}{y} & \nonumber\\
	=
	& \outputp{x}{\quotep{(\prefix{x}{y}{(\outputp{x}{y} | \dropn{y})) | P}}}
	  | {(\prefix{x}{y}{(\outputp{x}{y} | \dropn{y})) | P}} & \nonumber\\
	\red
	& \ldots & \nonumber\\
	\red^*
	& P | P | \ldots & \nonumber
\end{eqnarray}

Of course, this encoding, as an implementation, runs away, unfolding
$\bangp{P}$ eagerly. A lazier and more implementable replication
operator, restricted to input-guarded processes, may be obtained as follows.

\begin{eqnarray}
\bangp{\prefix{u}{v}{P}} 
	:= 
	\binpar{\lift{x}{\prefix{u}{v}{(\binpar{D(x)}{P})}}}{D(x)} \nonumber
\end{eqnarray}

\begin{remark}
  Note that the lazier definition still does not deal with summation
  or mixed summation (i.e. sums over input and output). The reader is
  invited to construct definitions of replication that deal with these
  features. 

  Further, the definitions are parameterized in a name, $x$. Can you,
  gentle reader, make a definition that eliminates this parameter and
  guarantees no accidental interaction between the replication
  machinery and the process being replicated -- i.e. no accidental
  sharing of names used by the process to get its work done and the
  name(s) used by the replication to effect copying. This latter
  revision of the definition of replication is crucial to obtaining
  the expected identity $!!P \sim !P$.
\end{remark}

\begin{remark}\label{rem:paradoxical_combinator}
  The reader familiar with the lambda calculus will have noticed the
  similarity between $D$ and the paradoxical combinator.
\end{remark}

\subsubsection{Bisimulation}

The computational dynamics gives rise to another kind of equivalence,
the equivalence of computational behavior. As previously mentioned
this is typically captured \emph{via} some form of bisimulation.

% The notion we use in this paper is weak barbed bisimulation
% \cite{milner91polyadicpi}.

The notion we use in this paper is derived from weak barbed
bisimulation \cite{milner91polyadicpi}. 

\begin{definition}
An \emph{observation relation}, $\downarrow_{\mathcal N}$, over a set
of names, $\mathcal N$, is the smallest relation satisfying the rules
below.

\infrule[Out-barb]{y \in {\mathcal N}, \; x \nameeq y}
		  {\outputp{x}{v} \downarrow_{\mathcal N} x}
\infrule[Par-barb]{\mbox{$P\downarrow_{\mathcal N} x$ or $Q\downarrow_{\mathcal N} x$}}
		  {\binpar{P}{Q} \downarrow_{\mathcal N} x}

We write $P \Downarrow_{\mathcal N} x$ if there is $Q$ such that 
$P \wred Q$ and $Q \downarrow_{\mathcal N} x$.
\end{definition}

\begin{definition}
%\label{def.bbisim}
An  ${\mathcal N}$-\emph{barbed bisimulation} over a set of names, ${\mathcal N}$, is a symmetric binary relation 
${\mathcal S}_{\mathcal N}$ between agents such that $P\rel{S}_{\mathcal N}Q$ implies:
\begin{enumerate}
\item If $P \red P'$ then $Q \wred Q'$ and $P'\rel{S}_{\mathcal N} Q'$.
\item If $P\downarrow_{\mathcal N} x$, then $Q\Downarrow_{\mathcal N} x$.
\end{enumerate}
$P$ is ${\mathcal N}$-barbed bisimilar to $Q$, written
$P \wbbisim_{\mathcal N} Q$, if $P \rel{S}_{\mathcal N} Q$ for some ${\mathcal N}$-barbed bisimulation ${\mathcal S}_{\mathcal N}$.
\end{definition}

\subsubsection{Contexts}

One of the principle advantages of computational calculi from the
$\lambda$-calculus to the $\pi$-calculus is a well-defined notion of context,
contextual-equivalence and a correlation between
contextual-equivalence and notions of bisimulation. The notion of
context allows the decomposition of a process into (sub-)process and
its syntactic environment, its context. Thus, a context may be
thought of as a process with a ``hole'' (written $\Box$) in it. The
application of a context $K$ to a process $P$, written $K[P]$, is
tantamount to filling the hole in $K$ with $P$. In this paper we do
not need the full weight of this theory, but do make use of the notion
of context in the proof the main theorem. 

\begin{mathpar}
\inferrule* [lab=context] {} {K \bc \Box \;\bm\; \mathsf{for}( \arrvec{y} \leftarrow x )K \;\bm\; x\mathsf{!}(\arrvec{P},K,\arrvec{Q}) \;\bm\; K\mathsf{|}P }
\end{mathpar}

\begin{definition}[contextual application] Given a context $K$, and
  process $P$, we define the \emph{contextual application}, $K[P] :=
  K\{P/\Box\}$. That is, the contextual application of K to P is the
  substitution of $P$ for $\Box$ in $K$.
\end{definition}

\begin{remark}
  Note that we can extend the definition of free and bound names to contexts.
\end{remark}

\subsection{Additional notation}

We achieve some notational compression with the following convention

\begin{mathpar}
  \mathsf{for(}{y_{1}}\leftarrow{x_{1}}\mathsf{;}\;\ldots\mathsf{;}\;{y_{n}}\leftarrow{x_{n}}\mathsf{)}P := \prefix{x_{1}}{y_{1}}{\prefix{x_{2}}{y_{2}}{\ldots \prefix{x_{n}}{y_{n}}{P}}}
\end{mathpar}

Even though we already have the notation $x = \quotep{P}$, allowing us
to pick out the process a name quotes, it will be convenient to
introduce an alternate notation, $\procn{x}$, when we want to
emphasize the connection to the use of the name. Note that, by virtue
of name equivalence, $\quotep{\procn{x}} \;\nameeq\; x$; so, the notation
is consistent with previous definitions.

In \cite{DBLP:journals/entcs/MeredithR05} an interpretation of the new operator is
given. It turns out that there are several possible interpretations
all enjoying the requisite algebraic properties of the operator (see
\cite{milner91polyadicpi}). We will therefore make liberal use of
$(\mathsf{new}\; \arrvec{x})P$.

\subsection{Extensions to the calculus}
\subsubsection{Values}
While it is standard in calculi such as the $\lambda$-calculus to
define a variety of common values such as the natural numbers and
booleans in terms of Church-numeral style encodings, it is equally
common to simply embed values directly into the calculus. Not being
higher-order, this presents some challenges for the $\pi$-calculus,
but for the rho-calculus, everything works out very nicely if we treat
values, e.g. the naturals, the booleans, the reals, etc as processes. This
choice means we can meaninfully write expressions like
$x!(\mathsf{5})$ or $u!(\mathsf{true})$, and in the context
$\prefix{x}{y}P[*y] \mathsf{|} x!(\mathsf{5})$ the value $\mathsf{5}$
will be substituted into $P$. Indeed, since operations like addition,
multiplication, etc.  can also be defined in terms of processes, it is
meaningful to write expressions like
$\mathsf{5}\mathsf{|}\mathsf{+}\mathsf{|}\mathsf{1}$, and be confident
that this expression will reduce to a process representing
$\mathsf{6}$. Thus, we can also use standard mathematical expressions,
such as $\mathsf{5+1}$, as processes, and know that these will
evaluate to their expected values. Further, when combined with
$\mathsf{for}$-comprehensions, we can write algebraic expressions,
such as $\prefix{x}{y}\mathsf{5}\mathsf{+}\mathsf{*}y$, and in
contexts like
$(\prefix{x}{y}\mathsf{5}\mathsf{+}\mathsf{*}y)\mathsf{|}\outputp{x}{\mathsf{1}}$
this will evaluate as expected, producing the process (aka value)
$\mathsf{6}$.

With these conventions in place it is useful to reduce the
proliferation of $\mathsf{*}$'s, by adopting a pattern-matching
convention. Thus, we write $\prefix{x}{@v}{P}$ to denote binding $v$
to the value passed and not the \emph{name} of the value. Hence, we
may write $\prefix{x}{@v}{\mathsf{5+v}}$ without any loss of clarity,
confident that this translates unambiguously into the formal calculus
presented above. We achieve even greater compression and a more
familiar notation if we also adopt the notation

\begin{mathpar}
  \mathsf{let}\; x \;\mathsf{=}\; v \;\mathsf{in}\; P := (\mathsf{new} \; u)(\prefix{u}{@x}P)\mathsf{|}\outputp{u}{v}
\end{mathpar}

and generalized to nested expressions via

\begin{mathpar}
  \mathsf{let}\; x_{1} \;\mathsf{=}\; v_{1} \mathsf{;}\; \ldots \mathsf{;}\; x_{n} \;\mathsf{=} \; v_{n}  \;\mathsf{in}\; P := (\mathsf{new} \; \arrvec{u})\mathsf{for(}{x_{1}}\leftarrow{u_{1}}\mathsf{;}\;\ldots\mathsf{;}\;{x_{n}}\leftarrow{u_{n}}\mathsf{)}P \mathsf{|}\Pi\outputp{u_{i}}{v_{i}}
\end{mathpar}

\subsubsection{Annihilation}
Another important variation has to do with the rewrite rules. There is
a recursive version of the $\mathsf{COMM}$-rule that aligns with
intuitions about the recursive nature of behavior in compositionally
defined agents. The maths make it clearer than the English. First, we
define what it means for two processes to annihilate each other.

\begin{definition}
  Annihilation: Processes $P$ and $Q$ are said to annihilate one another, written $P \bot Q$, just when $\forall R. P \mathsf{|} Q \rightarrow^{*} R \Rightarrow R \rightarrow^{*} \pzero$.
\end{definition}

Thus, when $P \bot Q$, all rewrites out of $P \mathsf{|} Q$ eventually
lead to $\pzero$. Evidently, $P \bot Q \iff Q \bot P$, and $\pzero
\bot \pzero$. Naturally, we can extend annihilation to names: $x \bot
y \iff \procn{x} \bot \procn{y}$.

Annihilation affords a new version of the $\mathsf{COMM}$-rule:

\begin{mathpar}
  \inferrule* [lab=COMM] {x_{t} \;\bot x_{s}, \;\;\; |\arrvec{y}| = |\arrvec{Q}|} {(R_1 + \mathsf{for}( \arrvec{y} \leftarrow x_{t} )P) \;\mathsf{|}\; (x_{s}!(\arrvec{Q}).P' + R_2)
  \red P\substn{\arrvec{\quotep{Q}}}{\arrvec{y}}\mathsf{|}P'}
\end{mathpar}

All annihilation-based reduction happens in terms of reductions that
happen at a lower degree of quotation, and grounds out in the fact
that $\pzero \bot \pzero$ and thus $\quotep{\pzero} \bot \quotep{\pzero}$.

% subsection the_syntax_and_semantics_of_the_notation_system (end)   


\vspace{1\baselineskip}
How can this view, which puts most of the computation outside of conscious reasoning, be reconciled with a view of the mind as essentially, indeed definitionally reflective? The rho-calculus was designed with an answer to this question in mind. The rho-calculus says that computational agents come in just six shapes:

\vspace{1\baselineskip}
\begin{itemize}
	\item 0 - the stopped or null agent that does nothing;

	\item for( y <- x )P - the agent that is listening on the channel x waiting for data that it will bind to the variable y before becoming the agent P;

	\item x!( Q ) - the agent that is sending a piece of code/data on the channel x;

	\item P$\vert$Q - the agent that is really the parallel composition of two agents, P and Q, running concurrently, autonomously;

	\item $\ast$x - the agent that is reflecting the code referred to by x back into a running computation

\end{itemize}
\vspace{1\baselineskip}
Notice how three of these constructs use the symbol x. Two of them use x as though it were a channel for communication between agents, and one of them uses x as though it were a reference to a piece of code. The one magic trick that the rho-calculus has up its sleeve is that channels are references to a piece of code. It takes a bit of getting used to, but it comes with time.

\vspace{1\baselineskip}
Armed with just this much information about the rho-calculus we can return to our narrative about Alice and find parsimonious representations of all of the challenges facing her developing social and introspective intelligence. As outside observers of Alice’s social context we can write down its behavior as a parallel composition of the behavior of each individual. In symbols that’s P1 $\vert$ P2 $\vert$ $\ldots$ $\vert$ Pn where Pi is the model of the ith individual in Alice’s social context. Now, a model of Alice’s behavior needs a representation of that parallel composition for her own behavior to represent reasoning about it. In symbols that’s @( P1 $\vert$ P2 $\vert$ $\ldots$ $\vert$ Pn ). For Alice to have this data located somewhere she has access to it she puts the model on a channel x!( P1 $\vert$ P2 $\vert$ $\ldots$ $\vert$ Pn ), and when she needs to retrieve it she executes

\vspace{1\baselineskip}
for( y <- x )AliceThinkingAboutHerColleagues( y ) $\vert$ x!( P1 $\vert$ P2 $\vert$ $\ldots$ $\vert$ Pn )

\vspace{1\baselineskip}
The workhorse rule of computation in the rho-calculus, which is very similar in spirit to the lambda calculus’ beta reduction, is that an expression like this evolves to

\vspace{1\baselineskip}
AliceThinkingAboutHerColleagues( @( P1 $\vert$ P2 $\vert$ $\ldots$ $\vert$ Pn ) )

\vspace{1\baselineskip}
So, now Alice’s thoughts about her colleagues have an explicit representation of their behavior available to Alice. With it, she can simulate her colleagues behavior by simulating the behavior of P1 $\vert$ P2 $\vert$ ... $\vert$ Pn through operations on @( P1 $\vert$ P2 $\vert$ ... $\vert$ Pn ). We can model Alice observing her colleague’s actual behavior with an expression like Alice $\vert$ P1 $\vert$ P2 $\vert$ ... $\vert$ Pn. Alice can compare her simulation with her observations. In fact, whatever we can model is also available to Alice both to run as well as to reify into data and compare the code and her simulations of it to what she observes of the actual behavior of her social context. This includes Alice’s own behavior. 

\vspace{1\baselineskip}
This may have gone by a little fast, but think about it. This is the smallest set of operations needed for Alice to simultaneously model her social context and herself in it. In particular, threads are ‘consciously available’ to Alice just when her own behavior reifies those threads into data and her processing interacts with that data. This argument is part of what went into the design deliberations for the rho-calculus. It is the smallest model of computation that reconciles Smith’s arguments for computational reflection with Minsky’s arguments for a Society of Mind that fits with evolutionary biology’s account of organisms with a theory of mind. Anything smaller misses a key component of the situation.

\vspace{1\baselineskip}
\section{We have seen the enemy and we are they}

This argument is why it is plausible for a model of computation like the rho-calculus to find purchase on the hardware in Alice’s brain. She needs all the elements of this model to compete with other members of her species who are likewise racing to model the behavior of their social context. This is why, very much to the contrary of Joscha’s position, i would argue that mobile concurrency is at the heart of artificial general intelligence.


\end{document}
