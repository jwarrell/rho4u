\section{A presentation of the semantics of MeTTa}

Accordingly, we present the semantics of MeTTa in terms of a monad describing the algebra of states, a structural equivalence quotienting the algebra of states, and some rewrite rules describing state transitions. 

\subsection{Rationale for such a presentation}

The rationale for such a presentation is not simply that this is the way it’s done. Instead, the benefits include

\begin{itemize}
  \item an effective (if undecidable) notion of program equality;
  \item an independent specification allowing independent implementations;
  \item meta-level computation, including type checking, model checking, macros, computational reflection, etc.
\end{itemize}

\subsubsection{Effective program equality}
Of course, the notion we are calling program equality is called bisimulation in the literature. One of the key benefits of having a notion of bisimulation explicitly spelled out is that it makes possible both by-hand and automated proofs of correctness of implementations of MeTTa. We illustrate this in a later section of the paper where we specify a compilation from MeTTa to the rho-calculus and rholang and provide both a clear statement of what it means for the compiler to be correct and a proof that it is so.

%% One of the motivations for providing the example to is foster within the MeTTa developer community a development methodology known as correct-by-construction. Some of the benefits of correct-by-construction include avoiding bytecode injection attacks, avoiding concurrency issues, and in general avoiding technical debt.

%% With a clearly spelled out operational semantics the correct-by-construction software development methodology becomes a real practice and not just an ideal to be striven for -- except when we're under deadling, and we're always under deadline!

\subsubsection{Independent implementations}
Hand in hand with being able to prove an implementation correct is the ability to support multiple independent implementations, each of which is provably in compliance with the specification. Pioneered by efforts like the JVM, this approach has been remarkably effective in modern projects, like Ethereum, where the Yellow Paper made it possible for many independent teams to implement compilant Ethereum clients. \cite{wood2014ethereum} This network of clients is regularly responsible for the deployment and correct execution of millions of \$ of transactions each month.

Perhaps more salient to the MeTTa developer community, it means that no one project needs to do all the development of MeTTa clients. This spreads not only the cost of development around, but the risk. In a word, it makes MeTTa more robust against the failure of any one project or team. Correct-by-construction methodology and the tools of operational semantics dramatically enhance the already proven power of open source development to decentralize cost and risk.

\subsection{Meta-level computation}
Presumably efforts by human developers to develop provably correct compilation schemes from MeTTa to other computational models are just the first generation of intelligences that will seek to do so. Over time the hope is that many different kinds of intelligences will model, and hence make amenable to adaptation, MeTTa's model of computation. In particular, AGI's will seek to do the same and at dramatically different scales and timeframes. 

\subsection{MeTTa Operational Semantics}
The complexity of MeTTa's operational semantics is somewhere between the simplicity of the $\lambda$-calculus and the enormity of the JVM. Note that MeTTa is not WYSIWYG, however, it is not such a stretch to make a version of MeTTa that is.

\subsubsection{Algebra of States}

%% Non-terminals are enclosed between $<$ and $>$. 
%% The symbols -$>$ (production),  \textbf{$|$}  (union) 
%% and \textbf{eps} (empty rule) belong to the BNF notation. 
%% All other symbols are terminals.

\paragraph{Terms}
\begin{mathpar}
  \inferrule*[lab=AtomSpace]{}{Term \bc \mathsf{(} Term \; [Term] \mathsf{)} \;\bm\; \mathsf{\{} Term \; [Term] \mathsf{\}} \;\bm\; Atom \;\bm\; () \;\bm\; \{\}}
\end{mathpar}

We impose the equation $\mathsf{\{} \ldots, t, u, \ldots \mathsf{\}} = \mathsf{\{} \ldots, u, t, \ldots \mathsf{\}}$, making terms of this form multisets. Note that for multiset comprehensions this amounts to non-determinism in the order of the terms delivered, but they are still streams. We use $\mathsf{\{}Term\mathsf{\}}$ to denote the set of terms that are (extensionally or intensionally) defined multisets, and $\mathsf{(}Term\mathsf{)}$ to denote the set of terms that are (extensionally or intensionally) defined lists. Informally, we will refer to elements in $\mathsf{\{}Term\mathsf{\}}$ as spaces.

We assume a number of polymorphic operators, such as $\pplus$ which acts as union on multisets and append on lists and concatenation on strings, and $::$ which acts as cons on lists and the appropriate generalization for the other data types.

\subsubsection{Extensionally vs intensionally defined spaces}
The complete operational semantics makes a distinction between extensionally defined spaces and terms, where each element of the space or term has been explicitly constructed, versus intensionally defined spaces and terms where elements are defined by a rule. The latter we call comprehensions.

This design is adopted for numerous reasons:
\begin{itemize}
  \item it provides an explicit representation for bindings;
  \item it provides an explicit representation for infinite terms and spaces;
  \item it provides an explicit scope for access to remotely accessed data.
\end{itemize}

The reader is encouraged to read the full semantics for more details. In this summary we elide the presentation of intensionally defined spaces and terms.

\paragraph{Term sequences}
\begin{mathpar}
  \inferrule*[lab=Sequence]{}{ [Term] \bc \epsilon \;\bm \; Term \;\bm \; Term \; [Term] }
\end{mathpar}

\paragraph{Literals and builtins}
\begin{mathpar}
  \inferrule*[lab=Atom]{}{ Atom \bc Ground \;\bm \; Builtin \;\bm \; Var }
  \and
  \inferrule*[lab=Boolean]{}{ BoolLiteral \bc \mathsf{true} \;\bm \; \mathsf{false}}
  \and
  \inferrule*[lab=Ground]{}{ Ground \bc BoolLiteral \;\bm \; LongLiteral \;\bm \; StringLiteral \;\bm \; UriLiteral} \\
  \and
  \inferrule*[lab=Builtin]{}{ Builtin \bc \;\bm \; \mathsf{=} \;\bm \; \mathsf{transform} \;\bm \; \mathsf{addAtom} \;\bm \; \mathsf{remAtom}}
  \and
  \inferrule*[lab=Var]{}{TermVar \bc \mathsf{\_} \;\bm \; Var}
\end{mathpar}

\paragraph{States}
\begin{eqnarray*}
  State & \bc & \langle \mathsf{\{}Term\mathsf{\}} \mathsf{,} \mathsf{\{}Term\mathsf{\}} \mathsf{,} \mathsf{\{}Term\mathsf{\}} \mathsf{,} \mathsf{\{}Term\mathsf{\}} \rangle
\end{eqnarray*}

We will use $S, T, U$ to range over states and $\mathsf{i} := \pi_{1}$, $\mathsf{k} := \pi_{2}$, $\mathsf{w} := \pi_{3}$, and $\mathsf{o} := \pi_{4}$ for the first, second, third, and fourth projections as accessors for the components of states. Substitutions are ranged over by $\sigma$, and as is standard, substitution application will be written postfix, e.g. $t\sigma$.

A state should be thought of as consisting of $4$ \emph{registers}:
$\mathsf{i}$ is the input register where queries are issued;
$\mathsf{k}$ is the knowledge base; $\mathsf{w}$ is a workspace;
$\mathsf{o}$ is the output register. We separate the input, workspace,
and output registers to allow for coarse-graining of bisimulation. An
external agent cannot necessarily observe the transitions related to
the workspace.

\subsubsection{Rewrite Rules}

\begin{mathpar}
  \inferrule* [lab=Query]{\sigma_{i} = \mathsf{unify}(t',t_{i}), k = \mathsf{\{} (\mathsf{=}\; t_{1} \; u_{1}), \ldots, (\mathsf{=}\; t_{n} \; u_{n}) \mathsf{\}} \pplus k', \mathsf{insensitive}(t',k')}{\langle \mathsf{\{} t' \mathsf{\}} \pplus i, k, w, o \rangle \to \langle i, k, \mathsf{\{} u_{1}\sigma_{1} \mathsf{\}} \pplus\; \ldots\; \pplus \mathsf{\{} u_{n}\sigma_{n} \mathsf{\}} \pplus w, o \rangle} \\
  \and
  \inferrule* [lab=Chain]{\sigma_{i} = \mathsf{unify}(u,t_{i}), k = \mathsf{\{} (\mathsf{=}\; t_{1} \; u_{1}), \ldots, (\mathsf{=}\; t_{n} \; u_{n}) \mathsf{\}} \pplus k', \mathsf{insensitive}(u,k')}{\langle i, k, \mathsf{\{} u \mathsf{\}} \pplus w, o \rangle \to \langle i, k, \mathsf{\{} u_{1}\sigma_{1} \mathsf{\}} \pplus\; \ldots\; \pplus \mathsf{\{} u_{n}\sigma_{n} \mathsf{\}} \pplus w, o \rangle} \\
  \and
  \inferrule* [lab=Transform]{\sigma_{i} = \mathsf{unify}(t,t_{i}), k = \mathsf{\{} t_{1}, \ldots, t_{n} \mathsf{\}} \pplus k',\mathsf{insensitive}(t,k')}{\langle \mathsf{\{} \mathsf{(}\mathsf{transform}\; t \; u\mathsf{)} \mathsf{\}} \pplus i, k, w, o \rangle \to \langle i, k, \mathsf{\{} u\sigma_{1} \mathsf{\}} \pplus\; \ldots\; \pplus \mathsf{\{} u\sigma_{n} \mathsf{\}} \pplus w, o \rangle} \\
  \and
  \inferrule* [lab=AddAtom1]{}{\langle \mathsf{\{} \mathsf{(} \mathsf{addAtom}\; t\mathsf{)}\mathsf{\}}  \pplus i, k, w, o \rangle \to \langle i, k \pplus \mathsf{\{} t\mathsf{\}}, w, \mathsf{\{} \mathsf{()}\mathsf{\}} \pplus o \rangle} \\
  \and
  \inferrule* [lab=AddAtom2]{\langle i_{1}, k_{1}, w_{1}, o_{1}\rangle \to \langle i_{2}, k_{2}, w_{2}, o_{2} \rangle, k_{3} = \mathsf{\{} \mathsf{(} \mathsf{addAtom}\; t\mathsf{)}\mathsf{\}} \pplus k_{1}}{\langle i_{1}, k_{3}, w_{1}, o_{1}\rangle \to \langle i_{2}, \mathsf{\{} \mathsf{(} \mathsf{addAtom}\; t\mathsf{)}, t\mathsf{\}} \pplus k_{2}, w_{2}, \mathsf{\{} \mathsf{()}\mathsf{\}} \pplus o_{2} \rangle} \\
  \inferrule* [lab=RemAtom1]{}{\langle \mathsf{\{} \mathsf{(} \mathsf{remAtom}\; t\mathsf{)}\mathsf{\}}  \pplus i, \mathsf{\{} t \mathsf{\}} \pplus k, w, o \rangle \to \langle i, k, w, \mathsf{\{} \mathsf{()}\mathsf{\}} \pplus o \rangle} \\
  \and
  \inferrule* [lab=RemAtom2]{\langle i_{1}, k_{1}, w_{1}, o_{1}\rangle \to \langle i_{2}, k_{2}, w_{2}, o_{2} \rangle, k_{3} = \mathsf{\{} \mathsf{(} \mathsf{remAtom}\; t\mathsf{)} \mathsf{\}} \pplus \mathsf{\{} t \mathsf{\}} \pplus k_{1}}{\langle i_{1}, k_{3}, w_{1}, o_{1}\rangle \to \langle i_{2}, \mathsf{\{} \mathsf{(} \mathsf{remAtom}\; t\mathsf{)}\mathsf{\}} \pplus k_{2}, w_{2}, \mathsf{\{} \mathsf{()}\mathsf{\}} \pplus o_{2} \rangle} \\
  \inferrule* [lab=Output]{\mathsf{insensitive}(u,k)}{\langle i, k, \mathsf{\{} u \mathsf{\}} \pplus w, o \rangle \to \langle i, k, w, \mathsf{\{} u \mathsf{\}} \pplus o \rangle}
\end{mathpar}

Where $\mathsf{insensitive}(t,k)$ means that $\mathsf{(} \mathsf{=}\; t'\; u \mathsf{)} \in k \Rightarrow \neg \mathsf{unify}(t,t')$.


