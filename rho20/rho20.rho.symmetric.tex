\section{Symmetric rho}

Before we leave the world of crisp computation, we present one more
variant that explores unification as the means of data
exchange. Terms, possibly with variables, in them are passed on
channels. Interaction only takes place when the terms unify. This
choice means that there isn't a distinction between sender and
receiver. Instead, both agents may have continuations. In symbols, we have

\begin{mathpar}
  \inferrule* [lab=process] {} {P, Q \bc \pzero \;\bm\; \mathsf{for}(
    t \leftrightarrow x )P \;\bm\; \mathsf{*}x \;\bm\; P\mathsf{|}Q }
  \and
  \inferrule* [lab=name] {} {x, y \bc \mathsf{@}P } \\
  \and
  \inferrule* [lab=term] {} {t, u \bc atom \;\bm\; \mathsf{(}t^{*}\mathsf{)}} \\
  \and
  \inferrule* [lab=term] {} {atom \bc x \;\bm\; \mathsf{Bool} \;\bm\; \mathsf{String} \;\bm\; \mathsf{Int} \;\bm\; P}
\end{mathpar}

\begin{mathpar}
  \inferrule* [lab=equiv] {} {P\mathsf{|}\pzero \equiv P \;\; P\mathsf{|}Q \equiv Q\mathsf{|}P \;\; P\mathsf{|}(Q\mathsf{|}R) \equiv (P\mathsf{|}Q)\mathsf{R}} \\
  \and
  \inferrule* [lab=alpha] { \mathsf{occurs}(t,y)} {\mathsf{for}(t \leftrightarrow x )P \equiv \mathsf{for}(t\{z/y\} \leftrightarrow x )(P\{z/y\}) \; \mathsf{if} z \notin \mathsf{FN}(P)}
\end{mathpar}

\begin{mathpar}
  \inferrule* [lab=COMM] {\sigma = \mathsf{unify}(t,u)} {\mathsf{for}( t \leftrightarrow x )P \;\mathsf{|}\; \mathsf{for}( u \leftrightarrow x )Q
    \red P\dot{\sigma}\mathsf{|}Q\dot{\sigma}} \\
  \and
  \inferrule* [lab=PAR]{P \red P'}{P\mathsf{|}Q \red P'\mathsf{|}Q}
  \and
  \inferrule* [lab=EQUIV]{{P \;\scong\; P'} \andalso {P' \red Q'} \andalso {Q' \;\scong\; Q}}{P \red Q}
\end{mathpar}

where $\dot{\sigma}$ denotes the substitution that replaces all variable to process bindings with variable to name bindings. Thus, $\dot{\{P / x\}} = \{\mathsf{@}P / x\}$.

Note that the $\mathsf{REFL}$ rule for backchaining can also be added
to this system, providing the semantics for a concurrent form of the
language $\mathsf{MeTTa}$.
