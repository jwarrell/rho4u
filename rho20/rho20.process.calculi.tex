\section{Concurrent process calculi and spatial logics }\label{sec:concurrent_process_calculi_and_spatial_logics_} % (fold) 
In the last thirty years the \emph{process calculi} have matured into
a remarkably powerful analytic tool for reasoning about concurrent and
distributed systems, such as Internet communication, security, or
application protocols (\cite{DBLP:conf/popl/AbadiB02},
\cite{DBLP:conf/epew/BrownLM05}, \cite{DBLP:conf/fossacs/LaneveZ05}),
but also chemical reactions, and biological protocols such as cell
signaling regimes (\cite{DBLP:conf/psb/RegevSS01},
\cite{DBLP:journals/ipl/PriamiRSS01}.). Process-calculus-based
algebraic specification of computational processes began with Milner's
Calculus for Communicating Systems (CCS) \cite{DBLP:books/sp/Milner80}
and Hoare's Communicating Sequential Processes (CSP)
\cite{DBLP:books/ph/Hoare85}. Being largely influenced by the hardware
models of the time, these formalisms were based on a fixed
communication topology, where computational processes are considered
to be something like components soldered to a motherboard.

But as radio and cellular technologies, and protocols like
$\mathsf{TCP/IP}$ became popularized, this influenced the evolution of
these formalisms. In particular, the trend toward physical mobility of
computational devices gave rise to the mobile process calculi, called
so because they envisioned a network of computational processes
connected by a dynamic communication topology that evolves as the
processes compute. In a sense the computational processes contemplated
by these formalisms are mobile in much the same way as participants at
a conference who randomly bump into one another and trade mobile
numbers and email addresses, or molecules that randomly bump into one
another and exchange electrons. It was this innovation of mobility
that greatly expanded the scope of applicability of these formalisms.

Perhaps the most paradigmatic version of these calculi is Milner,
Parrow, and Walker's $\pi$-calculus
\cite{DBLP:journals/iandc/MilnerPW92a},
\cite{DBLP:journals/iandc/MilnerPW92b}, which Milner refined with
\cite{DBLP:journals/mscs/Milner92} and \cite{milner91polyadicpi}. In
the $\pi$-calculus the dominant metaphor is very much processes as
mobile agents (e.g., conference participants, or molecules) bumping
into one another exchanging channels over which they can
communicate. Yet, another equally compelling metaphor is that
processes are mobile agents swimming in various nested compartments or
membranes. These membranes are called \emph{ambients} because they
both surround and are surrounded by other ambients, but also because
they \emph{move}. That is, they can enter or exit one another are. In
this sense the membranes themselves are the computational
agents. These intuitions are formalized in Cardelli and Gordon's
ambient calculus \cite{DBLP:journals/tcs/CardelliG00}, and
considerably refined by Cardelli's brane calculi
\cite{DBLP:conf/cmsb/Cardelli04}.

Note that it is common practice in the mobile process calculi
literature to refer to channels as \emph{names} and in this note we
will use these terms interchangeably. In the more abstract setting of
that includes calculi like the ambient calculus where names are really
tags associated with mobile membranes this more generic moniker makes
sense. In both the $\pi$ and ambient calculi names are the means of
synchronization and rendezvous for mobile processes.

It is notable that to this day there is no fully abstract encoding of
the ambient calculus into the $\pi$-calculus, suggesting that the
ambient calculus is somehow more expressive than the
$\pi$-calculus. That said, expressivity is not always a benefit. It is
quite difficult to map the ambient style of computation to trustless
distributed Internet-based protocols. The expressivity requires a
great deal more coordination which, in turn, requires a great deal
more trust. Meanwhile, the $\pi$-calculus has a ready mapping onto a
wide range of Internet protocols, which i exploited vigorously in
building Microsoft's BizTalk Process Orchestration. And the rho
calculus has an even better mapping to Internet protocols, which i
exploited even more vigorously in building RChain and F1R3FLY.io.

\subsection{Hennessy-Milner and spatial logics, and session types}

Companion to these calculi are logics and type systems which enable
practitioners not only to reason about the correctness of individual
agents, or specific ensembles of agents, but also about entire classes
of agents, identified as the witnesses of logical formulae. These
began with the Hennessy-Milner logics which build on Kripke's
interpretation of modal logic \cite{milner91polyadicpi}. More
recently, these logics have been given new superpowers in the form of
being able to detect structural properties. Notably, we find Cardelli
and Caires's groundbreaking spatial logic
\cite{DBLP:conf/fossacs/Caires04} \cite{DBLP:journals/iandc/CairesC03}
\cite{DBLP:journals/tcs/CairesC04}. The interested reader might also
look at \emph{session types}
\cite{DBLP:conf/wsfm/Dezani-Ciancaglinid09}.

This topic is worthy of a much richer and deeper discussion. However,
we relegate that to a separate paper in which we unify them all with a
single algorithm, the $\mathsf{OSLF}$ algorithm. This takes as input a
model of computation formatted as a graph structured lambda theory and
outputs a spatial-behavioral type system for that model of
computation. The type system enjoys the Hennessy-Milner property,
namely that two computations inhabit the same bisimulation equivalence
class iff they satisfy exactly the same set of types. We leave the
details to the paper describing this algorithm.

\subsection{Compositionality and bisimulation}
Among the many reasons for the continued success of the mobile process
calculi's approach to organizing and analyzing computation are two
central points. First, the process algebras provide a compositional
approach to the specification, analysis and execution of concurrent
and distributed systems. Owing to Milner's original insights into
computation as interaction \cite{DBLP:journals/cacm/Milner93}, the
process calculi are so organized that the behavior ---the semantics---
of a system may be composed from the behavior of its components. This
means that specifications can be constructed in terms of components
---without a global view of the system--- and assembled into
increasingly complete descriptions.

The second central point is that process algebras have a potent proof
principle, yielding a wide range of effective and novel proof
techniques \cite{DBLP:conf/concur/SangiorgiM92}
\cite{DBLP:conf/fmco/Sangiorgi05}
\cite{DBLP:journals/toplas/Sangiorgi09}. In particular,
\emph{bisimulation} encapsulates an effective notion of process
equivalence that has been used in applications as far-ranging as
algorithmic games semantics \cite{DBLP:conf/sas/Abramsky05} and the
construction of model-checkers \cite{caires_2004}. The essential
notion can be stated in an intuitively recursive formulation: a
\emph{bisimulation} between two processes $P$ and $Q$ is an
equivalence relation $E$ relating $P$ and $Q$ such that: whatever
action of $P$ can be observed, taking it to a new state $P'$, can be
observed of $Q$, taking it to a new state $Q'$, such that $P'$ is
related to $Q'$ by $E$ and vice versa. $P$ and $Q$ are
\emph{bisimilar} if there is some bisimulation relating them. Part of
what makes this notion so robust and widely applicable is that it is
parameterized in the actions observable of processes $P$ and $Q$, thus
providing a framework for a broad range of equivalences and up-to
techniques \cite{DBLP:conf/concur/SangiorgiM92} all governed by the
same core principle \cite{DBLP:journals/toplas/Sangiorgi09}.

\section{The reflective higher order calculus}

It is in the context of the evolution of the mobile process calculi
that Meredith and Radestock's \emph{r}eflective
\emph{h}igher-\emph{o}rder, or rho calculus arises \footnote{The
$\pi$-calculus gets its name because it was the first Greek letter
available after the $\lambda$ in Church's calculus by that name, and
just happens to be the Greek letter corresponding to the English 'p'
for process. Likewise, the rho calculus gets its name from its
delivery of higher-order features through the use of reflection, and
it just happens that the acronym spells out \emph{rho}, the English
spelling of the Greek letter that comes after $\pi$.}
\cite{DBLP:journals/entcs/MeredithR05}. While it is possible to give a
faithful encoding of the ambient calculus into the rho calculus, the
rho calculus can be viewed as the natural successor to the
$\pi$-calculus. One way to look at the relationship between the two is
that the $\pi$-calculus is not actually a single calculus. Instead, it
is a recipe or \emph{function} for constructing a calculus of agents
that communicate over \emph{some} collection of channels. That is, it
is parametric in a \emph{theory} of channels\footnote{Note that here
we are using the term 'theory' in a technical sense, such as Peano
arithmetic, or a Lawvere theory. We don't mean theoretical in the
usual sense of that word, but rather a collection of entities (such as
the Natural Numbers) that can be generated from a small set of
rules.}. This intuition can be made very precise and corresponds to a
particular strategy of implementation called two-level type
decomposition. But, if the $\pi$-calculus is a function taking a
theory of channels to a theory of agents communicating over those
channels, then the rho calculus is the \emph{least fixed point} of
that function. The rho calculus says we need look no farther than the
\emph{communicating agent themselves} for our theory of channels.

More specifically, the calculus envisions a new kind of distinction,
one between the \emph{code} of a process, and a \emph{running
instance} of a process. It will not have escaped the reader's
attention that this is a commonplace notion in the practice of
building concurrent and distributed systems. Indeed, the development
of the rho calculus stole a page from the previous evolution of
process algebras. The step introducing mobility was the theory coming
into better alignment with practice. Computing devices became mobile,
to keep up the formalisms had to follow suit and devise a notion of
mobile processes. Likewise, the practice of building the software that
comprises concurrent and distributed systems makes a distinction
between code and running instances of the code. We argue, however,
that providing an account of this distinction in our formalism is much
more fundamental than just keeping up with practice. We argue that
this reason this is such a ubiquitous practice is because it is a
fundamental aspect of the underlying order and organization of
computation.

\subsection{Coding as a fundamental aspect of computation}

Indeed, G\"odel's incompleteness results fundamentally depend on
coding. To achieve the self-reference needed to construct a logical
statement of the form ``This sentence is not provable.'' G\"odel
encoded logical formulae into the natural numbers and then used
logical statements about the natural numbers to construct
self-referential statements like that one. In his construction, the
coding trick appears to be exogenous to logic and arithmetic, indeed a
careful study of these formalisms shows that it is, in fact,
\emph{intrinsic}. Moreover, this intrinsic, loopy self-reference that
arises from the notion of code is also at the heart of Turing's
solution to the Entscheidungsproblem and Cantor's separation of the
countably infinite from the uncountably infinite.

Moreover, the scope of this phenomenon is not restricted to some of
the most profound results in mathematics and logic over the last
couple of centuries. It's also at the heart of the origin of
life. Some four billion years ago chemistry stumbled on a way to code
for the production of proteins involved in certain self-replicating
chemical reactions. Arguments about the validity of the Weissman
barrier to one side \cite{William2018TheGA}, it is a matter of
experimentally established fact that the codons at the heart of the
genetic apparatus provide the mechanism for both the self-replication
of the chemical processes underlying life, but also the transmission
of inherited information about fitness to environment. As such, we
believe that coding is a much more profound phenomenon than has been
appreciated and deserves to be given a first-class representation in
our computational formalisms.

Once reflection is introduced into the mobile process calculus
setting, higher-order capabilities and recursion are child's
play. More surprisingly, so is the mysterious $\mathsf{new}$ operator
for generating fresh channels. But, we are getting ahead of
ourselves. Suffice it to say that using reflection as a first-class
representation of the trick of coding -- a trick originally discovered
by life and then independently rediscovered by Cantor, G\"odel, and
Turing -- is a real banger.

Intriguingly, not only does introducing coding explicitly into the
formalism solve a number of practical problems associated with the
$\pi$-calculus, it simultaneously solves a number of problems
associated with correspinding logics and type systems, as well. The
interested reader should see namespace logic
\cite{DBLP:conf/tgc/MeredithR05}.

\section{An idiosyncratic view of related work}

Here we direct the reader to a small sample of the considerable work
on the mobile process calculi. This sample is far from representative
of the overall body of work, but instead constitutes a snapshot of
what your author felt was salient at some point during the last twenty
years. Indeed, the reader is encouraged to investigate the literature
on her own to get an independent perspective on this rich field of
research.

\begin{itemize}
\item telecommunications, networking, security and application level protocols
\cite{DBLP:conf/popl/AbadiB02} 
\cite{DBLP:journals/tcs/AbadiB03} 
\cite{DBLP:conf/epew/BrownLM05} 
\cite{DBLP:conf/fossacs/LaneveZ05}; 
\item programming language semantics and design
\cite{DBLP:conf/epew/BrownLM05}
\cite{djoin}
\cite{DBLP:conf/afp/FournetFMS02}
\cite{DBLP:journals/toplas/SewellWU10};
\item webservices
\cite{DBLP:conf/epew/BrownLM05}
\cite{DBLP:conf/fossacs/LaneveZ05}
\cite{DBLP:conf/wise/Meredith03};
\item{blockchain}
  \cite{meredith_2017}
\item and biological systems
\cite{DBLP:conf/cmsb/Cardelli04}
\cite{DBLP:conf/esop/DanosL03}
\cite{DBLP:conf/psb/RegevSS01}
\cite{DBLP:journals/ipl/PriamiRSS01}.
\end{itemize}


% section concurrent_process_calculi_and_spatial_logics_ (end)
