\subsection{Summation}

\subsubsection{Mixed summation}
The presentation given so far is often referred to as the monadic,
asynchronous version of the rho-calculus. There is a natural polyadic
version, in which send and receive result in an exchange of a tuple of
processes. Further there is a natural extension in which we recognize
that certain processes can be made to behave like values and can
therefore be replaced by more familiar representations of those
values, and on top of this extension it becomes natural and compelling
to add pattern-matching to send and receive.

All of these extensions, which we will explore in some detail in the
subsequent sections, really only constitute syntactic sugar. Values,
pattern-matching, let expressions, and more can be desugared back down
to the original calculus. However, summation, aka non-deterministic
choice, where a process is in a superposition of states that get
``collapsed'' through interaction with and environment, affords
versions of the calculus that significantly extend the expressive
power of the calculus.

Specifically, mixed summation, or non-deterministic choice over both
guarded input ($\mathsf{for}$-comprehension) and output is not merely
a conservative extension of the calculus. Although there is an
encoding of the calculus with mixed summation to the asynchronous
calculus, it is not par-preserving. That is, if $\meaningof{-}_{async}
: \MixSumProc \to \Proc$ is a mapping from the rho-calculus with mixed
summation to the asynchronous polyadic rho-calulus, then it cannot be
the case that

\begin{mathpar}
  \meaningof{P\mathsf{|}Q}_{async} = \meaningof{P}_{async}\mathsf{|}\meaningof{Q}_{async}
\end{mathpar}

for any such encoding. For good measure we throw in synchronous
communication, but it is the mixed summation that constitutes the real
jump in expressive power. These features are useful when dealing with
probabilistic exection for both stochastic and quantum regimes. When
relating our model of concurrent computation to these other notions it
is important to track the points where there are significant increases
in expressive power of our target language.

Because Milner's presentation of the polyadic $\pi$-calculus
with mixed summation is so parsimonious we use it as a template for a
similar version of the rho-calculus.

\begin{mathpar}
  \inferrule* [lab=summation] {} {{M,N} \bc \pzero \;|\; x.A \;|\; M+N}
  \and
  \inferrule* [lab=agent] {} {A \bc (\arrvec{x})P \;| \; [\arrvec{P}]Q}
  \and \\
  \inferrule* [lab=process] {} {P,Q \bc M \;|\; P|Q \;|\; \mathsf{*}x}
  \and
  \inferrule* [lab=name] {} {x \bc \mathsf{@}P}
\end{mathpar}

In this presentation we adopt the syntactic conventions

\begin{mathpar}
  \mathsf{for(}\arrvec{y} \leftarrow x\mathsf{)}P := x.(\arrvec{y})P
  \and
  x\mathsf{!}(\arrvec{Q})\mathsf{;}P := x.[\arrvec{Q}]P
\end{mathpar}

The structural equivalence is modified thusly.

\begin{definition}
  The {\em structural congruence} $\equiv$ between processes is the
  least congruence containing alpha-equivalence and satisfying the
  commutative monoid laws (associativity, commutativity and $\pzero$
  as identity) for parallel composition $|$ and summation $+$.
\end{definition}

The $\mathsf{COMM}$ rule is modified to incorporate non-deterministic choice.

\begin{mathpar}
  \inferrule* [lab=COMM] {x_{t} \;\nameeq\; x_{s}, \;\;\; |\arrvec{y}| = |\arrvec{Q}|} {(R_1 + \mathsf{for}( \arrvec{y} \leftarrow x_{t} )P) \;\mathsf{|}\; (x_{s}!(\arrvec{Q}).P' + R_2)
  \red P\substn{\arrvec{\quotep{Q}}}{\arrvec{y}}\mathsf{|}P'}
\end{mathpar}

And contexts are likewise extended in the obvious manner.

\begin{mathpar}
  \inferrule* [lab=summation-context] {} {{K_{M}} \bc \Box \;|\; x.K_{A} \;|\; K_{M}+M}
  \and
  \inferrule* [lab=agent-context] {} {K_{A} \bc (\arrvec{x})K_{P} \;|\; [\arrvec{P},K_{P},\arrvec{P'}]Q \;|\; [\arrvec{P}]K_{P}}
  \and \\
  \inferrule* [lab=process-context] {} {K_{P} \bc K_{M} \;|\; P\mathsf{|}K_{P}}
\end{mathpar}

\begin{remark}
  Below we detail some of the useful syntactic sugar that makes
  reasoning and programming with the rho calculus considerably more
  convenient. The reader will be able to verify that that all the
  notational conventions developed below still make sense for the
  calculus extended with mixed summation.
\end{remark}

%% \begin{mathpar}
%%   \mathsf{for(}{y_{1}}\leftarrow{x_{1}}\mathsf{;}\;\ldots\mathsf{;}\;{y_{n}}\leftarrow{x_{n}}\mathsf{)}P
%%   \and \\
%%   \mathsf{let}\; x_{1} \;\mathsf{=}\; v_{1} \mathsf{;}\; \ldots \mathsf{;}\; x_{n} \;\mathsf{=} \; v_{n}  \;\mathsf{in}\; P
%% \end{mathpar}

\section{Useful syntactic sugar}

\subsection{First class values}
While it is standard in calculi such as the $\lambda$-calculus to
define a variety of common values such as the natural numbers and
booleans in terms of Church-numeral style encodings, it is equally
common to simply embed values directly into the calculus. Not being
higher-order, this presents some challenges for the $\pi$-calculus,
but for the rho-calculus, everything works out very nicely if we treat
values, e.g. the naturals, the booleans, the reals, etc as processes. This
choice means we can meaninfully write expressions like
$x!(\mathsf{5})$ or $u!(\mathsf{true})$, and in the context
$\prefix{x}{y}P[*y] \mathsf{|} x!(\mathsf{5})$ the value $\mathsf{5}$
will be substituted into $P$. Indeed, since operations like addition,
multiplication, etc.  can also be defined in terms of processes, it is
meaningful to write expressions like
$\mathsf{5}\mathsf{|}\mathsf{+}\mathsf{|}\mathsf{1}$, and be confident
that this expression will reduce to a process representing
$\mathsf{6}$. Thus, we can also use standard mathematical expressions,
such as $\mathsf{5+1}$, as processes, and know that these will
evaluate to their expected values. Further, when combined with
$\mathsf{for}$-comprehensions, we can write algebraic expressions,
such as $\prefix{x}{y}\mathsf{5}\mathsf{+}\mathsf{*}y$, and in
contexts like
$(\prefix{x}{y}\mathsf{5}\mathsf{+}\mathsf{*}y)\mathsf{|}\outputp{x}{\mathsf{1}}$
this will evaluate as expected, producing the process (aka value)
$\mathsf{6}$.

\begin{mathpar}
  \inferrule* [lab=values] {} {P \bc \dots \bm \mathsf{Bool}
    \bm \mathsf{Long}
    \bm \mathsf{String}
    \bm \mathsf{Uri} \bm \mathsf{Collection} \bm \ldots}
\end{mathpar}

\begin{mathpar}
  \inferrule* [lab=collection] {} {\mathsf{Collection} \bc \texttt{[} \mathsf{[}P\mathsf{]} \texttt{]}
    \bm \texttt{(} P \texttt{,}\; \mathsf{[}P\mathsf{]} \texttt{)}
    \bm \texttt{Set} \texttt{(} \mathsf{[}P\mathsf{]} \texttt{)}
    \bm \texttt{\{} \mathsf{[}P \texttt{:} P \mathsf{]} \texttt{\}}} \\
\end{mathpar}

\subsection{Pattern matching}
Remember that rho allows processes to be sent: $x\mathsf{!}(Q)$. Since
we can send values, it will be useful to be able to pattern match on
values. The following pattern-matching syntax captures very commonly
used idioms.

\begin{mathpar}
  \inferrule* [lab=pattern] {} {\mathsf{ProcPattern} \bc \texttt{\_}
    \bm \mathsf{Var}
    \bm \texttt{@} \mathsf{ProcPattern}
    \bm\; \mathsf{VarRefKind}\; \mathsf{Var} \bm \mathsf{LogicalPattern} \bm \mathsf{ValuePattern}} \\
  \and
  \inferrule* [lab=logical-pattern] {} {\mathsf{LogicalPattern} \bc \mathsf{ProcPattern} \;\vee\; \mathsf{ProcPattern}
    \bm \mathsf{ProcPattern} \;\wedge\; \mathsf{ProcPattern}
    \bm \texttt{\~{}} \mathsf{ProcPattern}} \\
  \and
  \inferrule* [lab=value-pattern] {} {\mathsf{ValuePattern} \bc \mathsf{Ground}
    \bm \mathsf{Collection}
    \bm \texttt{Nil}
    \bm \mathsf{SimpleType}} \\
  \and
  \inferrule* [lab=var-ref-kind] {} {\mathsf{VarRefKind} \bc \texttt{=} \bm \texttt{=} \texttt{*}} \\
\end{mathpar}
\subsection{Let syntax}
We achieve even greater compression and a more
familiar notation if we also adopt the notation

\begin{mathpar}
  \mathsf{let}\; x \;\mathsf{=}\; v \;\mathsf{in}\; P := (\mathsf{new} \; u)(\prefix{u}{@x}P)\mathsf{|}\outputp{u}{v}
\end{mathpar}
\subsection{Joins}

\begin{mathpar}
  \inferrule* [lab=io] {} {\mathsf{Proc2} \bc \mathsf{Proc3} \;\bm\; \;\ldots\; \;\bm\; \texttt{for}\; \texttt{(} \mathsf{[Receipt]} \texttt{)} \texttt{\{} \;\mathsf{Proc}\; \texttt{\}}} \\
    \and
  \inferrule* [lab=] {} {\mathsf{Receipt} \bc \mathsf{[LinearBind]} \;\bm\; \mathsf{[RepeatedBind]} \;\bm\; \mathsf{[PeekBind]}} \\
  \inferrule* [lab=] {} {\mathsf{[Receipt]} \bc \mathsf{Receipt} \;\bm\; \mathsf{Receipt} \;\texttt{;}\; \mathsf{[Receipt]}} \\
  \inferrule* [lab=] {} {\mathsf{LinearBind} \bc \mathsf{[Name]} \; \mathsf{NameRemainder} \;\sngllarrow\; \mathsf{NameSource}} \\
  \inferrule* [lab=] {} {\mathsf{[LinearBind]} \bc \mathsf{LinearBind} \;\bm\; \mathsf{LinearBind} \;\texttt{\&}\; \mathsf{[LinearBind]}} \\
  \inferrule* [lab=] {} {\mathsf{NameSource} \bc \mathsf{Name} \;\bm\; \mathsf{Name} \texttt{?!}\;\bm\; \mathsf{Name} \texttt{!?} \texttt{(} \mathsf{[Proc]} \texttt{)}} \\
  \inferrule* [lab=] {} {\mathsf{RepeatedBind} \bc \mathsf{[Name]}\; \mathsf{NameRemainder} \;\dbltllarrow\; \mathsf{Name}} \\
  \inferrule* [lab=] {} {\mathsf{[RepeatedBind]} \bc \mathsf{RepeatedBind} \;\bm\; \mathsf{RepeatedBind} \texttt{\&} \mathsf{[RepeatedBind]}} \\
  \inferrule* [lab=] {} {\mathsf{PeekBind} \bc \mathsf{[Name]}\; \mathsf{NameRemainder} \;\dblhdlarrow\; \mathsf{Name}} \\
  \inferrule* [lab=] {} {\mathsf{[PeekBind]} \bc \mathsf{PeekBind} \;\bm\; \mathsf{PeekBind} \;\texttt{\&}\; \mathsf{[PeekBind]}} \\
\end{mathpar}

\subsection{Unguessable versus unforgeable names}
Since all the names of the rho calculus are generated from the codes
of processes we know all of them up front. Security on channels,
therefore, amounts to unguessability. There are an infinite number of
names and we have to arrange our protocols to make it very, very hard
to guess which channels are in use at any given time. On the one hand,
we could delegate that to some black box which delivers us the next
unguessable name when we ask for it. In this sense, it is therefore
useful to reintroduce the $\pi$-calculus' $\mathsf{new}$ operator as a
standin for that black box.

However, in many settings, such as in the $\mathsf{RChain}$ and
$\mathsf{F1R3FLY.io}$ implementations execution is arranged so that
names are not merely unguessable, but \emph{unforgeable}, meaning that
the execution environment guarantees not to allow outside agents to
generate certain names. Thus, even if some malefactor guesses a
particular name, it cannot eavesdrop on the channel associated with it
because it can't execute code that uses that name because it didn't
have the right to generate it. It could only do so had its code been
in a scope where the generated name was sent to it by a (presumably)
willing party.

Thus, the $\mathsf{new}$ operator not only provides useful abstraction
over a black boxed unguessability algorithm, but can also be used to
support unforgeability. Hence, we reintroduce to rho as a conservative
extension.

\begin{mathpar}
  \inferrule* [lab=unforgeable] {} {P \bc \ldots \bm \texttt{new}\; \mathsf{[NameDecl]}\; \texttt{in}\; P} \\
  \and
  \inferrule* [lab=] {} {\mathsf{NameDecl} \bc \mathsf{Var} \bm \mathsf{Var} \texttt{(} \mathsf{Uri} \texttt{)}} \\
  \and
  \inferrule* [lab=] {} {\mathsf{[NameDecl]} \bc \mathsf{NameDecl} \bm \mathsf{NameDecl} \texttt{,} \mathsf{[NameDecl]}} \\
\end{mathpar}

\subsection{Guarded summation}

One form of summation that \emph{is} a conservative extension of the
core rho calculus is guarded summation. This is so useful that it
deserves its own sugar.

\begin{mathpar}
  \inferrule*[lab=summation]{}{\mathsf{P} \bc \ldots \texttt{select}\; \texttt{\{} \;\mathsf{[Branch]}\; \texttt{\}}} \\
  \and
  \inferrule* [lab=] {} {\mathsf{Branch} \bc \mathsf{ReceiptLinearImpl} \texttt{=$>$} \mathsf{Proc3}} \\
\inferrule* [lab=] {} {\mathsf{[Branch]} \bc \mathsf{Branch} \\
 \;\bm\; \mathsf{Branch} \mathsf{[Branch]}} \\
\end{mathpar}
