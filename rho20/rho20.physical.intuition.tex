\section{From formalism to physical intuition and back}

\subsection{Indexed namespaces}
Let's begin the discussion by identifying an indexed set of names.

\begin{mathpar}
  \mathsf{Z}(0) = \mathsf{@}0 \\
  \mathsf{Z}(n) = \mathsf{@}(\mathsf{for}( y \leftarrow \mathsf{Z}(n-1) )0) \\
  \mathsf{Z} = \{ \mathsf{Z}(n) \;|\; n \in \mathbb{N} \}
\end{mathpar}

While process states are perfectly suited for using channels as
locations at which to rendezvous for data exchange, correlating that
to more common notions of locations, such as points in a metric space,
is not immediately obvious. The namespace $\mathsf{Z}$, because of its
natural indexed structure, provides an intuitive bridge between the
two ways of thinking about locations. Indeed, given a map like this
from the naturals to names, it should be relatively straightforward
for the reader to extend this idea to the rationals, and indeed to the
computable reals.

\subsection{Processes as physical forces}
Next, we consider a collection of processes that we will think of as
components in a ``force'', where by force we mean something like a
physical force, such as the electromagnetic force or the strong or
weak forces.

\begin{mathpar}
  \mathbb{F}(n) = \mathsf{for}( y \leftarrow \mathsf{Z}(n) )\{\mathsf{Z}(n+1)\mathsf{!}(\mathsf{*}y) \;\mathsf{|}\;\mathbb{F}(n)\} \\
  \mathbb{F} = \Pi_{n \in \mathbb{N}} \mathbb{F}
\end{mathpar}

Now, model the action of this force on an object, we represent the
object as a process as well, say $Q$, and locate it somewhere in
$\mathsf{Z}$, i.e. $\mathsf{Z}(i)\mathsf{!}(Q)$. Then the action of
$\mathbb{F}$ on $Q$, is just the parallel composition,
$\mathbb{F}\;\mathsf{|}\;\mathsf{Z}(i)\mathsf{!}(Q)$. Note that
because $\mathsf{Z}(i)\mathsf{!}(Q)$ cannot reduce, it tends to stay
at rest. However, when acted on by an outside force, namely
$\mathbb{F}$, we notice that $Q$ is relocated first to
$\mathsf{Z}(i+1)$ and then $\mathsf{Z}(i+2)$ and tends to stay on this
linear trajectory unless there are other forces at play.

It is also worth noting that there is a natural notion of \emph{rate}
of travel. Specifically, $\mathsf{COMM}$ events provide a local notion
of time. We can see that $\mathbb{F}$ moves objects located in
$\mathsf{Z}$ along at a rate of one unit of distance (as measure by
the index) per $\mathsf{COMM}$ event.

\begin{mathpar}
  \mathbb{F}\;\mathsf{|}\;\mathsf{Z}(i)\mathsf{!}(Q) \red \mathbb{F}\;\mathsf{|}\;\mathsf{Z}(i+1)\mathsf{!}(Q) \\
\end{mathpar}


It is quite straightforward to write down forces that are non-linear. For example, 

\begin{mathpar}
  \mathbb{G}(n) = \mathsf{for}( y \leftarrow \mathsf{Z}(n) )\{\mathsf{Z}(n^{2})\mathsf{!}(\mathsf{*}y) \;\mathsf{|}\;\mathbb{G}(n)\} \\
  \mathbb{G} = \Pi_{n \in \mathbb{N}} \mathbb{G}
\end{mathpar}

The reader is invited to write down, following this method, an analog
of Newtonian gravity. Of course, to do this requires we associate to
objects a notion of mass. One of the most natural ways to do this is
to measure the computational complexity of a process and use (a
function of) this number as a stand in for mass.
