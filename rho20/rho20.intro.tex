\section{Introduction}\label{sec:introduction} % (fold)
This December will find my submission of the first paper about the rho
calculus to be 20 years in the rear view mirror. A lot has happened
since then. Apart from production implementations of scalable
decentralized asset management systems based on the calculus, a lot of
work has been done on the calculus itself. Indeed, just this May
(2024) i discovered a conservative extension of the calculus that
fills a hole in the account of reflection that has long bothered
me. And, last April (2023) i found a fuzzy version of the rho calculus
that is just waiting for someone to take further. In 2018 i discovered
a variant of the calculus (which i dubbed the \emph{space calculus})
that makes it a computational analog (in a certain precise sense) of
solutions of the Einstein field equations.

Thankfully, i am not the only one who has found the rho calculus
interesting enough to think about. Stay and Williams studied a variant
they called the $\pi$-rho calculus as an example of a system that
could be typed in their native types construction. And Stian Lybech
showed that rho calculus is more expressive than the the
$\pi$-calculus. The list goes on.

The aim of this note is to collect, and summarize in one place,
developments over the last 20 years in the study of the rho
calculus. i suspect that there are more researchers than i know about
who are interested in the questions raised by the calculus. Hopefully,
this note can serve that community, and possibly spark wider interest
due to the discoveries to date.

\subsection{Summary of contributions and outline of paper}
More specifically, this note will develop the following.
\begin{itemize}
  \item the modern, programming-language-friendly syntax for the calculus;
  \item a handful of useful syntactic sugar coatings that make the
    calculus a suitable semantic framework for a realistic
    protocol-oriented programming language, including, but not limited
    to:
    \begin{itemize}
      \item adding first class values;
      \item adding unforgeable names (the $\pi$-rho calculus);
      \item adding syntax for joins and sequences and choice;
    \end{itemize}
  \item a procedurally reflective rho calculus;
  \item the fuzzy variant of rho;
  \item multi-level agency mechanisms in rho;
  \item the space calculus;
  \item probabilistic and quantum variations of rho;
  \item revisiting the encodings of $\pi$ into rho and vice versa.
\end{itemize}

But, we will also touch briefly on the techniques used to represent
reflective computation in the rho calculus and demonstrate that they
are not bound to this specific setting. In particular, they can be
applied the the $\lambda$-calculus and to set theory, yielding
variations of these theories that have very useful features.

Additionally, we will look at relating the rho calculus to physical
intuition. There has been a great deal of work on using mobile process
calculi to describe network protocols and concurrent algorithms. It is
not too much of an abstraction to see how these techniques apply to
physical processes such as chemical reactions, signaling regimes
within and between cells. But the rho calculus can also be used to
model ordinary classical physics problems. We illustrate some of
these.

Finally, we will conclude with some thoughts about the foundations of
mathematics and its relationship to agency. As a preview, i believe
that set theory (and even to some extent category theory) are theories
of \emph{data structures}. Implicitly, the community has believed that
(only) mathematicians have agency and their formalisms merely carry
information content between these agents. But the advances in
artificial intelligence ($\mathsf{AI}$) tell a different story. To
some extent $\mathsf{AI}$ has brought their formalisms to life and is
on the verge of giving them agency. As a result, i want to argue that
a proper foundation of mathematics needs to include a notion of agency
and that the ingredients of a foundational account of agency are
present in the mobile process calculi, generally, and a pragmatically
foundational account is present in the rho calculus, specifically.

% section introduction (end)
