\section{Conclusions and future work}

One way to understand the rho-calculus is a study in the \emph{seams}
of the $\pi$-calculus. There are several holes in Milner's calculus:

\begin{itemize}
  \item \emph{the zero process}, $\pzero$, is an input to the theory;
    adding other elements here corresponds to supplying ``builtin''
    processes, and as such constitutes a way to add seamlessly add
    values to the language;
  \item \emph{names} are an input to the theory, which we discussed at length;
  \item \emph{the source of non-determinism} is an input to the theory.
\end{itemize}

The results in the stochastic and quantum regimes can really be seen
as an initial exploration of the a general theory of this third
dependency.

Stepping back to an even wider perspective we need to understand the
rho calculus not as a single model of computation. Instead, it is a
way of thinking about computation. It says that \emph{computation
  arises from interaction}. This is a remarkable shift from the point
of view normally espoused in theoretical computer science. Both Turing
and Church's models saw computation as \emph{functions acting on
  data}.

Indeed, there is a similar kind of divergence of viewpoints in the
physical sciences. Chemistry arises from interactions of
molecules. With the exception of gravitation, standard model physics
arises from interaction of particles. Yet, all these disciplines are
using the computational toolkit initially developed by Newton, which
again envisages computation as functions acting on data.

Even more modern views of computation, such as are embodied in
category theory, are still strongly oriented towards this functional
view. A topos, for example, is really just a Cartesian closed category
($\mathsf{CCC}$) with a subobject classifer; but as a $\mathsf{CCC}$
it is a model of the $\lambda$-calculus, which is decidedly functional
in its perspective.

We hope that this review of the vast range of computational phenomenon
successfully, indeed elegantly, accounted for by the interactive
paradigm will inspire the reader who has made it this far to launch
her own investigation into the matter. The rho calculus provides not a
single model of computation, but a conceptual, and mathematically
precise toolkit for explore the interactive paradigm.

But the development doesn't stop here. In much the same way that the
rho calculus and its variants provide a framework for the
investigation of individual protocols, algorithms, and processes seen
as patterns of interaction, the $\mathsf{OSLF}$ algorithm provides a
framework for the investigation of populations of protocols,
algorithms, and processes seen as patterns of interactions. Rather
than having to craft -- by hand -- a new type system for each variant
of the rho calculus (or indeed each variant model of computation) that
is discovered, $\mathsf{OSLF}$ allows one to \emph{generate} a type
system automatically. The formulae or types of these generated type
systems correspond to populations of processes that are bisimilar to
each other. This gives an entirely different vantage point on the
relationships of computation, logic, and interaction.

In particular, this cuts to the very foundations of mathematics. From
one perspective we can view set theory as a \emph{data} structure
designed to support protocols (namely proofs) between agents (namely
mathematicians). This view might have made sense as a language for
mathematicians practicing before the advent of high performance
networked computing infrastructure, but in the age of
$\mathsf{ChapGPT}$ we have to acknowledge that computer programs --
which are themselves just mathematical objects -- have enough agency
not only to do proofs and come up with counter examples, but to make
conjectures!

We submit that a language of mathematics that supports modern practice
should embrace the language of agency and take it to its bosom. That
is, we believe that the mobile process calculi not only provide a much
more subtle language for the data structures used to conduct proofs,
but also a language to reason about agents that used those data
structures, be they mathematicians, computer programs or collectives
comprised of both.

% subsection other_calculi_other_bisimulations_and_geometry_as_behavior (end)


