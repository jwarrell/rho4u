\section{The syntax and semantics of the core rho calculus}\label{sub:the_syntax_and_semantics_of_the_notation_system} % (fold)

We now summarize a technical presentation of the core calculus. The
typical presentation of such a calculus generalizes a generators and
relations presentation of an algebra
\cite{nlab:generators_and_relations}. The grammar below, describing
term constructors in the language of process expressions (which we
abbreviate to processes in the sequel), freely generates the set of
processes. We denote this set $\Proc$, and the set of channels (aka
names) over which these processes communicate by $\QProc$. The set
$\Proc$ (and by extension $\QProc$) is then quotiented by a relation
known as structural congruence, denoted $\equiv$, and it is over this
set that the notion of computation is expressed.

In particular, computation is expressed as a small handful of rewrite
rules which should be viewed as defining a state transition
relation. Thus, process expressions capture \emph{states} and the form
of the expression determines how one process (state) \emph{might}
evolve to another (state). The word 'might' is doing some heavy
lifting in that previous sentence, as the calculus is \emph{not}
deterministic, in fact it is not even confluent. As such, the calculus
can be used to represent all manner of common human (and natural)
resource distribution protocols, such as first come first served,
which sits at the heart of everything from airline reservation systems
to concert ticket sales. By comparison, Church's $\lambda$-calulus
(the core of programming languages such as $\mathsf{Haskell}$ or
$\mathsf{Scala}$) does not -- without considerable modification --
natively support the expression of such protocols.

\subsection{The rho calculus in less than a page}

\begin{mathpar}
\inferrule* [lab=process] {} {P, Q \bc \pzero \;\bm\; \mathsf{for}(
  y \leftarrow x )P \;\bm\; x\mathsf{!}(Q) \;\bm\;
  \mathsf{*}x \;\bm\; P\mathsf{|}Q } \and \inferrule* [lab=name] {}
            {x, y \bc \mathsf{@}P }
\end{mathpar}

\begin{mathpar}
  \inferrule* [lab=equiv] {} {P\mathsf{|}\pzero \equiv P \;\; P\mathsf{|}Q \equiv Q\mathsf{|}P \;\; P\mathsf{|}(Q\mathsf{|}R) \equiv (P\mathsf{|}Q)\mathsf{R}} \\
  \and
  \inferrule* [lab=alpha] {} {\mathsf{for}(y \leftarrow x )P \equiv \mathsf{for}(z \leftarrow x )(P\{z/y\} \; \mathsf{if} z \notin \mathsf{FN}(P)}
\end{mathpar}

\begin{mathpar}
  \inferrule* [lab=COMM] {} {\mathsf{for}( y \leftarrow x )P \;\mathsf{|}\; x!(\arrvec{Q})
    \red P\substn{\quotep{Q}}{y}} \\
  %% \and
  %% \inferrule* [lab=PAR]{P \red P'}{P\mathsf{|}Q \red P'\mathsf{|}Q}
  %% \and
  %% \inferrule* [lab=EQUIV]{{P \;\scong\; P'} \andalso {P' \red Q'} \andalso {Q' \;\scong\; Q}}{P \red Q}
\end{mathpar}

This presentation is essentially that of
\cite{DBLP:journals/entcs/MeredithR05} recast with a
programmer-friendly syntax that extends smoothly to fork-join patterns
commonly found in human decision-making processes. The reader should
note that, even with a programmer-friendly syntax, this calculus is
considerably smaller than a corresponding presentation of the
$\pi$-calculus. Indeed is as concise and compact as the
$\lambda$-calculus, yet considerably more expressive. What follows
below is an exegesis of the technical details hidden in this more
compact presentation. Notably, we provide definitions of free and
bound names, we add two more rewrite rules that prescribe how rewrites
may occur in context (necessary, since the calculus is not
deterministic) and other housekeeping matters such as name
equality. The reader more familiar with these notions should feel free
to skim this section, however the section on bisimulation is of some
importance and worthy of more focused attention.

\subsection{The fine print}

\paragraph{Notational interlude} when it is clear that some expression $t$ is a sequence (such as a list or a vector), and $a$ is an object that might be meaningfully and safely prefixed to that sequence then we write $a:t$ for the sequence with $a$ prefixed (aka ``consed'') to $t$. We write $t(i)$ for the $i$th element of $t$. In the sequel we use $\arrvec{x}$ (resp. $\arrvec{P}$) denotes a vector of \emph{names}
(resp. \emph{processes}) of length $|\arrvec{x}|$
(resp. $|\arrvec{P}|$).

\subsubsection{Process grammar}\label{subsub:process_grammar}

\begin{mathpar}
\inferrule* [lab=process] {} {P, Q \bc \pzero \;\bm\; \mathsf{for}(
  y \leftarrow x )P \;\bm\; x\mathsf{!}(Q) \;\bm\;
  \mathsf{*}x \;\bm\; P\mathsf{|}Q } \and \inferrule* [lab=name] {}
            {x, y \bc \mathsf{@}P }
\end{mathpar}

As mentioned above we use $\Proc$ (resp. $\QProc$) to denote the
language of processes (resp. names) freely generated by this
grammar. But how are we to think about expressions in this language,
intuitively?

\begin{itemize}
  \item $\pzero$ is the ground of this tiny little language. It represents the \emph{stopped} process, or the process that does nothing; in some real sense this is the quintessential neutral element in this computational dynamics, as we shall see;
  \item $\mathsf{for}( y \leftarrow x )P$ is the process that is waiting at channel $x$ for some data or message that it will bind to $y$ and then proceed to do $P$; if $\pzero$ is neutral, then $\mathsf{for}$-comprehension is receptive, it waits for action at $x$ before it can do anything;
  \item $x\mathsf{!}(Q)$ is the process that is outputting the datum or message $Q$ (which is also a process) along the channel, $x$; if $\pzero$ is neutral, and $\mathsf{for}( y \leftarrow x )P$ is reception, then $x\mathsf{!}(Q)$ is active; it represents actively sending $Q$ floating down the channel $x$ to be picked up by some $\mathsf{for}( y \leftarrow x )P$;
  \item $\mathsf{*}x$ is the process that takes the code represented by $x$ and begins running it;
  \item $P\mathsf{|}Q$ is the process that is the parallel or concurrent composition of the process $P$ with the process $Q$; it introduces the principle that individual computations can actually be collectives of autonomous, yet coordinating computation.  
\end{itemize}

As foreshadowed by the syntactic category of expressions of the form
$\mathsf{*}x$, a channel or name is merely the code of some process,
$\mathsf{@}P$. In this sense we may think of these two operators
$\mathsf{*}$ and $\mathsf{@}$ as dual to one another. The latter
delivers the code of some process, while the former delivers a running
process from its code.

Next, we quotient this set by the smallest equivalence
relation containing $\alpha$-equivalence that makes $(\Proc, |, 0)$
into a commutative monoid. In order to define this relation we require
a definition of free and bound names to define $\alpha$-equivalence.

\begin{definition}
  \emph{Free and bound names} The calculation of the free names of a
  process, $P$, denoted $\freenames{P}$ is given recursively by
  
  \begin{mathpar}
    \freenames{\pzero} = \emptyset
    \and
    \freenames{\mathsf{for}(y \leftarrow x)(P)} = \{ x \} \cup \freenames{P}\setminus\{y\}
    \and
    \freenames{x!(P)} = \{ x \} \cup \freenames{P}
    \and
    \freenames{P|Q} = \freenames{P} \cup \freenames{Q}
    \and
    \freenames{\mathsf{}{x}} = \{ x \} \\
  \end{mathpar}
  
  An occurrence of $x$ in a process $P$ is \textit{bound} if it is not
  free. The set of names occurring in a process (bound or free) is
  denoted by $\names{P}$.
\end{definition}

\subsection{Substitution}

We use the notation $\id{\{}\arrvec{y} / \arrvec{x} \id{\}}$ to denote
partial maps, $s : \QProc \rightarrow \QProc$. A map, $s$ lifts,
uniquely, to a map on process terms, $\widehat{s} : \Proc \rightarrow
\Proc$. Historically, it is convention to use $\sigma$ to range over
lifted subsitutions, $\widehat{s}$, to write the application of a
substitution, $\sigma$ to a process, $P$, using postfix notation, with
the substitution on the right, $P\sigma$, and the application of a
substitution, $s$, to a name, $x$, using standard function application
notation, $s(x)$. In this instance we choose not to swim against the
tides of history. Thus,

\begin{definition}
  given $x = \quotep{P'}$, $u = \quotep{Q'}$, $s =
  \substn{u}{x}$ we define the lifting of $s$ to $\widehat{s}$ (written
  below as $\sigma$) recursively by the following equations.
  \begin{mathpar}
    0 \sigma := 0 \\
    (P \mathsf{|} Q) \sigma
    :=    
    P\sigma \mathsf{|} Q\sigma \\
    (\mathsf{for}(y \leftarrow v)P) \sigma    
    :=
    \mathsf{for}(z \leftarrow \sigma(v))((P \psubstn{z}{y}) \sigma) \\
    (\lift{x}{Q}) \sigma  
    :=
    \lift{\sigma(x)}{ Q \sigma } \\
    (\dropn{y})  \sigma       
    := 
    \left\{ 
      \begin{array}{ccc} 
        Q' & & y \;\nameeq\; x \\
        \dropn{y} & & otherwise \\
      \end{array}
      \right.
  \end{mathpar} 

  where

  \begin{eqnarray}
    \psubstp{Q}{P}(x) = \id{\{} \quotep{Q} / \quotep{P} \id{\}}(x) = 
    \left\{ 
      \begin{array}{ccc}
        \quotep{Q} & & x \;\nameeq\; \quotep{P} \\
        x & & otherwise \\
      \end{array}
      \right. \nonumber
  \end{eqnarray}
\end{definition}

and $z$ is chosen distinct from $\quotep{P}$, $\quotep{Q}$, the free
names in $Q$, and all the names in $R$. Our $\alpha$-equivalence will
be built in the standard way from this substitution.

\begin{definition}
Then two processes, $P,Q$, are alpha-equivalent if $P = Q\{\arrvec{y}/\arrvec{x}\}$ for
some $\arrvec{x} \in \boundnames{Q},\arrvec{y} \in \boundnames{P}$, where $Q\{\arrvec{y}/\arrvec{x}\}$
denotes the capture-avoiding substitution of $\arrvec{y}$ for $\arrvec{x}$ in $Q$.
\end{definition}

\begin{definition}
  The {\em structural congruence} $\equiv$
  between processes \cite{DBLP:books/daglib/0004377} is the least congruence containing
  alpha-equivalence and satisfying the commutative monoid laws
  (associativity, commutativity and $\pzero$ as identity) for parallel
  composition $|$.
\end{definition}

\begin{definition}
  The {\em name equivalence} $\nameeq$ is the least congruence
  satisfying these equations
  \begin{mathpar}
  \inferrule*[lab=Quote-drop] {}{ \quotep{\dropn{x}} \;\nameeq\; x }
  \and
  \inferrule*[lab=Struct-equiv] { P \;\scong\; Q } { \quotep{P} \;\nameeq\; \quotep{Q} }
  \end{mathpar}
\end{definition}

The astute reader will have noticed that the mutual recursion of names
and processes imposes a mutual recursion on alpha-equivalence and
structural equivalence via name-equivalence. Fortunately, all of this
works out pleasantly and we may calculate in the natural way, free of
concern. The reader interested in the details is referred to the
original paper on the rho calculus
\cite{DBLP:journals/entcs/MeredithR05}.

\begin{remark}\label{rem:no_self_referential_names}
  One particularly useful consequence of these definitions is that
  $\forall P. \quotep{P} \not\in \freenames{P}$. It gives us a
  succinct way to construct a name that is distinct from all the names
  in $P$ and hence fresh in the context of $P$. For those readers
  familiar with the work of Pitts and Gabbay, this consequence allows
  the system to completely obviate the need for a fresh operator, and
  likewise provides a canonical approach to the semantics of
  freshness.
\end{remark}

Equipped with the structural features of the term language we can
present the dynamics of the calculus.

\subsection{Operational semantics}

Finally, we introduce the computational dynamics. What marks these
algebras as distinct from other more traditionally studied algebraic
structures, e.g. vector spaces or polynomial rings, is the manner in
which dynamics is captured. In traditional structures, dynamics is typically
expressed through morphisms between such structures, as in linear maps
between vector spaces or morphisms between rings. In algebras
associated with the semantics of computation, the dynamics is
expressed as part of the algebraic structure itself, through a
reduction reduction relation typically denoted by $\red$. Below, we
give a recursive presentation of this relation for the calculus used
in the encoding.

\begin{mathpar}
  \inferrule* [lab=COMM] {x_{t} \;\nameeq\; x_{s}} {\mathsf{for}( y \leftarrow x_{t} )P \;\mathsf{|}\; x_{s}!(\arrvec{Q})
  \red P\substn{\quotep{Q}}{y}}
  \and
  \inferrule* [lab=PAR]{P \red P'}{P\mathsf{|}Q \red P'\mathsf{|}Q}
  \and
  \inferrule* [lab=EQUIV]{{P \;\scong\; P'} \andalso {P' \red Q'} \andalso {Q' \;\scong\; Q}}{P \red Q}
\end{mathpar}

We write $P\red$ if $\exists Q $ such that $ P \red Q$ and $P\not\red$, otherwise.

\subsection{ Dynamic quote: an example }

Anticipating something of what's to come, let $z = \quotep{P}$, $u = \quotep{Q}$, and $x = \quotep{\lift{y}{\dropn{z}}}$. Now consider applying the substitution,
$\widehat{\id{\{}u / z \id{\}}}$, to the following pair of processes,
$\lift{w}{y!(\dropn{z})}$ and $\lift{w}{\dropn{x}} = \lift{w}{\dropn{\quotep{\lift{y}{\dropn{z}}}}}$.

\begin{eqnarray}
	\lift{w}{\lift{y}{\dropn{z}}}\widehat{\id{\{}u / z \id{\}}}
		& = &
		\lift{w}{\lift{y}{Q}} \nonumber\\
	\lift{w}{\dropn{x}} \widehat{ \id{\{}u / z \id{\}} }
		& = &
		\lift{w}{\dropn{x}} \nonumber
\end{eqnarray}

The body of the quoted process, $\quotep{\lift{y}{\dropn{z}}}$, is
impervious to substitution, thus we get radically different
answers. In fact, by examining the first process in an input context,
e.g. $\mathsf{for}(z \leftarrow x)\lift{w}{\lift{y}{\dropn{z}}}$, we see that the process
under the output operator may be shaped by prefixed inputs binding a
name inside it. In this sense, the combination of input prefix binding
and output operators will be seen as a way to dynamically construct
processes before reifying them as names.

\section{Replication}

As mentioned before, it is known that replication (and hence
recursion) can be implemented in a higher-order process algebra
\cite{DBLP:books/daglib/0004377}. As our first example of calculation with the
machinery thus far presented we give the construction explicitly in
the rho calculus.

\begin{eqnarray}
	D_{x} & := & \prefix{x}{y}{(\binpar{\outputp{x}{y}}{\dropn{y}})} \nonumber\\
	\bangp_{x}{P} & := & \binpar{\lift{x}{\binpar{D_{x}}{P}}}{D_{x}} \nonumber
\end{eqnarray}

\begin{eqnarray}
	\bangp_{x}{P} & & \nonumber\\
	=
	& \lift{x}{(\prefix{x}{y}{(\outputp{x}{y} | \dropn{y})) | P}} 
	      | \prefix{x}{y}{(\outputp{x}{y} | \dropn{y})} & \nonumber\\
	\red
	& (\outputp{x}{y} | \dropn{y})\substn{\quotep{(\prefix{x}{y}{(\dropn{y} | \outputp{x}{y})) | P}}}{y} & \nonumber\\
	=
	& \outputp{x}{\quotep{(\prefix{x}{y}{(\outputp{x}{y} | \dropn{y})) | P}}}
	  | {(\prefix{x}{y}{(\outputp{x}{y} | \dropn{y})) | P}} & \nonumber\\
	\red
	& \ldots & \nonumber\\
	\red^*
	& P | P | \ldots & \nonumber
\end{eqnarray}

Of course, this encoding, as an implementation, runs away, unfolding
$\bangp{P}$ eagerly. A lazier and more implementable replication
operator, restricted to input-guarded processes, may be obtained as follows.

\begin{eqnarray}
\bangp{\prefix{u}{v}{P}} 
	:= 
	\binpar{\lift{x}{\prefix{u}{v}{(\binpar{D(x)}{P})}}}{D(x)} \nonumber
\end{eqnarray}

\begin{remark}
  Note that the lazier definition still does not deal with summation
  or mixed summation (i.e. sums over input and output). The reader is
  invited to construct definitions of replication that deal with these
  features. 

  Further, the definitions are parameterized in a name, $x$. Can you,
  gentle reader, make a definition that eliminates this parameter and
  guarantees no accidental interaction between the replication
  machinery and the process being replicated -- i.e. no accidental
  sharing of names used by the process to get its work done and the
  name(s) used by the replication to effect copying. This latter
  revision of the definition of replication is crucial to obtaining
  the expected identity $!!P \sim !P$.
\end{remark}

\begin{remark}\label{rem:paradoxical_combinator}
  The reader familiar with the lambda calculus will have noticed the
  similarity between $D$ and the paradoxical combinator.

  [Ed. note: the existence of this seems to suggest we have to be more
  restrictive on the set of processes and names we admit if we are to
  support no-cloning.]
\end{remark}

\subsubsection{Bisimulation}

The computational dynamics gives rise to another kind of equivalence,
the equivalence of computational behavior. As previously mentioned
this is typically captured \emph{via} some form of bisimulation.

% The notion we use in this paper is weak barbed bisimulation
% \cite{milner91polyadicpi}.

The notion we use in this paper is derived from weak barbed
bisimulation \cite{milner91polyadicpi}. 

\begin{definition}
An \emph{observation relation}, $\downarrow_{\mathcal N}$, over a set
of names, $\mathcal N$, is the smallest relation satisfying the rules
below.

\infrule[Out-barb]{y \in {\mathcal N}, \; x \nameeq y}
		  {\outputp{x}{v} \downarrow_{\mathcal N} x}
\infrule[Par-barb]{\mbox{$P\downarrow_{\mathcal N} x$ or $Q\downarrow_{\mathcal N} x$}}
		  {\binpar{P}{Q} \downarrow_{\mathcal N} x}

We write $P \Downarrow_{\mathcal N} x$ if there is $Q$ such that 
$P \wred Q$ and $Q \downarrow_{\mathcal N} x$.
\end{definition}

\begin{definition}
%\label{def.bbisim}
An  ${\mathcal N}$-\emph{barbed bisimulation} over a set of names, ${\mathcal N}$, is a symmetric binary relation 
${\mathcal S}_{\mathcal N}$ between agents such that $P\rel{S}_{\mathcal N}Q$ implies:
\begin{enumerate}
\item If $P \red P'$ then $Q \wred Q'$ and $P'\rel{S}_{\mathcal N} Q'$.
\item If $P\downarrow_{\mathcal N} x$, then $Q\Downarrow_{\mathcal N} x$.
\end{enumerate}
$P$ is ${\mathcal N}$-barbed bisimilar to $Q$, written
$P \wbbisim_{\mathcal N} Q$, if $P \rel{S}_{\mathcal N} Q$ for some ${\mathcal N}$-barbed bisimulation ${\mathcal S}_{\mathcal N}$.
\end{definition}

\subsubsection{Contexts}

One of the principle advantages of computational calculi from the
$\lambda$-calculus to the $\pi$-calculus is a well-defined notion of context,
contextual-equivalence and a correlation between
contextual-equivalence and notions of bisimulation. The notion of
context allows the decomposition of a process into (sub-)process and
its syntactic environment, its context. Thus, a context may be
thought of as a process with a ``hole'' (written $\bigbox$) in it. The
application of a context $K$ to a process $P$, written $K[P]$, is
tantamount to filling the hole in $K$ with $P$. In this paper we do
not need the full weight of this theory, but do make use of the notion
of context in the proof the main theorem. 

\begin{mathpar}
\inferrule* [lab=context] {} {K \bc \bigbox \;\bm\; \mathsf{for}( \arrvec{y} \leftarrow x )K \;\bm\; x\mathsf{!}(\arrvec{P},K,\arrvec{Q}) \;\bm\; K\mathsf{|}P }
\end{mathpar}

\begin{definition}[contextual application] Given a context $K$, and
  process $P$, we define the \emph{contextual application}, $K[P] :=
  K\{P/\bigbox\}$. That is, the contextual application of K to P is the
  substitution of $P$ for $\bigbox$ in $K$.
\end{definition}

\begin{remark}
  Note that we can extend the definition of free and bound names to contexts.
\end{remark}

% subsection the_syntax_and_semantics_of_the_notation_system (end)   
