\section{Multilevel Agency}
The rho calculus provides a number of important perspectives on
multilevel agency. In this section we focus on three of them that can
be reduced to calculi that can be coded up and executed on modern
computer hardware.

\subsection{Annihilation}
Another important variation has to do with the rewrite rules. There is
a nested recursion version of the $\mathsf{COMM}$-rule that aligns
with intuitions about the recursive nature of behavior in
compositionally defined agents. While for your author the maths make
it clearer than the English it is worth setting out some of the
motivations for this variation of the dynamics of the rho
calculus.

The reductionist view of science is that phenomena like medicinal
cures rest on the phenomena of biology. That is, we find explanation
of the mechanism underlying the medical procedure in the biological
phenomena the procedure interacts with in some fashion. Meanwhile, the
phenomena of biology rests, in turn, on the phenomena of chemistry;
and chemistry rests on (nuclear) physics. By way of an example, in the
reductionist narrative the interaction of two blood cells in the veins
of the body of a person is ultimately explained in terms of the
interactions of various chemical compounds, which are in turn
explained in terms of the interaction of various atoms and their
subatomic components, such as electrons, etc.

While this is accepted philosophy and pedagogy, it is quite difficult
to reduce this account to actual computations, except in very limited
or schematic and abstract fashion. For example, getting differential
equations to work effectively across such scales (from cells to
quarks) is simply not practical. The reductionist picture is
comforting and to some extent supported by evidence, but not reducible
to practice except in extremely limited cases.

The variation of the rho calculus dynamics explored here was developed
to explore the power of compositionality to capture this kind of tower
of reduction. Essentially, we develop of notion of process interaction
that in turn relies on the interaction of ``lower level'' processes,
and so on all the way down to a real bottom. The notion gives a
precise definition of what it means to be ``lower level'' and how that
relates to computational dynamics. In some sense, this is the crudest
of pictures of multilevel agency, and yet given the efficacy of the
process calculi to faithfully represent chemical, biochemical, and
biological phenomena, there is some hope that it might not just be a
pleasant abstraction.

First, we define what it means for two processes to
annihilate each other.

\begin{definition}
  Annihilation: Processes $P$ and $Q$ are said to annihilate one another, written $P \bot Q$, just when $\forall R. P \mathsf{|} Q \rightarrow^{*} R \Rightarrow R \rightarrow^{*} \pzero$.
\end{definition}

Thus, when $P \bot Q$, all rewrites out of $P \mathsf{|} Q$ eventually
lead to $\pzero$. Evidently, $P \bot Q \iff Q \bot P$, and $\pzero
\bot \pzero$. Naturally, we can extend annihilation to names: $x \bot
y \iff \procn{x} \bot \procn{y}$.

Annihilation affords a new version of the $\mathsf{COMM}$-rule:

\begin{mathpar}
  \inferrule* [lab=COMM] {x_{t} \;\bot x_{s}, \;\;\; |\arrvec{y}| = |\arrvec{Q}|} {(R_1 + \mathsf{for}( \arrvec{y} \leftarrow x_{t} )P) \;\mathsf{|}\; (x_{s}!(\arrvec{Q}).P' + R_2)
  \red P\substn{\arrvec{\quotep{Q}}}{\arrvec{y}}\mathsf{|}P'}
\end{mathpar}

All annihilation-based reduction happens in terms of reductions that
happen at a lower degree of quotation, and grounds out in the fact
that $\pzero \bot \pzero$ and thus $\quotep{\pzero} \bot \quotep{\pzero}$.

\begin{example}
  For example, let $P_{1} := \prefix{\quotep{\pzero}}{\quotep{\pzero}}{\pzero} \mathsf{|} \outputp{\quotep{\pzero}}{\pzero}$. Then $P_{1} \red \pzero$ because $\pzero \bot
\pzero$. Suppose now that we set $x_{0}^{-}, x_{0}^{+} := \quotep{\pzero}$. Then, we can write $P_{1}$ as
$\prefix{x_{0}^{-}}{x_{0}^{-}}{\pzero} \mathsf{|} \outputp{x_{0}^{+}}{\pzero}$. Now, set

\begin{mathpar}
  x_{1}^{-}:= \quotep{(\prefix{x_{0}^{-}}{x_{0}^{-}}{\pzero})}
  \and
  x_{1}^{+} := \quotep{(\outputp{x_{0}^{+}}{\pzero})}
\end{mathpar}

Then define 
$P_{2} := \prefix{x_{1}^{-}}{x_{1}^{-}}{\pzero} \mathsf{|} \outputp{x_{1}^{+}}{\pzero}$. Then $P_{2} \red \pzero$ because $x_{1}^{-} \bot x_{1}^{+}$,
and hence $\prefix{x_{1}^{-}}{x_{1}^{-}}{\pzero} \bot \outputp{x_{1}^{+}}{\pzero}$.

More generally, set

\begin{mathpar}
  x_{i}^{-} := \quotep{(\prefix{x_{i-1}^{-}}{x_{i-1}^{-}}{\pzero})}
  \and
  x_{i}^{+} := \quotep{\outputp{x_{i-1}^{+}}{\pzero}} \\
  \and P_{i} := \prefix{\quotep{x_{i-1}^{-}}}{x_{i-1}^{-}}{\pzero} \mathsf{|} \outputp{\quotep{x_{i-1}^{+}}}{\pzero}
\end{mathpar}

Then $P_{i} \red \pzero$ and hence $x_{i}^{-} \bot x_{i}^{+}$.
\end{example}
This allows
for a measure of reduction complexity.

\subsection{Procedural reflection}

\begin{mathpar}
\inferrule* [lab=process] {} {P, Q \bc \pzero \;\bm\; \mathsf{for}(
  y \leftarrow x )P \;\bm\; x\mathsf{!}(Q) \;\bm\;
  \mathsf{*}x \;\bm\; \mathsf{?}P \;\bm\; P\mathsf{|}Q } \and \inferrule* [lab=name] {}
            {x, y \bc \mathsf{@}P }
\end{mathpar}

\subsection{Programmable contexts}
\subsubsection{Process grammar}\label{subsub:process_grammar}

\begin{mathpar}
  \inferrule* [lab=process] {} {P, Q \bc \pzero \;\bm\; \mathsf{U}(x) \;\bm\; \mathsf{for}(y \leftarrow x )P \;\bm\; x\mathsf{!}(Q) \;\bm\;
  P\mathsf{|}Q \;\bm\; \mathsf{*}x \;\bm\; \mathsf{COMM}(K) }
  \and
  \inferrule* [lab=name] {} {x, y \bc \mathsf{@}\langle K, P\rangle }
  \and
  \inferrule* [lab=context] {} {K \bc \bigbox \;\bm\;  \mathsf{for}(y \leftarrow x )K \;\bm\; x\mathsf{!}(K) \;\bm\; P\mathsf{|}K}
\end{mathpar}

\begin{definition}
  \emph{Free and bound names} The calculation of the free names of a
  process, $P$, denoted $\freenames{P}$ is given recursively by
  
  \begin{mathpar}
    \freenames{\pzero} = \emptyset
    \and
    \freenames{\mathsf{U}(x)} = \{ x \}
    \and
    \freenames{\mathsf{for}(y \leftarrow x)P} = \{ x \} \cup \freenames{P}\setminus\{y\}
    \and
    \freenames{x!(P)} = \{ x \} \cup \freenames{P}
    \and
    \freenames{P|Q} = \freenames{P} \cup \freenames{Q}
    \and
    \freenames{\mathsf{*}{x}} = \{ x \}
    \and
    \freenames{\mathsf{COMM}(K)} = \freenames{K}
    \and 
    \freenames{\bigbox} = \emptyset
    \and
    \freenames{\mathsf{for}(y \leftarrow x)K} = \{ x \} \cup \freenames{K}\setminus\{y\}
    \and
    \freenames{x!(K)} = \{ x \} \cup \freenames{K}
    \and
    \freenames{P|K} = \freenames{P} \cup \freenames{K}
  \end{mathpar}
  
  An occurrence of $x$ in a process $P$ is \textit{bound} if it is not
  free. The set of names occurring in a process (bound or free) is
  denoted by $\names{P}$.
\end{definition}

\subsection{Operational semantics}

Finally, we introduce the computational dynamics. What marks these
algebras as distinct from other more traditionally studied algebraic
structures, e.g. vector spaces or polynomial rings, is the manner in
which dynamics is captured. In traditional structures, dynamics is typically
expressed through morphisms between such structures, as in linear maps
between vector spaces or morphisms between rings. In algebras
associated with the semantics of computation, the dynamics is
expressed as part of the algebraic structure itself, through a
reduction reduction relation typically denoted by $\red$. Below, we
give a recursive presentation of this relation for the calculus used
in the encoding.

\begin{mathpar}
  \inferrule* [lab=Catalyze] {} {\mathsf{U}(x) \;\mathsf{|}\; \dropn{\quotep{\langle K,Q \rangle}} \red \mathsf{COMM}(K) \;\mathsf{|}\; x\mathsf{!}(Q)} \\
  \and
  \inferrule* [lab=Comm] {x_{t} \;\nameeq\; x_{s}} {\mathsf{COMM}(K) \;\mathsf{|}\; \mathsf{for}( y \leftarrow x_{t} )P \;\mathsf{|}\; x_{s}!(Q)
    \red P\substn{\quotep{\langle K,Q \rangle}}{y}} \\
  \and
  \inferrule* [lab=Par]{P \red P'}{P\mathsf{|}Q \red P'\mathsf{|}Q} \\
  \and
  \inferrule* [lab=Equiv]{{P \;\scong\; P'} \andalso {P' \red Q'} \andalso {Q' \;\scong\; Q}}{P \red Q}
\end{mathpar}

We write $P\red$ if $\exists Q $ such that $ P \red Q$ and $P\not\red$, otherwise.

\subsection{ Movement in space: an example }
In the following example let $x = \spacen{\bigbox}{\pzero}$ and $y = \spacen{K}{Q}$ for some $K$ and $Q$.
\begin{eqnarray*}
  & \binpar{\binpar{\prefix{x}{y}{\dropn{y}}}{\outputp{x}{P}}}{\mathsf{COMM}(K_{1})} & \\
  \red & & \\
  & \dropn{y}\substn{\spacen{K_{1}}{P}}{y} & \\
  = & & \\
  & K_{1}[P] &
\end{eqnarray*}

Now, if $K_{1}$ is also of the form $\binpar{\binpar{\binpar{\prefix{x'}{y'}{\dropn{y'}}}{\outputp{x'}{\bigbox}}}{\mathsf{COMM}(K_{2})}}{R}$, then $P$ will move to the location $\binpar{K_{2}[P]}{R}$. That is, $K_{1}[P] \red \binpar{K_{2}[P]}{R}$. And if $K_{2}$ is likewise of the form $\binpar{\binpar{\binpar{\prefix{x''}{y''}{\dropn{y''}}}{\outputp{x''}{\bigbox}}}{\mathsf{COMM}(K_{3})}}{R'}$, then $P$ will move to the location $\binpar{K_{3}[P]}{\binpar{R}{R'}}$.

Thus, we have a means to describe movement of a process from location to location.
