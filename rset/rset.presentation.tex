\section{Formal theory}
The judgment \(\Sigma; \Gamma \vdash S\) is pronounced
``\(\Sigma\) thinks that \(S\) is well-formed, given
dependencies, \(\Gamma\) .'' Similarly,
\(\Sigma; \Gamma \vdash x\) is pronounced
``\(\Sigma\) thinks that \(x\) is an admissible atom, given
dependencies, \(\Gamma\).''

We think of $X$ as some infinite supply of ``atoms'' and
$\mathsf{S}[X]$ as a \emph{stream} of atoms drawn from $X$. We assume
basic stream operations on $\mathsf{S}[X]$, such as
$\mathsf{take}(\mathsf{S}[X],n)$ which returns the pair
$(\sigma,\Sigma)$ where $\sigma = x_{1}:\ldots:x_{n}$ and $\Sigma = \mathsf{drop}(\mathsf{S}[X],n)$. Given $\rho = \{j_{1}/i_{1}:\ldots:j_{n}/i_{n}\}$ the operation
$\mathsf{swap}(\mathsf{S}[X],\rho)$ denotes the application of the
permutation $\rho$ to $\mathsf{S}[X]$.

Thus, $\mathsf{S}[X]$ may be thought of as an ordering on $X$.

A rule of the form

\[\frac{ H_1, \ldots , H_n }{ C }R\]

is pronounced ``\(R\) concludes that \(C\) given \(H_1\), \(\ldots\),
\(H_n\)''.

We use $e,f,\ldots$ to range over sets and atoms.

\subsection{Embracing content}
The rules for judging when an atom is admissible or set is well
formed are as follows

\[\frac{ }{ \Sigma; () \vdash \emptyset}Foundation\]

\[\frac{ (x,\Sigma') = \mathsf{take}(\Sigma,1) \;\;\; \Sigma; \Gamma \vdash S}{ \Sigma'; x,\Gamma \vdash \{ x \} \cup S}Atomicity\]

\[\frac{ \Sigma; \Gamma \vdash S }{ \Sigma; \Gamma \vdash \{ S \}}Nesting\]

\[\frac{ \Sigma; \Gamma_1 \vdash S_1 \;\;\; \Sigma; \Gamma_2 \vdash S_2}{ \Sigma; \Gamma_1, \Gamma_2 \vdash S_1 \cup S_2}Union\]

\[\frac{ \Sigma; \Gamma_1 \vdash S_1 \;\;\; \Sigma; \Gamma_2 \vdash S_2}{ \Sigma; \Gamma_1, \Gamma_2 \vdash S_1 \cap S_2}Intersection\]

\[\frac{ \Sigma; \Gamma_1 \vdash S_1 \;\;\; \Sigma; \Gamma_2 \vdash S_2}{ \Sigma; \Gamma_1, \Gamma_2 \vdash S_1 \setminus S_2}Subtraction\]

\subsection{Recognizing elements}
\[\frac{ \Sigma; \Gamma \vdash e }{ \Sigma; \Gamma \vdash e \in \{ e \}}Presence\]
\[\frac{ \Sigma; \Gamma \vdash e }{ \Sigma; \Gamma \vdash e \notin \emptyset}Emptiness\]

\[\frac{ \Sigma; \Gamma \vdash S_{1} \;\;\; \Sigma; \Gamma \vdash S_{2} \;\;\; \Sigma; \Gamma \vdash e \in S_{1}}{ \Sigma; \Gamma \vdash e \in S_{1} \cup S_{2}}Inclusion\]

\[\frac{ \Sigma; \Gamma \vdash S_{1} \;\;\; \Sigma; \Gamma \vdash S_{2} \;\;\; \Sigma; \Gamma \vdash e \in S_{1} \;\;\; \Sigma; \Gamma \vdash e \in S_{2}}{ \Sigma; \Gamma \vdash e \in S_{1} \cap S_{2}}Collusion\]

\[\frac{ \Sigma; \Gamma \vdash S_{1} \;\;\; \Sigma; \Gamma \vdash S_{2} \;\;\; \Sigma; \Gamma \vdash e \in S_{1} \;\;\; \Sigma; \Gamma \vdash e \notin S_{2}}{ \Sigma; \Gamma \vdash e \in S_{1} \setminus S_{2}}Exclusion\]

\subsection{Equations}\label{equations}

The syntactic theory is too fine grained. It makes syntactic
distinctions that do not correspond to distinct sets. We erase these
syntactic distinctions with a set of equations on set expressions.

\[\frac{ \Sigma; \Gamma \vdash e }{ \Sigma; \Gamma  \vdash e = e}Identity\]
\[\frac{\Sigma; \Gamma_{1} \vdash _{1}\; \Sigma; \Gamma_{2} \vdash e_{2}\; \Sigma; \Gamma_{3}  \vdash e_{1} = e_{2}}{ \Sigma; \Gamma_{3}  \vdash e_{1} = e_{2}}Symmetry\]
\[\frac{\Sigma; \Gamma_{1}  \vdash e_{1}\; \Sigma; \Gamma_{2}  \vdash e_{2}\; \Sigma; \Gamma_{3} \vdash e_{3}\; \Sigma; \Gamma_{4}  \vdash e_{1} = e_{2} \; \Sigma; \Gamma_{5}  \vdash e_{2} = e_{3}}{ \Sigma; \Gamma_{4},\Gamma_{5}  \vdash e_{1} = e_{3}}Transitivity\]

\[\frac{\Sigma; \Gamma  \vdash S}{\Sigma; \Gamma \vdash S \cup S =  S}Idempotence_{\cup}\]
\[\frac{\Sigma; \Gamma  \vdash S}{\Sigma; \Gamma \vdash S \cap S =  S}Idempotence_{\cap}\]

\subsection{Tying the first recursive knot}
\[\frac{ \Sigma; \Gamma \vdash e_{1} \;\;\; \Sigma; \Gamma \vdash e_{2} \;\;\; \Sigma; \Gamma \vdash e_{1} \neq e_{2} }{ \Sigma; \Gamma \vdash e_{2} \notin \{ e_{1}\}}Absence\]
\[\frac{ \Sigma; \Gamma \vdash S \;\;\; \Sigma; \Gamma_{1} \vdash S_{1} \;\;\;\ldots\;\;\; \Sigma; \Gamma_{n} \vdash S_{2}\; \Sigma; \Delta \vdash f : S_{1} \times \ldots \times S_{n} \to S }{ \Sigma; \Gamma,\Gamma_{1},\ldots, \Gamma_{n},\Delta \vdash \{ f(s_{1},\ldots,s_{n}) \; : \; s_{1} \in S_{1},\; \ldots\;, s_{n} \in S_{n}\}}Comprehension\]

\subsection{Set's signature}
The signature of the theory is given by
\begin{mathpar}
  \emptyset, |\emptyset| = 0
  \and
  \mathsf{embrace}_{i}, |\mathsf{embrace}_{i}| = i
  \and
  \cap, |\cap| = 2
  \and
  \cup, |\cup| = 2
  \and
  \setminus, |\setminus| = 2
\end{mathpar}

Which we write as $\mathsf{Set}[X]$.

\section{Tying the second recursive knot}
Now for the main course. We take two copies of the theory, that is two
versions of the signature and the theories generated by each version. For simplicity, we write \textcolor{red}{$\mathsf{Set}[X]$} and $\mathsf{Set}[X]$. Then we set \textcolor{red}{$X$}$=\mathsf{Set}[X]$ and $X=$\textcolor{red}{$\mathsf{Set}[X]$}.




