\section{The calculus from outer space}
\subsection{Syntax and semantics}\label{sub:the_syntax_and_semantics_of_the_notation_system} % (fold)

We now give a technical presentation of the calculus. The typical
presentation of such a calculus follows the style of giving generators
and relations on them. The grammar, below, describing term
constructors, freely generates the set of processes, $\Proc$. This set
is then quotiented by a relation known as structural congruence and it
is over this set that the notion of dynamics is expressed.

\subsubsection{Process grammar}\label{subsub:process_grammar}

\begin{mathpar}
  \inferrule* [lab=process] {} {P, Q \bc \pzero \;\bm\; \mathsf{U}(x) \;\bm\; \mathsf{for}(y \leftarrow x )P \;\bm\; x\mathsf{!}(Q) \;\bm\;
  P\mathsf{|}Q \;\bm\; \mathsf{*}x \;\bm\; \mathsf{COMM}(K) }
  \and
  \inferrule* [lab=name] {} {x, y \bc \mathsf{@}\langle K, P\rangle }
  \and
  \inferrule* [lab=context] {} {K \bc \bigbox \;\bm\;  \mathsf{for}(y \leftarrow x )K \;\bm\; x\mathsf{!}(K) \;\bm\; P\mathsf{|}K}
\end{mathpar}

\begin{definition}
  \emph{Free and bound names} The calculation of the free names of a
  process, $P$, denoted $\freenames{P}$ is given recursively by
  
  \begin{mathpar}
    \freenames{\pzero} = \emptyset
    \and
    \freenames{\mathsf{U}(x)} = \{ x \}
    \and
    \freenames{\mathsf{for}(y \leftarrow x)P} = \{ x \} \cup \freenames{P}\setminus\{y\}
    \and
    \freenames{x!(P)} = \{ x \} \cup \freenames{P}
    \and
    \freenames{P|Q} = \freenames{P} \cup \freenames{Q}
    \and
    \freenames{\mathsf{*}{x}} = \{ x \}
    \and
    \freenames{\mathsf{COMM}(K)} = \freenames{K}
    \and 
    \freenames{\bigbox} = \emptyset
    \and
    \freenames{\mathsf{for}(y \leftarrow x)K} = \{ x \} \cup \freenames{K}\setminus\{y\}
    \and
    \freenames{x!(K)} = \{ x \} \cup \freenames{K}
    \and
    \freenames{P|K} = \freenames{P} \cup \freenames{K}
  \end{mathpar}
  
  An occurrence of $x$ in a process $P$ is \textit{bound} if it is not
  free. The set of names occurring in a process (bound or free) is
  denoted by $\names{P}$.
\end{definition}

\subsection{Substitution}

We use $\Proc$ for the set of processes, $\QProc$ for the set of
names, and $\id{\{}\arrvec{y} / \arrvec{x} \id{\}}$ to denote partial
maps, $s : \QProc \rightarrow \QProc$. A map, $s$ lifts, uniquely, to
a map on process terms, $\widehat{s} : \Proc \rightarrow
\Proc$. Historically, it is convention to use $\sigma$ to range over
lifted subsitutions, $\widehat{s}$, to write the application of a
substitution, $\sigma$ to a process, $P$, with the substitution on the
right, $P\sigma$, and the application of a substitution, $s$, to a
name, $x$, using standard function application notation, $s(x)$. In
this instance we choose not to swim against the tides of
history. Thus, 

\begin{definition}
  given $x = \quotep{P'}$, $u = \quotep{\langle K, Q' \rangle}$, $s = \substn{u}{x}$ we define the lifting of $s$ to $\widehat{s}$ (written
  below as $\sigma$) recursively by the following equations.
  \begin{mathpar}
    0 \sigma := 0 \\
    (P \mathsf{|} Q) \sigma
    :=    
    P\sigma \mathsf{|} Q\sigma \\
    (\mathsf{for}(y \leftarrow v)P) \sigma    
    :=
    \mathsf{for}(z \leftarrow \sigma(v))((P \psubstn{z}{y}) \sigma) \\
    (\lift{x}{Q}) \sigma  
    :=
    \lift{\sigma(x)}{ Q \sigma } \\
    (\dropn{y})  \sigma       
    := 
    \left\{ 
      \begin{array}{ccc} 
        K[Q'] & & y \;\nameeq\; x \\
        \dropn{y} & & otherwise \\
      \end{array}
      \right.
  \end{mathpar} 

  where

  \begin{eqnarray}
    \psubstn{w}{v}(x) = \substn{w}{v}(x) = 
    \left\{ 
      \begin{array}{ccc}
        w & & x \;\nameeq\; v \\
        x & & otherwise \\
      \end{array}
      \right. \nonumber
  \end{eqnarray}
\end{definition}

and $z$ is fresh for $P$. Our $\alpha$-equivalence will be built in
the standard way from this substitution.

\begin{definition}
Then two processes, $P,Q$, are alpha-equivalent if $P = Q\{\arrvec{y}/\arrvec{x}\}$ for
some $\arrvec{x} \subseteq \boundnames{Q},\arrvec{y} \subseteq \boundnames{P}$, where $Q\{\arrvec{y}/\arrvec{x}\}$
denotes the capture-avoiding substitution of $\arrvec{y}$ for $\arrvec{x}$ in $Q$.
\end{definition}

\begin{definition}
  The {\em structural congruence} $\equiv$
  between processes \cite{DBLP:books/daglib/0004377} is the least congruence containing
  alpha-equivalence and satisfying the commutative monoid laws
  (associativity, commutativity and $\pzero$ as identity) for parallel
  composition $|$.
\end{definition}

\begin{definition}
  The {\em name equivalence} $\nameeq$ is the least congruence
  satisfying these equations
  \begin{mathpar}
  \inferrule*[lab=Quote-drop] {}{ \quotep{\dropn{x}} \;\nameeq\; x }
  \and
  \inferrule*[lab=Struct-equiv] { P \;\scong\; Q } { \quotep{P} \;\nameeq\; \quotep{Q} }
  \end{mathpar}
\end{definition}

The astute reader will have noticed that the mutual recursion of names
and processes imposes a mutual recursion on alpha-equivalence and
structural equivalence via name-equivalence. Fortunately, all of this
works out pleasantly and we may calculate in the natural way, free of
concern. The reader interested in the details is referred to
\cite{DBLP:journals/entcs/MeredithR05}.

\begin{remark}\label{rem:no_self_referential_names}
  One particularly useful consequence of these definitions is that
  $\forall K,P. \quotep{\langle K,P\rangle} \not\in \freenames{P}$. It gives us a
  succinct way to construct a name that is distinct from all the names
  in $P$ and hence fresh in the context of $P$. For those readers
  familiar with the work of Pitts and Gabbay, this consequence allows
  the system to completely obviate the need for a fresh operator, and
  likewise provides a canonical approach to the semantics of
  freshness.
\end{remark}

\subsection{Operational semantics}

Finally, we introduce the computational dynamics. What marks these
algebras as distinct from other more traditionally studied algebraic
structures, e.g. vector spaces or polynomial rings, is the manner in
which dynamics is captured. In traditional structures, dynamics is typically
expressed through morphisms between such structures, as in linear maps
between vector spaces or morphisms between rings. In algebras
associated with the semantics of computation, the dynamics is
expressed as part of the algebraic structure itself, through a
reduction reduction relation typically denoted by $\red$. Below, we
give a recursive presentation of this relation for the calculus used
in the encoding.

\begin{mathpar}
  \inferrule* [lab=Catalyze] {} {\mathsf{U}(x) \;\mathsf{|}\; \dropn{\quotep{\langle K,Q \rangle}} \red \mathsf{COMM}(K) \;\mathsf{|}\; x\mathsf{!}(Q)} \\
  \and
  \inferrule* [lab=Comm] {x_{t} \;\nameeq\; x_{s}} {\mathsf{COMM}(K) \;\mathsf{|}\; \mathsf{for}( y \leftarrow x_{t} )P \;\mathsf{|}\; x_{s}!(Q)
    \red P\substn{\quotep{\langle K,Q \rangle}}{y}} \\
  \and
  \inferrule* [lab=Par]{P \red P'}{P\mathsf{|}Q \red P'\mathsf{|}Q} \\
  \and
  \inferrule* [lab=Equiv]{{P \;\scong\; P'} \andalso {P' \red Q'} \andalso {Q' \;\scong\; Q}}{P \red Q}
\end{mathpar}

We write $P\red$ if $\exists Q $ such that $ P \red Q$ and $P\not\red$, otherwise.

\subsection{ Dynamic quote: an example }
TBD

%% Anticipating something of what's to come, let $z = \quotep{P}$, $u = \quotep{Q}$, and $x = \quotep{\lift{y}{\dropn{z}}}$. Now consider applying the substitution,
%% $\widehat{\id{\{}u / z \id{\}}}$, to the following pair of processes,
%% $\lift{w}{y!(\dropn{z})}$ and $\lift{w}{\dropn{x}} = \lift{w}{\dropn{\quotep{\lift{y}{\dropn{z}}}}}$.

%% \begin{eqnarray}
%% 	\lift{w}{\lift{y}{\dropn{z}}}\widehat{\id{\{}u / z \id{\}}}
%% 		& = &
%% 		\lift{w}{\lift{y}{Q}} \nonumber\\
%% 	\lift{w}{\dropn{x}} \widehat{ \id{\{}u / z \id{\}} }
%% 		& = &
%% 		\lift{w}{\dropn{x}} \nonumber
%% \end{eqnarray}

%% The body of the quoted process, $\quotep{\lift{y}{\dropn{z}}}$, is
%% impervious to substitution, thus we get radically different
%% answers. In fact, by examining the first process in an input context,
%% e.g. $\mathsf{for}(z \leftarrow x)\lift{w}{\lift{y}{\dropn{z}}}$, we see that the process
%% under the output operator may be shaped by prefixed inputs binding a
%% name inside it. In this sense, the combination of input prefix binding
%% and output operators will be seen as a way to dynamically construct
%% processes before reifying them as names.

\section{Replication}
TBD

%% As mentioned before, it is known that replication (and hence
%% recursion) can be implemented in a higher-order process algebra
%% \cite{DBLP:books/daglib/0004377}. As our first example of calculation with the
%% machinery thus far presented we give the construction explicitly in
%% the {\rhoc}.

%% \begin{eqnarray}
%% 	D_{x} & := & \prefix{x}{y}{(\binpar{\outputp{x}{y}}{\dropn{y}})} \nonumber\\
%% 	\bangp_{x}{P} & := & \binpar{\lift{x}{\binpar{D_{x}}{P}}}{D_{x}} \nonumber
%% \end{eqnarray}

%% \begin{eqnarray}
%% 	\bangp_{x}{P} & & \nonumber\\
%% 	=
%% 	& \lift{x}{(\prefix{x}{y}{(\outputp{x}{y} | \dropn{y})) | P}} 
%% 	      | \prefix{x}{y}{(\outputp{x}{y} | \dropn{y})} & \nonumber\\
%% 	\red
%% 	& (\outputp{x}{y} | \dropn{y})\substn{\quotep{(\prefix{x}{y}{(\dropn{y} | \outputp{x}{y})) | P}}}{y} & \nonumber\\
%% 	=
%% 	& \outputp{x}{\quotep{(\prefix{x}{y}{(\outputp{x}{y} | \dropn{y})) | P}}}
%% 	  | {(\prefix{x}{y}{(\outputp{x}{y} | \dropn{y})) | P}} & \nonumber\\
%% 	\red
%% 	& \ldots & \nonumber\\
%% 	\red^*
%% 	& P | P | \ldots & \nonumber
%% \end{eqnarray}

%% Of course, this encoding, as an implementation, runs away, unfolding
%% $\bangp{P}$ eagerly. A lazier and more implementable replication
%% operator, restricted to input-guarded processes, may be obtained as follows.

%% \begin{eqnarray}
%% \bangp{\prefix{u}{v}{P}} 
%% 	:= 
%% 	\binpar{\lift{x}{\prefix{u}{v}{(\binpar{D(x)}{P})}}}{D(x)} \nonumber
%% \end{eqnarray}

%% \begin{remark}
%%   Note that the lazier definition still does not deal with summation
%%   or mixed summation (i.e. sums over input and output). The reader is
%%   invited to construct definitions of replication that deal with these
%%   features. 

%%   Further, the definitions are parameterized in a name, $x$. Can you,
%%   gentle reader, make a definition that eliminates this parameter and
%%   guarantees no accidental interaction between the replication
%%   machinery and the process being replicated -- i.e. no accidental
%%   sharing of names used by the process to get its work done and the
%%   name(s) used by the replication to effect copying. This latter
%%   revision of the definition of replication is crucial to obtaining
%%   the expected identity $!!P \sim !P$.
%% \end{remark}

%% \begin{remark}\label{rem:paradoxical_combinator}
%%   The reader familiar with the lambda calculus will have noticed the
%%   similarity between $D$ and the paradoxical combinator.

%%   [Ed. note: the existence of this seems to suggest we have to be more
%%   restrictive on the set of processes and names we admit if we are to
%%   support no-cloning.]
%% \end{remark}

\subsubsection{Bisimulation}

The computational dynamics gives rise to another kind of equivalence,
the equivalence of computational behavior. As previously mentioned
this is typically captured \emph{via} some form of bisimulation.

% The notion we use in this paper is weak barbed bisimulation
% \cite{milner91polyadicpi}.

The notion we use in this paper is derived from weak barbed
bisimulation \cite{milner91polyadicpi}. 

\begin{definition}
An \emph{observation relation}, $\downarrow_{\mathcal N}$, over a set
of names, $\mathcal N$, is the smallest relation satisfying the rules
below.

\infrule[Out-barb]{y \in {\mathcal N}, \; x \nameeq y}
		  {\outputp{x}{v} \downarrow_{\mathcal N} x}
\infrule[Par-barb]{\mbox{$P\downarrow_{\mathcal N} x$ or $Q\downarrow_{\mathcal N} x$}}
		  {\binpar{P}{Q} \downarrow_{\mathcal N} x}

We write $P \Downarrow_{\mathcal N} x$ if there is $Q$ such that 
$P \wred Q$ and $Q \downarrow_{\mathcal N} x$.
\end{definition}

\begin{definition}
%\label{def.bbisim}
An  ${\mathcal N}$-\emph{barbed bisimulation} over a set of names, ${\mathcal N}$, is a symmetric binary relation 
${\mathcal S}_{\mathcal N}$ between agents such that $P\rel{S}_{\mathcal N}Q$ implies:
\begin{enumerate}
\item If $P \red P'$ then $Q \wred Q'$ and $P'\rel{S}_{\mathcal N} Q'$.
\item If $P\downarrow_{\mathcal N} x$, then $Q\Downarrow_{\mathcal N} x$.
\end{enumerate}
$P$ is ${\mathcal N}$-barbed bisimilar to $Q$, written
$P \wbbisim_{\mathcal N} Q$, if $P \rel{S}_{\mathcal N} Q$ for some ${\mathcal N}$-barbed bisimulation ${\mathcal S}_{\mathcal N}$.
\end{definition}

%% \subsubsection{Contexts}

%% One of the principle advantages of computational calculi from the
%% $\lambda$-calculus to the $\pi$-calculus is a well-defined notion of context,
%% contextual-equivalence and a correlation between
%% contextual-equivalence and notions of bisimulation. The notion of
%% context allows the decomposition of a process into (sub-)process and
%% its syntactic environment, its context. Thus, a context may be
%% thought of as a process with a ``hole'' (written $\Box$) in it. The
%% application of a context $K$ to a process $P$, written $K[P]$, is
%% tantamount to filling the hole in $K$ with $P$. In this paper we do
%% not need the full weight of this theory, but do make use of the notion
%% of context in the proof the main theorem. 

%% \begin{mathpar}
%% \inferrule* [lab=context] {} {K \bc \Box \;\bm\; \mathsf{for}( \arrvec{y} \leftarrow x )K \;\bm\; x\mathsf{!}(\arrvec{P},K,\arrvec{Q}) \;\bm\; K\mathsf{|}P }
%% \end{mathpar}

%% \begin{definition}[contextual application] Given a context $K$, and
%%   process $P$, we define the \emph{contextual application}, $K[P] :=
%%   K\{P/\Box\}$. That is, the contextual application of K to P is the
%%   substitution of $P$ for $\Box$ in $K$.
%% \end{definition}

%% \begin{remark}
%%   Note that we can extend the definition of free and bound names to contexts.
%% \end{remark}

%% \subsection{Lifted types} 

%% In this section, for clarity, we illustrate the procedure for lifting
%% the rules of a specific rewrite system by way of
%% example. Specifically, we illustrate the procedure lifting the
%% rho-calculus theory to the $\mathsf{NT}(\mathsf{CCC})$ level.

%% \subsubsection{Lifted term constuctors}
%% Note that because of the lifted-singleton rule, it is not necessary to
%% give a typing for the $\pzero$ term.

%% \begin{mathpar}
%%   \inferrule* [lab=For-Comprehension] {}{y : \mathbf{\quotep{W}} : \mathbf{N}, \Gamma \vdash P : \mathbf{U} : \mathbf{P} \;\;\; \Delta \vdash x : \mathbf{V} : \mathbf{N} \Vdash \Gamma, \Delta \vdash \prefix{x}{y}{P} : \prefixt{V}{\quotep{W}}{U} : \mathbf{P}} \\
%%   \and
%%   \inferrule* [lab=Output] {}{\Gamma \vdash x : \mathbf{U} : \mathbf{N} \;\;\; \Delta \vdash Q : \mathbf{V} : \mathbf{P} \Vdash \Gamma, \Delta \vdash \outputp{x}{Q} : \outputt{U}{V} : \mathbf{P}} \\
%%   \and
%%   \inferrule* [lab=Parallel] {}{\Gamma \vdash P : \mathbf{U} : \mathbf{N} \;\;\; \Delta \vdash Q : \mathbf{V} : \mathbf{P} \Vdash \Gamma, \Delta \vdash {P}\mathsf{|}{Q} : \mathbf{U} \mathbf{|} \mathbf{V} : \mathbf{P}} \\
%%   \and
%%   \inferrule* [lab=Deref] {}{\Gamma \vdash x : \mathbf{U} : \mathbf{N} \Vdash \Gamma \vdash \dropn{x} : \dropt{U} : \mathbf{P}} \\
%% \inferrule* [lab=Quote] {}{\Gamma \vdash P : \mathbf{U} : \mathbf{P} \Vdash \Gamma \vdash \quotep{x} : \quotet{U} : \mathbf{P}} \\
%% \end{mathpar}

%% \subsubsection{Equations}
%% \begin{mathpar}
%%   \inferrule* [lab=ParMonoidId] {\Gamma \vdash P : \mathbf{U} : \mathbf{P}}{\Gamma \vdash \binpar{P}{0} = P : \mathbf{U} : \mathbf{P}} \\
%%   \and
%%   \inferrule* [lab=ParMonoidAssoc] {\Gamma \vdash \binpar{(\binpar{P}{Q})}{R} : \mathbf{U} : \mathbf{P}}{\Gamma \vdash \binpar{(\binpar{P}{Q})}{R} = \binpar{P}{(\binpar{Q}{R})} : \mathbf{U} : \mathbf{P}} \\
%%   \and
%%   \inferrule* [lab=ParMonoidComm] {\Gamma \vdash \binpar{P}{Q} : \mathbf{U} : \mathbf{P}}{\Gamma \vdash \binpar{P}{Q} = \binpar{Q}{P} : \mathbf{U} : \mathbf{P}}
%% \end{mathpar}

%% \subsubsection{Lifted redex constuctors}
%% \begin{mathpar}
%%   \inferrule* [lab=Comm] {\Gamma_{1} \vdash x : \mathbf{V} : \mathbf{N} \;\;\; y : \mathbf{\quotep{V}} : \mathbf{N}, \Gamma_{2} \vdash P : \mathbf{W} : \mathbf{P} \;\;\; \Gamma_{3} \vdash Q : \mathbf{V} : \mathbf{P}}{\Gamma_{1}, \Gamma_{2}, \Gamma_{3} \vdash \commr{x}{y}{P}{Q} : \commt{U}{V}{W} : \mathbf{R}} \\
%%   \and  
%%   \inferrule* [lab=Eval] {}{\Gamma \vdash P : \mathbf{U} : \mathbf{P} \Vdash \Gamma \vdash \mathsf{eval}(P) : \mathbf{eval}(\mathbf{U}) : \mathbf{R}} \\
%%   \and
%%   \inferrule* [lab=ParL] {}{\Gamma \vdash R : \mathbf{E} : \mathbf{R} \;\;\; \Delta \vdash P : \mathbf{U} : \mathbf{P} \Vdash \Gamma, \Delta \vdash \mathsf{par}_{L}(R, P) : \binpartl{E}{U} : \mathbf{R}} \\
%%   \and
%%   \inferrule* [lab=ParR] {}{\Gamma \vdash R : \mathbf{E} : \mathbf{R} \;\;\; \Delta \vdash P : \mathbf{U} : \mathbf{P} \Vdash \Gamma, \Delta \vdash \mathsf{par}_{R}(P, R) : \binpartr{U}{E} : \mathbf{R}}
%% \end{mathpar}

%% \subsubsection{Reductions from the $\lambda$-theory}
%% \begin{mathpar}
%%   \inferrule* [lab=Comm-Src] {\Gamma \vdash \commr{x}{y}{P}{Q} : \commt{U}{V}{W} : \mathbf{R}}{\Gamma \vdash \mathsf{src}(\commr{x}{y}{P}{Q}) = \prefix{x}{y}{P} \mathsf{|} \outputp{x}{Q} : \prefixt{U}{\quotep{V}}{W} \mathbf{|} \outputt{U}{V} : \mathbf{P}} \\
%%   \and
%%   \inferrule* [lab=Comm-Trgt] {\Gamma \vdash \commr{x}{y}{P}{Q} : \commt{U}{V}{W} : \mathbf{R}}{\Gamma \vdash \mathsf{trgt}(\commr{x}{y}{P}{Q}) = P\substn{\quotep{Q}}{y} : \mathbf{W} : \mathbf{P}} \\
%%   \and
%%   \inferrule* [lab=Eval-Src] {}{\Gamma \vdash \mathsf{eval}(P) : \mathbf{eval}(\mathbf{U}) : \mathbf{R} \Vdash \Gamma \vdash \mathsf{src}(\mathsf{eval}(P)) = \dropn{\quotep{P}} : \dropt{\quotet{U}} : \mathbf{P}} \\
%%   \and
%%   \inferrule* [lab=Eval-Trgt] {}{\Gamma \vdash \mathsf{eval}(P) : \mathbf{eval}(\mathbf{U}) : \mathbf{R} \Vdash \Gamma \vdash \mathsf{trgt}(\mathsf{eval}(P)) = P : \mathbf{U} : \mathbf{P}} \\
%%   \inferrule* [lab=Par-Src] {\Gamma \vdash \binparx{R}{P} : \binpart{E}{U} : \mathbf{R}}{\Gamma \vdash \mathsf{src}(\binparx{R}{P}) = \mathsf{src}(R) : \binpart{src(E)}{U} : \mathbf{R}} \\
%%   \and
%%   \inferrule* [lab=Par-Trgt] {\Gamma \vdash \binparx{R}{P} : \binpart{E}{U} : \mathbf{R}}{\Gamma \vdash \mathsf{trgt}(\binparx{R}{P}) = \binparx{\mathsf{trgt}(R)}{P} : \binpart{trgt(E)}{U} : \mathbf{R}} \\
%% \end{mathpar}

% section rholang (end)
