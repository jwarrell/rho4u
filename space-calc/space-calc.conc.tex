\section{Conclusions and future work}

We have presented a calculus enjoying a notion of space derived from a
reconciliation of Huet's zipper and Milner's channels. We have shown
this suffices to encode calculi like Cardelli and Gordon's ambient
calculus.

One of the interesting directions of investigation is to integrate
these ideas with the quantum interpretation of the rho-calculus, using
a 2-norm probability distribution as the source of non-determinism. This brings together in a single, compact calculus both space-like phenomena and quantum non-determinism.
%% We have presented an algorithm for generating a spatial-behavioral
%% type system for a model of computation.

%% One of the most important discoveries of process calculi -- as models
%% of computation -- is that they can abstractly describe
%% non-deterministic behavior and leave to implementation the
%% \emph{source of that non-determinism}. Thus, for the rho-calculus
%% non-determism can be actual races derived from message arrival order
%% non-determinism over physically implemented protocols, or it can be
%% from a 1-norm probability distribution, or as recently discovered, it
%% can be from a 2-norm probability distribution.

%% These last two cases are important as they extend the models to
%% simulating and potentially programming physical substrates including
%% chemical and biological computation as well as quantum computation. In
%% the same way that Hennessey-Milner style logics have been extended to
%% the stochastic and quantum settings, we believe that this algorithm
%% can be extended to cover these cases.

% subsection other_calculi_other_bisimulations_and_geometry_as_behavior (end)


