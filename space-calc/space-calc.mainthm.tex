\section{Main theorem}
There are calculi that emphasize space. In particular, Cardelli and
Gordon proposed the ambient calculus \cite{DBLP:journals/entcs/CardelliG97}. It is strikingly different from
the $\pi$ and rho-calculus. Specifically, the engine of computation in
Milner's and Meredith's calculi is substitution. This is a local
transformation of a single term. The pure ambient calculus evolution
does not use substitution, but whole term tree rewrite. This is
astonishingly expressive. In fact, Cardelli has often pointed out that
there is fully abstract encoding of the ambient calculus into the
$pi$-calculus. Beyond the ambient calculus, Cardelli has proposed his brane calculi \cite{DBLP:conf/cmsb/Cardelli04}.

We don't address this more recent development. Instead, we demonstrate
a fully abstract encoding of the ambient calculus into the space
calculus showing a substitution-based calculus can enjoy much of the same characteristics of a whole tree term rewrite calculus.

\subsection{Review of the ambient calculus}

\subsubsection{Syntax}
\begin{mathpar}
  \inferrule* [lab=process] {} {P, Q \bc \pzero \;\bm\; n[P] \;\bm\; M\mathsf{.}(P) \;\bm\; (\mathsf{new}\; n)P \;\bm\; \mathsf{!}P \;\bm\; P\mathsf{|}Q }
  \and
  \inferrule* [lab=action] {} {m \bc \mathsf{in}\; n \;\bm\; \mathsf{out}\; n \;\bm\; \mathsf{open}\; n }
\end{mathpar}

\subsubsection{Free and bound names}
\begin{definition}
  \emph{Free and bound names} The calculation of the free names of a
  process, $P$, denoted $\freenames{P}$ is given recursively by
  
  \begin{mathpar}
    \freenames{\pzero} = \emptyset
    \and
    \freenames{n[P]} = \{ n \} \cup \freenames{P} \\
    \and
    \freenames{M\mathsf{.}P} = \freenames{M} \cup \freenames{P}
    \and
    \freenames{(\mathsf{new}\; n)P} = \freenames{P} \setminus \{ n \}
    \and
    \freenames{\binpar{P}{Q}} = \freenames{P} \cup \freenames{Q}
    \and
    \freenames{\mathsf{!}P} =  \freenames{P} \\
    \and
    \freenames{\mathsf{in}\; n} = \{ n \}
    \and 
    \freenames{\mathsf{out}\; n} = \{ n \}
    \and
    \freenames{\mathsf{open}\; n} = \{ n \}
  \end{mathpar}
  
  An occurrence of $x$ in a process $P$ is \textit{bound} if it is not
  free. The set of names occurring in a process (bound or free) is
  denoted by $\names{P}$.
\end{definition}

\subsubsection{Structural equivalence}
\begin{mathpar}
  P = P \\
  P = Q \Rightarrow Q = P \\
  P = Q \Rightarrow (\mathsf{new}\; n)P = (\mathsf{new}\; n)Q \\
  P = Q \Rightarrow \mathsf{!}P = \mathsf{!}Q \\
  P = Q \Rightarrow n[P] = n[Q] \\
  P = Q \Rightarrow M\mathsf{.}P = M\mathsf{.}Q \\
  \binpar{P}{Q} = \binpar{Q}{P} \\
  \binpar{P}{(\binpar{Q}{R})} = \binpar{(\binpar{P}{Q})}{R} \\
  \mathsf{!}P = \binpar{P}{\mathsf{!}P} \\
  (\mathsf{new}\; m)(\mathsf{new}\; n)P = (\mathsf{new}\; n)(\mathsf{new}\; m)P \\
  \binpar{((\mathsf{new}\; n)P)}{Q} = (\mathsf{new}\; n)(\binpar{P}{Q}) \; if \; n \notin \freenames{Q} \\
  (\mathsf{new}\; n)(m[P]) = m[(\mathsf{new}\; n)P] \; if \; m \neq n \\
  \binpar{P}{\pzero} \\
  (\mathsf{new}\; n)\pzero = \pzero \\
  \mathsf{!}\pzero = \pzero \\
\end{mathpar}

\subsubsection{Operational semantics}
\begin{mathpar}
  \inferrule* [lab=in]{}{\binpar{n[\binpar{\mathsf{in}\; m\mathsf{.}P}{Q}]}{m[R]} \red m[\binpar{n[\binpar{P}{Q}]}{R}]}
  \and
  \inferrule* [lab=out]{}{m[\binpar{\binpar{n[\binpar{\mathsf{out}\; m\mathsf{.}P}{Q}]}}{R}] \red \binpar{n[\binpar{P}{Q}]}{m[R]}}
  \and
  \inferrule* [lab=open]{}{\binpar{\mathsf{open}\; n\mathsf{.}P}{n[Q]} \red \binpar{P}{Q}} \\
  \and
  \inferrule* [lab=new]{P \red P'}{(\mathsf{new})P \red (\mathsf{new})P'}
  \and
  \inferrule* [lab=ambient]{P \red P'}{n[P] \red n[P']}
  \and
  \inferrule* [lab=par]{P \red P'}{\binpar{P}{Q} \red \binpar{P'}{Q}} \\
  \and
  \inferrule* [lab=struct]{P = P', P' \red Q', Q' = Q}{P \red Q}
\end{mathpar}

\subsection{Encoding}
There are two important ideas. The first is that we can model an
ambient $n[P]$ as $\outputp{\meaningof{n}}{\meaningof{P}}$. The second
is that we can encode contexts as higher order communication. So, we
are never limited to a single context.

\subsection{Full abstraction}
\begin{theorem}
  $P \wbbisim Q \iff \meaningof{P} \wbbisim \meaningof{Q}$
\end{theorem}


