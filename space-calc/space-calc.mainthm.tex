\section{Main theorem}
TBD
%% The Liskov substitution principle \cite{DBLP:conf/oopsla/Liskov87} is
%% vitally important for a type systems, especially if we want our type
%% system to provide guidance for automated refactoring and other kinds
%% of support we might expect of a type-directed interactive design
%% environment, such as IntelliJ or Eclipse. In Milner's polyadic
%% $\pi$-calculus tutorial \cite{milner91polyadicpi} he demonstrates
%% exactly the sort of theorem for his logic that we need to formalize
%% Liskov's principle.

%% \begin{mathpar}
%%   P \wbbisim Q \iff \forall \phi.( P \vDash \phi \iff Q \vDash \phi)
%% \end{mathpar}

%% As Caires points out, spatial connectives can see structural
%% properties that Milner's logic cannot. In particular, bisimulation
%% cannot distinguish parallel composition from the summ of all
%% interleavings. Fortunately, by including proved versions of the
%% structural equalities we recover Milner's theorem.

%% \begin{theorem}[substitutability]
%%   \begin{mathpar}
%%     P \wbbisim Q \iff \forall (\mathbf{U},\mathbf{X}).( P : (\mathbf{U},\mathbf{X}) \iff Q : (\mathbf{U},\mathbf{X}))
%%   \end{mathpar}
%% \end{theorem}
