\subsection{The syntax and semantics of the notation system}\label{sub:the_syntax_and_semantics_of_the_notation_system} % (fold)

We now summarize a technical presentation of the calculus that
embodies our theory of dynamics. The typical presentation of such a
calculus follows the style of giving generators and relations on
them. The grammar, below, describing term constructors, freely
generates the set of processes, $\Proc$. This set is then quotiented
by a relation known as structural congruence and it is over this set
that the notion of dynamics is expressed. This presentation is
essentially that of \cite{MeredithR05} with the addition of
polyadicity and summation. For readability we have relegated some of
the technical subtleties to an appendix.

\subsubsection{Process grammar}\label{subsub:process_grammar}

\begin{mathpar}
\inferrule* [lab=process] {} {P, Q \bc \pzero \;\bm\; \mathsf{for}( \arrvec{y} \leftarrow x )P \;\bm\; x\mathsf{!}(\arrvec{Q}) \;\bm\; \mathsf{*}x \;\bm\; P\mathsf{|}Q }
\and
\inferrule* [lab=name] {} {x, y \bc \mathsf{@}P }
\end{mathpar}

Note that $\arrvec{x}$ (resp. $\arrvec{P}$) denotes a vector of names
(resp. processes) of length $|\arrvec{x}|$ (resp. $|\arrvec{P}|$). We adopt
the following useful abbreviations.

\begin{mathpar}
  \Pi \arrvec{P} := \Pi_{i=1}^{|\arrvec{P}|}P_i := P_1 | \ldots | P_{|\arrvec{P}|}
\end{mathpar}

\begin{definition}
\emph{Free and bound names} The calculation of the free names of a
process, $P$, denoted $\freenames{P}$ is given recursively by

\begin{mathpar}
  \freenames{\pzero} = \emptyset
  \and
  \freenames{\mathsf{for}(\arrvec{y} \leftarrow x)(P)} = \{ x \} \cup \freenames{P}\setminus\{\arrvec{y}\}
  \and
  \freenames{x!(\arrvec{P})} = \{ x \} \cup \freenames{\arrvec{P}}
  \and
  \freenames{P|Q} = \freenames{P} \cup \freenames{Q}
  \and
  \freenames{\mathsf{}{x}} = \{ x \} \\
\end{mathpar}

where

\begin{mathpar}
  \{\arrvec{x}\} := \{x_1, \ldots, x_{|\arrvec{x}|}\}
  \and \freenames{\arrvec{P}} := \bigcup \freenames{P_i}
\end{mathpar}

An occurrence of $x$ in a process $P$ is \textit{bound} if it is not
free. The set of names occurring in a process (bound or free) is
denoted by $\names{P}$.
\end{definition}

\subsection{Substitution}

We use $\Proc$ for the set of processes, $\QProc$ for the set of
names, and $\id{\{}\arrvec{y} / \arrvec{x} \id{\}}$ to denote partial maps,
$s : \QProc \rightarrow \QProc$. A map, $s$ lifts, uniquely, to a map
on process terms, $\widehat{s} : \Proc \rightarrow \Proc$ by the
following equations.

\begin{mathpar}
  (0) \psubstp{Q}{P} := 0 \\
  (R \juxtap S) \psubstp{Q}{P}
  :=    
  (R)\psubstp{Q}{P} \juxtap (S) \psubstp{Q}{P} \\
  (\mathsf{for}(\arrvec{y} \leftarrow x)R) \psubstp{Q}{P}    
  :=
  \mathsf{for}(\arrvec{z} \leftarrow (x)\substp{Q}{P})((R \psubstn{\arrvec{z}}{\arrvec{y}}) \psubstp{Q}{P}) \\
  (\lift{x}{R}) \psubstp{Q}{P}  
  :=
  \lift{(x)\substp{Q}{P}}{ R \psubstp{Q}{P} } \\
%   (\dropn{x})  \psubstp{Q}{P}       
%   := 
%   \left\{ 
%     \begin{array}{ccc} 
%       \dropn{\quotep{Q}} & & x \nameeq \quotep{P} \\
%       \dropn{x} & & otherwise \\
%     \end{array}
%   \right. 
  (\dropn{x})  \psubstp{Q}{P}       
  := 
  \left\{ 
    \begin{array}{ccc} 
      Q & & x \nameeq \quotep{P} \\
      \dropn{x} & & otherwise \\
    \end{array}
  \right.
\end{mathpar} 

where

\begin{eqnarray}
  (x)\id{\{} \quotep{Q} / \quotep{P} \id{\}}            = 
  \left\{ 
    \begin{array}{ccc}
      \quotep{Q} & & x \nameeq \quotep{P} \\
      x & & otherwise \\
    \end{array}
  \right. \nonumber
\end{eqnarray}

and $z$ is chosen distinct from $\quotep{P}$, $\quotep{Q}$, the free
names in $Q$, and all the names in $R$. Our $\alpha$-equivalence will
be built in the standard way from this substitution.

\begin{definition}
Then two processes, $P,Q$, are alpha-equivalent if $P = Q\{\arrvec{y}/\arrvec{x}\}$ for
some $\arrvec{x} \in \boundnames{Q},\arrvec{y} \in \boundnames{P}$, where $Q\{\arrvec{y}/\arrvec{x}\}$
denotes the capture-avoiding substitution of $\arrvec{y}$ for $\arrvec{x}$ in $Q$.
\end{definition}

\begin{definition}
  The {\em structural congruence} $\equiv$
  between processes \cite{SangiorgiWalker} is the least congruence containing
  alpha-equivalence and satisfying the commutative monoid laws
  (associativity, commutativity and $\pzero$ as identity) for parallel
  composition $|$.
\end{definition}

\begin{definition}
  The {\em name equivalence} $\nameeq$ is the least congruence
  satisfying these equations
  \begin{mathpar}
  \inferrule*[lab=Quote-drop] {}{ \quotep{\dropn{x}} \;\nameeq\; x }
  \and
  \inferrule*[lab=Struct-equiv] { P \;\scong\; Q } { \quotep{P} \;\nameeq\; \quotep{Q} }
  \end{mathpar}
\end{definition}

The astute reader will have noticed that the mutual recursion of names
and processes imposes a mutual recursion on alpha-equivalence and
structural equivalence via name-equivalence. Fortunately, all of this
works out pleasantly and we may calculate in the natural way, free of
concern. The reader interested in the details is referred to the
appendix \ref{appendix:rho_details}.

\begin{remark}\label{rem:no_self_referential_names}
  One particularly useful consequence of these definitions is that
  $\forall P. \quotep{P} \not\in \freenames{P}$. It gives us a
  succinct way to construct a name that is locally free in $P$.
\end{remark}

Finally equipped with these standard features we can present the
dynamics of the calculus.

\subsection{Operational semantics}

Finally, we introduce the computational dynamics. What marks these
algebras as distinct from other more traditionally studied algebraic
structures, e.g. vector spaces or polynomial rings, is the manner in
which dynamics is captured. In traditional structures, dynamics is typically
expressed through morphisms between such structures, as in linear maps
between vector spaces or morphisms between rings. In algebras
associated with the semantics of computation, the dynamics is
expressed as part of the algebraic structure itself, through a
reduction reduction relation typically denoted by $\red$. Below, we
give a recursive presentation of this relation for the calculus used
in the encoding.

\begin{mathpar}
  \inferrule* [lab=COMM] {x_{t} \;\nameeq\; x_{s}, \;\;\; |\arrvec{y}| = |\arrvec{Q}|} {\mathsf{for}( \arrvec{y} \leftarrow x_{t} )P \;\mathsf{|}\; x_{s}!(\arrvec{Q})
  \red P\substn{\arrvec{\quotep{Q}}}{\arrvec{y}}}
  \and
  \inferrule* [lab=PAR]{P \red P'}{P\mathsf{|}Q \red P'\mathsf{|}Q}
  \and
  \inferrule* [lab=EQUIV]{{P \;\scong\; P'} \andalso {P' \red Q'} \andalso {Q' \;\scong\; Q}}{P \red Q}
\end{mathpar}

We write $P\red$ if $\exists Q $ such that $ P \red Q$ and $P\not\red$, otherwise.

\subsection{ Dynamic quote: an example }

Anticipating something of what's to come, let $z = \quotep{P}$, $u = \quotep{Q}$, and $x = \quotep{\lift{y}{\dropn{z}}}$. Now consider applying the substitution,
$\widehat{\id{\{}u / z \id{\}}}$, to the following pair of processes,
$\lift{w}{y!(\dropn{z})}$ and $\lift{w}{\dropn{x}} = \lift{w}{\dropn{\quotep{\lift{y}{\dropn{z}}}}}$.

\begin{eqnarray}
	\lift{w}{\lift{y}{\dropn{z}}}\widehat{\id{\{}u / z \id{\}}}
		& = &
		\lift{w}{\lift{y}{Q}} \nonumber\\
	\lift{w}{\dropn{x}} \widehat{ \id{\{}u / z \id{\}} }
		& = &
		\lift{w}{\dropn{x}} \nonumber
\end{eqnarray}

The body of the quoted process, $\quotep{\lift{y}{\dropn{z}}}$, is
impervious to substitution, thus we get radically different
answers. In fact, by examining the first process in an input context,
e.g. $\mathsf{for}(z \leftarrow x)\lift{w}{\lift{y}{\dropn{z}}}$, we see that the process
under the output operator may be shaped by prefixed inputs binding a
name inside it. In this sense, the combination of input prefix binding
and output operators will be seen as a way to dynamically construct
processes before reifying them as names.

\section{Replication}

As mentioned before, it is known that replication (and hence
recursion) can be implemented in a higher-order process algebra
\cite{SangiorgiWalker}. As our first example of calculation with the
machinery thus far presented we give the construction explicitly in
the {\rhoc}.

\begin{eqnarray}
	D_{x} & := & \prefix{x}{y}{(\binpar{\outputp{x}{y}}{\dropn{y}})} \nonumber\\
	\bangp_{x}{P} & := & \binpar{\lift{x}{\binpar{D_{x}}{P}}}{D_{x}} \nonumber
\end{eqnarray}

\begin{eqnarray}
	\bangp_{x}{P} & & \nonumber\\
	=
	& \lift{x}{(\prefix{x}{y}{(\outputp{x}{y} | \dropn{y})) | P}} 
	      | \prefix{x}{y}{(\outputp{x}{y} | \dropn{y})} & \nonumber\\
	\red
	& (\outputp{x}{y} | \dropn{y})\substn{\quotep{(\prefix{x}{y}{(\dropn{y} | \outputp{x}{y})) | P}}}{y} & \nonumber\\
	=
	& \outputp{x}{\quotep{(\prefix{x}{y}{(\outputp{x}{y} | \dropn{y})) | P}}}
	  | {(\prefix{x}{y}{(\outputp{x}{y} | \dropn{y})) | P}} & \nonumber\\
	\red
	& \ldots & \nonumber\\
	\red^*
	& P | P | \ldots & \nonumber
\end{eqnarray}

Of course, this encoding, as an implementation, runs away, unfolding
$\bangp{P}$ eagerly. A lazier and more implementable replication
operator, restricted to input-guarded processes, may be obtained as follows.

\begin{eqnarray}
\bangp{\prefix{u}{v}{P}} 
	:= 
	\binpar{\lift{x}{\prefix{u}{v}{(\binpar{D(x)}{P})}}}{D(x)} \nonumber
\end{eqnarray}

\begin{remark}
  Note that the lazier definition still does not deal with summation
  or mixed summation (i.e. sums over input and output). The reader is
  invited to construct definitions of replication that deal with these
  features. 

  Further, the definitions are parameterized in a name, $x$. Can you,
  gentle reader, make a definition that eliminates this parameter and
  guarantees no accidental interaction between the replication
  machinery and the process being replicated -- i.e. no accidental
  sharing of names used by the process to get its work done and the
  name(s) used by the replication to effect copying. This latter
  revision of the definition of replication is crucial to obtaining
  the expected identity $!!P \sim !P$.
\end{remark}

\begin{remark}\label{rem:paradoxical_combinator}
  The reader familiar with the lambda calculus will have noticed the
  similarity between $D$ and the paradoxical combinator.

  [Ed. note: the existence of this seems to suggest we have to be more
  restrictive on the set of processes and names we admit if we are to
  support no-cloning.]
\end{remark}

\subsubsection{Bisimulation}

The computational dynamics gives rise to another kind of equivalence,
the equivalence of computational behavior. As previously mentioned
this is typically captured \emph{via} some form of bisimulation.

% The notion we use in this paper is weak barbed bisimulation
% \cite{milner91polyadicpi}.

The notion we use in this paper is derived from weak barbed
bisimulation \cite{milner91polyadicpi}. 

\begin{definition}
An \emph{observation relation}, $\downarrow_{\mathcal N}$, over a set
of names, $\mathcal N$, is the smallest relation satisfying the rules
below.

\infrule[Out-barb]{y \in {\mathcal N}, \; x \nameeq y}
		  {\outputp{x}{v} \downarrow_{\mathcal N} x}
\infrule[Par-barb]{\mbox{$P\downarrow_{\mathcal N} x$ or $Q\downarrow_{\mathcal N} x$}}
		  {\binpar{P}{Q} \downarrow_{\mathcal N} x}

We write $P \Downarrow_{\mathcal N} x$ if there is $Q$ such that 
$P \wred Q$ and $Q \downarrow_{\mathcal N} x$.
\end{definition}

\begin{definition}
%\label{def.bbisim}
An  ${\mathcal N}$-\emph{barbed bisimulation} over a set of names, ${\mathcal N}$, is a symmetric binary relation 
${\mathcal S}_{\mathcal N}$ between agents such that $P\rel{S}_{\mathcal N}Q$ implies:
\begin{enumerate}
\item If $P \red P'$ then $Q \wred Q'$ and $P'\rel{S}_{\mathcal N} Q'$.
\item If $P\downarrow_{\mathcal N} x$, then $Q\Downarrow_{\mathcal N} x$.
\end{enumerate}
$P$ is ${\mathcal N}$-barbed bisimilar to $Q$, written
$P \wbbisim_{\mathcal N} Q$, if $P \rel{S}_{\mathcal N} Q$ for some ${\mathcal N}$-barbed bisimulation ${\mathcal S}_{\mathcal N}$.
\end{definition}

\subsubsection{Contexts}

One of the principle advantages of computational calculi from the
$\lambda$-calculus to the $\pi$-calculus is a well-defined notion of context,
contextual-equivalence and a correlation between
contextual-equivalence and notions of bisimulation. The notion of
context allows the decomposition of a process into (sub-)process and
its syntactic environment, its context. Thus, a context may be
thought of as a process with a ``hole'' (written $\Box$) in it. The
application of a context $K$ to a process $P$, written $K[P]$, is
tantamount to filling the hole in $K$ with $P$. In this paper we do
not need the full weight of this theory, but do make use of the notion
of context in the proof the main theorem. 

\begin{mathpar}
\inferrule* [lab=context] {} {K \bc \Box \;\bm\; \mathsf{for}( \arrvec{y} \leftarrow x )K \;\bm\; x\mathsf{!}(\arrvec{P},K,\arrvec{Q}) \;\bm\; K\mathsf{|}P }
\end{mathpar}

\begin{definition}[contextual application] Given a context $K$, and
  process $P$, we define the \emph{contextual application}, $K[P] :=
  K\{P/\Box\}$. That is, the contextual application of K to P is the
  substitution of $P$ for $\Box$ in $K$.
\end{definition}

\subsubsection{Contextual duality}

Note that contexts extend the quotation operation to a family of
operations from processes to names. Given a context, $K$, we can
define a \emph{nominal context}, $\quotep{M}$ by $\quotep{K}[P] :=
\quotep{(K[P])}$. To foreshadow what is to come we observe that these
operations enjoy a duality with processes very much like the duality
between vectors and maps from vectors to scalars.

Further, because the calculus is essentially higher-order, we have a
correspondence between contexts and processes. More specifically,
given a name $x$ and a context $K$ we can construct $K^{*}_{x}$ such
that 

\begin{mathpar}
  K^{*}_{x} | \lift{x}{P} \red K[P]
\end{mathpar}

namely,

\begin{mathpar}
  K^{*}_{x} := \prefix{x}{y}{K[\dropn{y}]}
\end{mathpar}

This correspondence mirrors the usual correspondence between vectors and duals, where given a vector $v$ we can produce its dual $w \mapsto v \cdot w$ taking a vector $w$ to the dot product $v \cdot w$.

%% The dependence of $M^{*}_{x}$ on a name makes it an abstraction, 

%% \begin{mathpar}
%%   M^{*} := (x)x?(u).M[\dropn{u}]
%% \end{mathpar}

\subsection{Additional notation}

We achieve some not insignificant notational compression with the following convention

\begin{mathpar}
  \mathsf{for(}{y_{1}}\leftarrow{x_{1}}\mathsf{;}\;\ldots\mathsf{;}\;{y_{n}}\leftarrow{x_{n}}\mathsf{)}P := \prefix{x_{1}}{y_{1}}{\prefix{x_{2}}{y_{2}}{\ldots \prefix{x_{n}}{y_{n}}{P}}}
\end{mathpar}

Even though we already have the notation $x = \quotep{P}$, it will be
convenient to introduce an alternate notation, $\procn{x}$, when we
want to emphasize the connection to the use of the name. Note that, by
virtue of name equivalence, $\quotep{\procn{x}} \nameeq x$; so, the
notation is consistent with previous definitions.

Further, because names have structure it is possible to effect
substitutions on the basis of that structure. This means we need to
upgrade our notation for substitutions, which we accomplish by
adapting comprehension notation. Thus,

\begin{mathpar}
  P\{ y / x : x \in S \}
\end{mathpar}

is interpreted to mean the process derived from P by replacing (in a
capture-avoiding manner) each occurrence of $x$ in $S$ by $y$. For example,

\begin{mathpar}
  P\{ \quotep{(\procn{x}\mathsf{|}\procn{x})} / x : x \in \freenames{P} \}
\end{mathpar}

will replace each (occurrence) of a free name $x$ in $P$ by
$\quotep{(\procn{x}\mathsf{|}\procn{x})}$.

Also, we will avail ourselves of the notation $x^{L}$ and $x^{R}$ to
denote injections of a name into disjoint copies of the name
space. There are numerous ways to accomplish this. One example can be
found in \cite{MeredithR05}. This notation overloads to vectors of
names: $\arrvec{x}^{\pi} := (x_{i}^{\pi} \; : \; 0 \leq i < |\arrvec{x}| )$ where $\pi \in \{L,R\}$.

We also use $P^{\Box} := P|\Box$.

In \cite{MeredithR05} an interpretation of the new operator is
given. It turns out that there are several possible interpretations
all enjoying the requisite algebraic properties of the operator (see
\cite{milner91polyadicpi}). We will therefore make liberal use of
$(\nu\; \arrvec{x})P$.

\subsection{Extensions to the calculus}

The presentation given so far is often referred to as the polyadic,
asynchronous version of the rho-calculus. The current investigation is
made simpler if we expand to a version of the calculus that includes
mixed summation, that is non-deterministic choice over both guarded
input ($\mathsf{for}$-comprehension) and output. Because Milner's
presentation of the polyadic $\pi$-calculus with mixed summation is so
parsimonious we use it as a template for a similar version of the
rho-calculus.

\begin{mathpar}
  \inferrule* [lab=summation] {} {{M,N} \bc \pzero \;|\; x.A \;|\; M+N}
  \and
  \inferrule* [lab=agent] {} {A \bc (\arrvec{x})P \;| \; [\arrvec{P}]Q}
  \and \\
  \inferrule* [lab=process] {} {P,Q \bc M \;|\; P|Q \;|\; \mathsf{*}x}
  \and
  \inferrule* [lab=name] {} {x \bc \mathsf{@}P}
\end{mathpar}

In this presentation we adopt the syntactic conventions

\begin{mathpar}
  \mathsf{for(}\arrvec{y} \leftarrow x\mathsf{)}P := x.(\arrvec{y})P
  \and
  x\mathsf{!}(\arrvec{Q})\mathsf{;}P := x.[\arrvec{Q}]P
\end{mathpar}

The structural equivalence is modified thusly.

\begin{definition}
  The {\em structural congruence} $\equiv$ between processes is the
  least congruence containing alpha-equivalence and satisfying the
  commutative monoid laws (associativity, commutativity and $\pzero$
  as identity) for parallel composition $|$ and summation $+$.
\end{definition}

The $\mathsf{COMM}$ rule is modified to incorporate non-deterministic choice.

\begin{mathpar}
  \inferrule* [lab=COMM] {x_{t} \;\nameeq\; x_{s}, \;\;\; |\arrvec{y}| = |\arrvec{Q}|} {\mathsf{for}( \arrvec{y} \leftarrow x_{t} )P + R_1 \;\mathsf{|}\; x_{s}!(\arrvec{Q}).P' + R_2
  \red P\substn{\arrvec{\quotep{Q}}}{\arrvec{y}}\mathsf{|}P'}
\end{mathpar}

And contexts are likewise extended in the obvious manner.

\begin{mathpar}
  \inferrule* [lab=summation] {} {{K_{M}} \bc \Box \;|\; x.K_{A} \;|\; K_{M}+M}
  \and
  \inferrule* [lab=agent] {} {K_{A} \bc (\arrvec{x})K_{P} \;|\; [\arrvec{P},K_{P},\arrvec{P'}]Q \;|\; [\arrvec{P}]K_{P}}
  \and \\
  \inferrule* [lab=process] {} {K_{P} \bc K_{M} \;|\; K_{P}|P}
\end{mathpar}

The reader can check that all the notational conventions, such as $\mathsf{for(}{y_{1}}\leftarrow{x_{1}}\mathsf{;}\;\ldots\mathsf{;}\;{y_{n}}\leftarrow{x_{n}}\mathsf{)}P$, adopted above still make sense.

% subsection the_syntax_and_semantics_of_the_notation_system (end)   
