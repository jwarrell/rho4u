\section{Introduction}\label{sec:introduction} % (fold)
For the past several centuries there has been no serious competitor to
the ``Newtonian'' account of dynamics in the physical sciences. By the
Newtonian account of dynamics we means formulations guided by two principles:
\begin{itemize}
  \item the form of physical laws are given as laws of motion plus initial state;
  \item the laws of motion are stated in terms of the mathematical
    apparati of the integral and derivative and the principles of
    applications of those apparati to modeling dynamical systems, such
    as the Lagrangian and Hamiltonian \cite{375178}.
\end{itemize}
In recent years several distinguished physicists have begun to
question the scope and applicability of the first principle. Notably
David Deutsch and his collaborator Chiara Marletto have begun to ask
whether laws of physics might be better formulated as
counterfactuals. Their motivations are informed by considerations
going as far back as John Wheeler's suggestions that physics might
derive from information processing, it arising from bit, as it
were. As Deutsch and Marletto point out, the challenge in attempting
to carry out such a program is that in modern physics information is
\emph{emergent}, and not a quantity on par with properties like
position or momentum.

A similar, but perhaps more pointed critique can levied against the
second principal. It is common place to render versions of the laws of
physics in a programming language. From predictive models for particle
colliders to physics engines for games this activity has been standard
practice for decades. Yet, no one has rendered even the most basic of
programming models in the laws of physics \emph{considered as an
  abstract language}. Even quantum computation introduces abstractions
like quantum gates or the ZX-calculus. The standard model is not seen
as a programming paradigm. Programming paradigms, on the other hand,
are seen as just the right tool for rendering the full range of
dynamical abstractions from the Lagrangian and Hamiltonian to
perturbative techniques into a form where we can gain concrete
computational insights. Could it be that computation is better suited
as the language of dynamics, including physical dynamics, and not just
the dynamics of information?

We might even go so far as to speculate whether it might be
intellectually dangerous to limit our view of dynamics to one
abstraction. Several examples in the history of physics suggest it is
appropriate to worry about this. Feynman diagrams introduced a new
abstraction that dramatically expanded the scope of what could be
calculated and hence grasped. At their inception, and for years after,
well known physicists dismissed them as mere syntax, not
physics. Likewise, as Coecke and Kissinger point out, it took \emph{60
  years} from the discovery of quantum mechanics to the formulation of
quantum teleportation. Yet, when rendered in string diagrams, the
phenomenon is dead obvious and would have been spotted by a high
school student performing routine calculations. Both developments
share the same underlying drive: to formulate a clear computational
framework that expands what is concretely amenable to
calculation. They suggest an intriguing idea: what if syntax is an
integral part of physics? Indeed of all science.

It is from this perspective that we seek to explore new developments
in computing as a potential syntax for physical dynamics. In
particular, these new theories of computing offer new ways of thinking
about how computation, and by extension the physical world, is
organized. They re-imagine computation as interaction. Instead of
(active) functions acting on (passive) data, resulting in new,
(transformed) data -- or (active) forces acting on (passive) particles
-- these models view computation as the \emph{patterns of interaction}
amongst \emph{autonomous} agents. The view distributes agency
throughout the elements comprising a computation. Further, these
interactive models embody principles not found in traditional accounts
of physical dynamics. Among these, perhaps the most important, is
\emph{compositionality}.

This principle is innocent enough. It asks that our models of system
dynamics be built of the models of the dynamics of their
subsystems. This requirement has two important aspects: one, that we
can compose models to get new models of increasing complexity; and,
two, that we can decompose more complex models into models of reduced
complexity. But, it is not just about composing models. It must be the
case that the composition of models faithfully represents the
composition of the systems they represent. Likewise, if a model does
decompose, it must correspond to a natural decomposition of the system
it represents. This is not a property enjoyed, for example, by
differential equation models of stoichiometry. If we have a model of
the reactions in beaker A, and a model of the reactions in beaker B,
we cannot simply combine the equations of the model of A with the
equations of beaker B to get a model of the system obtained by pouring
the contents of A and B into a third beaker. In fact, differential
models often fail this particular test.

Challenges to classical accounts of physics as deep as General
Relativity \cite{Gravitation} and Quantum mechanics \cite{Dirac1930}
are still \emph{essentially} couched in the methods and procedures
derived from the Newtonian integral and derivative \footnote{The
  covariant derivative still bears the imprimature of the original
  notion. Inner product in the wave-function formulation is still an
  integral}. Meanwhile, the very successful program of the ZX-calculus
emphasizes compositionality to great benefit. The current work is an
attempt to push that particular envelope by encoding the conceptual
core of quantum mechanics into a calculus for concurrent computing
called the rho-calculus.

%% Recently, however, the theory of computation has matured
%% to the point where we have candidates for theories of dynamics that
%% offer very different perspectives on reasoning about dynamical systems
%% and situations. It is time to see just how good they are by addressing
%% the dynamics of interest in the physical sciences. Testing these
%% candidates against very successful accounts of dynamical situations,
%% like quantum mechanics, is going to give us some sense of maturity,
%% quality and range of applicability of computational notions of
%% dynamics.

%% To address this program we realize and carry out the calculations of
%% quantum mechanics over a novel formal theory of dynamics, a formal
%% theory of dynamics that corresponds to a theory of concurrent
%% computation with \emph{reflection}
%% \cite{bcsmith:phdthesis},\cite{319871}. It has the advantage that the
%% underlying theory is already `quantized', but supports analogues all
%% of the continuuous operations. Strikingly, this underlying theory has
%% recently been connected with a notion of metric
%% \cite{DBLP:conf/fossacs/BreugelMOW03} that coincides with the metric
%% induced by the inner product.

\subsection{Summary of contributions and outline of paper}

%% Rather,
%% we are developing a story that aligns with Wheeler's slogan: It from
%% Bit \cite{HowComeTime}. To do this we will first provide an account of
%% the theory of computation at play here. Then we will dive into a
%% calculation-driven interpretation of the operations of quantum
%% mechanics.

%% The reason we take this approach is that -- until very recently
%% (\cite{DBLP:conf/lics/Abramsky04},
%% \cite{DBLP:conf/calco/Abramsky05})-- there hasn't been an axiomatic
%% account of quantum mechanics. As a result there has been no sharp
%% delineation of the mathematical theory supporting interpretation of
%% the physical theory and the physical theory, itself. So, ambient
%% features of the maths are free to be exploited (or suppressed) without
%% a real accounting of their physical relevance. There is no sharp
%% statement ``here's the physical theory'' qua \emph{theory} and
%% ``here's the mathematical interpretation'' enabling a judgment of how
%% faithful the interpretation is -- apart from experimental
%% observation. When there is an axiomatic account we can judge how well
%% a given mathematical formalism supports an interpretation of the
%% axioms, independent of experimentation. Likewise, we can judge how
%% well we have captured our physical evidence and experience with our
%% axiomatics, independent of any specific mathematical implementation,
%% with accidental detail that may or may not have physical significance.

%% In lieu of a fully fleshed out and vetted axiomatic account of quantum
%% mechanics, interpreting the operational notions in service of modeling
%% physical systems will have to suffice. In other words, we are not in
%% the business of providing a model of Hilbert spaces and operators. We
%% are in the business of providing a model of quantum mechanics because
%% we are motivated by testing our notions of dynamics against physical
%% theory; and, the predictive calculations of the physical theory must
%% serve as the best formulation -- shy of a fully fleshed out axiomatic
%% account -- of the physical theory itself (as they have for scientific
%% theories since time immemorial). Put another way, despite a
%% whole-hearted commitment to an It-from-Bit ontology, we are firmly
%% aligned with the shut-up-and-calculate camp as the best way to obtain
%% results either from the physical perspective or as a quality assurance
%% measure of our fledgling theory of dynamics.

The plan is to develop an interpretation of the operations of quantum
mechanics normally interpreted by Hilbert spaces and operators
\cite{MathematicalFoundationsQM}. The primary novelty of the approach
is to do this over a theory of computation. Note that this is very
different than the usual quantum computation program which develops
notions of computation over quantum mechanics \cite{544199}. In
detail, we present a reflective process calculus. Then we develop
intuitive correspondences between the notions available in this
calculus and the usual physical notions supporting quantum mechanical
calculations.

\begin{table}[htp]
  \center{
    \fbox{
      \begin{tabular}{c|c}
        quantum mechanics & process calculus \\
        \hline
        scalar & name \\
        state vector & process \\
        dual & contextual duals \\
        matrix & formal sums of process-context-dual pairs \\
        orthogonality & process annihilation \\
        inner product & parallel composition + quoting
      \end{tabular}
    }
  }
  \caption{QM - process calculi correspondences}
\end{table}

Then we tighten up these intuitions to operational definitions. We
employ the Dirac notation as the best proxy we can find for an
abstract syntax of the quantum mechanical notions that also makes
contact with notations that working physicists understand. The
definitions we develop put us in contact with equational constraints
coming from the theory, and we demonstrate the definitions and
calculations satisfy.

This puts us in a position to shut up and calculate for the
Stern-Gerlach experimental set up, showing how these predictive
calculations become calculations on processes in our theory of a
reflective process calculus. The particular example is of interest
because it is a composite experiment, consisting of iterated
measurements. The framework developed here is particularly suited to
reasoning about composite physical systems and situations.

Apart from the novelty of reflective apparatus of the rho-calculus,
one must ask how it can span the gap between classical computation and
quantum computation. The key idea is the abstraction of
non-determinism. The calculus leaves to various models to supply a
notion of non-determinism. Thus, the same framework maps to classical
computing when non-determinism is mapped to a race-condition. It maps
to stochastic models when non-determinism is governed by a 1-norm
probability distribution. It maps to quantum computation with
non-determinism is governed by a 2-norm probability distribution, thus
allowing for negative probabilities which then set up interference in
the causal evolution of a computation.

%% Penultimately, we demonstrate that the notion of metric coming from
%% the inner product coincides with the notion of metric available from
%% the theory of bisimulation. This demonstration gives us the right to
%% think of space as arising from behavior. Finally, we consider where we
%% might go from the new vantage point we have obtained.

% section introduction (end)
